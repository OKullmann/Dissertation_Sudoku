% Yu-Ting Lu 24/7/2011 (Swansea)

\documentclass[11pt]{report}

\begin{document}

\title{Sudoku - Reference}
\author{Yu-Ting Lu}
\date{July 2011}
\maketitle

\tableofcontents

Sudoku is a puzzle which is popularized into the world for ages. And there are amount of people spend times on either hard puzzle generated, or looking for the best techniques to improve the efficiency on solving difficult puzzle.
When the puzzle difficulty becomes harder and harder, then more advanced techniques will be required to be used together or in order to cope the puzzle. Surely, as the difficulties level goes up, the cost of time raise up even thought the suitable technique is given to help. Consequently, in this article, the first chapter is going to know the background of Sudoku and what is Sudoku, and the secondly understand discussed after understanding how these special techniques applied on eliminating impossibilities in puzzle solving. Afterwards, the efficiency of the special technique for solving Sudoku will be the final topic to discuss about in the end.
In this dissertation, the goal to achieve is to experience the time of applying special techniques on solving Sudoku puzzles. In addition, to find out the most efficient way to apply the combination of more than techniques and how many puzzles could it achieve to solve.
\section{What is sudoku?}
Sudoku is a puzzle which contains N (N is equal to nxn) number of nxn sub-grids to compose to a Sudoku puzzle for digits to be filled in rows, columns and the blocks. Each digit has a unique number selected from 1 to 9 and not repeated in the same row, column or the block.
And to find the unique solution of the puzzle,  ~\cite{Berthier2007Sudoku}.
The classical Sudoku has these rules applied.

1. Number from 1 to 9 has to appear at least once in a row, a column, and a block.

2. Number from 1 to 9 has to appear at most once in a row, a column, and a block.

3. All the grids have to be no empty left in the end.

4. All Sudoku puzzles must have a solution.

5. Every Sudoku puzzle has a unique solution only.
\section
{Relationship between rows, columns, and blocks.}
\newpage
\bibliography{mybib}{}
\bibliographystyle{plain}
\end{document}
