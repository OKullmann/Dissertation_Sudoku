% Yu-Ting Lu 24/7/2011 (Swansea)

\documentclass[11pt]{report}

\usepackage{sudoku}


\begin{document}

\title{Sudoku Patterns}
\author{Yu-Ting Lu}
\date{July 2011}
\maketitle

\tableofcontents



\chapter{Introduction}
\label{cha:Introduction}


Sudoku is a puzzle which is popularised into the world for ages. And there are amount of people spend times on either hard puzzle generated, or looking for the best techniques to improve the efficiency on solving difficult puzzle.
When the puzzle difficulty becomes harder and harder, then more advanced techniques will be required to be used together or in order to cope the puzzle. Surely, as the difficulties level goes up, the cost of time raise up even thought the suitable technique is given to help. Consequently, in this article, the first chapter is going to know the background of Sudoku and what is Sudoku, and the secondly understand discussed after understanding how these special techniques applied on eliminating impossibilities in puzzle solving. Afterwards, the efficiency of the special technique for solving Sudoku will be the final topic to discuss about in the end.
In this dissertation, the goal to achieve is to experience the time of applying special techniques on solving Sudoku puzzles. In addition, to find out the most efficient way to apply the combination of more than techniques and how many puzzles could it achieve to solve.


\section{What is Sudoku?}
\label{sec:whatissudoku}

Sudoku uses an $N \times N$ matrix, where $N = n \cdot n$, and for standard Sudoku we have $n = 3$, thus $N = 9$. Each entry (``cell'') contains a number in $\{1, \dots, N\}$, or it might be empty, that is, to be filled out. A completed Sudoku puzzle must fulfil the following additional requirements:
\begin{enumerate}
\item In every row and in every column every number occurs only (exactly) once.
\item The same is true for the ``blocks'':
  \begin{enumerate}
  \item The whole matrix is sub-divided into $N$ blocks.
  \item Each block is an $n \times n$ matrix.
  \end{enumerate}
  Now in each block every number has also to occur (exactly) once.
\end{enumerate}
For ``classical Sudoku'' puzzles we have the following additonal rules:
\begin{enumerate}
\item A Sudoku puzzle must have at least one solution (no unsolvable puzzles are normally considered).
\item And in fact, a Sudoku puzzle must have a unique solution (no multiple solutions are normally allowed).
\end{enumerate}
An example:

\setlength\sudokusize{8cm}
\begin{figure}
\begin{sudoku}
 |2|5| | |3| |9| |1|.
 | |1| | | |4| | | |.
 |4| |7| | | |2| |8|.
 | | |5|2| | | | | |.
 | | | | |9|8|1| | |.
 | |4| | | |3| | | |.
 | | | |3|6| | |7|2|.
 | |7| | | | | | |3|.
 |9| |3| | | |6| |4|.
\end{sudoku}
\caption{Sudoku Problem}
\end{figure}

\begin{figure}
\begin{sudoku}
  |2|5|8|7|3|6|9|4|1|.
  |6|1|9|8|2|4|3|5|7|.
  |4|3|7|9|1|5|2|6|8|.
  |3|9|5|2|7|1|4|8|6|.
  |7|6|2|4|9|8|1|3|5|.
  |8|4|1|6|5|3|7|2|9|.
  |1|8|4|3|6|9|5|7|2|.
  |5|7|6|1|4|2|8|9|3|.
  |9|2|3|5|8|7|6|1|4|.
\end{sudoku}
\caption{Sudoku Solution}
\end{figure}





\section{Literature}
\label{sec:introLiterature}

\cite{Berthier2007Sudoku} XXX

\section{Relationship between rows, columns, and blocks}
\label{sec:Relationship}





\bibliographystyle{plain}

\bibliography{Sudoku}

\end{document}
