% Yu-Ting Lu 24/7/2011 (Swansea)

\documentclass[11pt]{report}

\usepackage{hyperref}
\usepackage{a4}

\usepackage{sudoku}
\usepackage{amsmath}
\usepackage{colortbl}
\usepackage{multirow}
\usepackage[normalem]{ulem}

\setlength\sudokusize{8cm}
\newcommand{\cell}[9]{%
\scriptsize
\setlength{\tabcolsep}{1pt}
\renewcommand{\arraystretch}{0.5}
\hspace{-0.6em}
\begin{tabular}{ccc}
#1 & #2 & #3\\
#4 & #5 & #6\\
#7 & #8 & #9
\end{tabular}
}

\newcommand{\set}[1]{\{ #1 \}}


\begin{document}

\title{Sudoku Patterns}
\author{Yu-Ting Lu 461467@swansea.ac.uk\\
  Computer Science\\
  Swansea University\\
 }
\maketitle

\begin{abstract}
  XXX 2 paragraphs XXX
The concept of this dissertation is to understand all the Candidate Elimination Techniques for solving harder Sudoku problems and evaluate the efficiency. In the report, special techniques will be explained in detail and give an example.
Firstly, the briefly introduction of Sudoku puzzle will be given. Then the detail of special techniques for solving Sudoku will be discussed secondly with the theoretical examples.
Subsequently, the implementation in all the special techniques will be practiced in coding to evaluate the applicability.
\end{abstract}


\tableofcontents



\chapter{Introduction}
\label{cha:Introduction}


Sudoku is a puzzle which is popularised into the world for ages. And there are amount of people spend times on either hard puzzle generated, or looking for the best techniques to improve the efficiency on solving difficult puzzle.
When the puzzle difficulty becomes harder and harder, then more advanced techniques will be required to be used together or in order to cope the puzzle. Surely, as the difficulties level goes up, the cost of time raise up even thought the suitable technique is given to help. Consequently, in this report, the first chapter is going to know the background of Sudoku and what is Sudoku, and the secondly understand discussed after understanding how these special techniques applied on eliminating impossibilities in puzzle solving. Afterwards, the efficiency of the special technique for solving Sudoku will be the final topic to discuss about in the end.
In this dissertation, the goal to achieve is to experience the time of applying special techniques on solving Sudoku puzzles. In addition, to find out the most efficient way to apply the combination of more than techniques and how many puzzles could it achieve to solve.


\section{What is Sudoku?}
\label{sec:whatissudoku}

Sudoku uses an $N \times N$ matrix, where $N = n \cdot n$, and for standard Sudoku we have $n = 3$, thus $N = 9$. Each entry (``cell'') contains a number in $\{1, \dots, N\}$, or it might be empty, that is, to be filled out. A completed Sudoku puzzle must fulfil the following additional requirements:
\begin{enumerate}
\item In every row and in every column every number occurs only (exactly) once.
\item The same is true for the ``blocks'':
\begin{enumerate}
\item The whole matrix is sub-divided into $N$ blocks.
\item Each block is an $n \times n$ matrix.
\end{enumerate}
Now in each block every number has also to occur (exactly) once.
\end{enumerate}
For ``classical Sudoku'' puzzles we have the following additional rules:
\begin{enumerate}
\item A Sudoku puzzle must have at least one solution (no unsolvable puzzles are normally considered).
\item And in fact, a Sudoku puzzle must have a unique solution (no multiple solutions are normally allowed).
\end{enumerate}
An example is given with Figure \ref{sudokuEx}, while for the solution see Figure \ref{fig:solutionsudokuEx}.

\begin{figure}[htbp]
\begin{sudoku}
 |1| | |2| | | | |4|.
 | |9| |5| |3|6|7| |.
 | |4| | |7| | | | |.
 | |1| | | | | |5|9|.
 | | |2| | | |3| | |.
 |9|7| | | || |2| |.
 | | | | |1| | |6| |.
 | |8|5|3| |9| |1| |.
 |4| | | | |6| | |5|.
\end{sudoku}
\caption{An example for a Sudoku problem}
\label{sudokuEx}
\end{figure}

\begin{figure}[htbp]
\begin{sudoku}
  |1|5|7|2|6|8|9|3|4|.
  |2|9|8|5|4|3|6|7|1|.
  |3|4|6|9|7|1|5|8|2|.
  |8|1|4|6|3|2|7|5|9|.
  |5|6|2|1|9|7|3|4|8|.
  |9|7|3|8|5|4|1|2|6|.
  |7|2|9|4|1|5|8|6|3|.
  |6|8|5|3|2|9|4|1|7|.
  |4|3|1|7|8|6|2|9|5|.
\end{sudoku}
\caption{The solution for the Sudoku problem given in Figure \ref{sudokuEx}}
\label{fig:solutionsudokuEx}
\end{figure}





\section{The science of Sudoku}
\label{sec:introLiterature}
 
To study Sudoku, the basic introduction about what is Sudoku and the history of Sudoku is the first topic to understand.When we have enough background about the background, then we can start to dig in-depth about the topics: ``Mathematics of Sudoku'', ``'' and ``Solving Sudoku''.

\subsection{Mathematics of Sudoku}
\label{sec:background}
To get started on Sudoku, the origin and the fundamental mathematics of is have to be studied beforehand. Sudoku is special kind of Latin Square\cite{website:LatinSquare}. Latin square was developed by \textit{Leonhard Euler} in 18th century\footnote{Leonhard Euler, a pioneering Swiss mathematician and physicist. \cite{website:LeonhardEuler}} which is an $N \times N$ matrices which is standard on a principle: ``All the candidates from 1, \dots , 9 can exact appear once in the unit''. It was named Latin square because the Latin characters are used as symbols to fill into the cells\cite{Delahay2006Science} when it has been developed firstly. Nevertheless, integers could also be used to replace Latin characters. Later on, in 19th, there is a magazine company called ``Dell Magazines'' publish a number puzzle ``Number Place'' with a $9 \times 9$ square grid using Euler's perspective. At that time, it was so popular till few years later a Japanese puzzle company Nikoli has introduced this number puzzle as ``Suji wa dokushin ni kagiru'' (means the numbers must occur only once) which can be abbreviated to ``Sudoku'' (Su: number, doku: single)\cite{GarciaPalomino2007SolvingSudoku}. 

Common Sudoku puzzle that we see in the newspaper, magazine or the Internet are usually shaped in $9 \times 9$ cells which is same as Latin square. However, they are slightly different from each other, which Latin square is an $N \times N$ matrices filled with $N$ symbols, but Sudoku is an  $N \times N$ matrices which also tiled by $N$ blocks that each block is an $n \times n \ (n = \sqrt{N}$) sub-matrices which filled with number $1,\ \dots ,\ 9$. Nevertheless, they both have rule that the either symbol or number are all exact appeared once in the same row, same column, or the same block. A formal Sudoku puzzle is a $N \times N$ matrices with some cells have already filled in with number between $1,\ \dots ,\ 9$ and does not have any number duplicated occurs in a unit. When a Sudoku puzzle has been given, then the next step to be done is make it annotated to generate an annotated instance to begin the rules apply.

The number of valid Sudoku grids has been determined in \cite{FelgenhauerJarvis2006SudokuI} as
\begin{displaymath}
  6670903752021072936960 \approx 6.671 \times 10^{21}
\end{displaymath}
(while the number of Latin squares is $5524751496156892842531225600 \approx 5.525 \times 10^{27}$, see  \cite{FelgenhauerJarvis2006SudokuI}). And in \cite{FelgenhauerJarvis2006SudokuII} the number of \emph{essentially different} Sudoku grids (XXX to be explained!) has been determined as $5472730538$.



\subsection{How to generate a Sudoku puzzle?}
\label{sec:gensudoku}

XXX



\subsection{Solving Sudoku}
\label{sec:solvingsudoku}

To puzzle a Sudoku problem, that ``How many solutions are there for a Sudoku puzzle?'' has to be discussed firstly. Some experts, such as \emph{Tom Sheldon} says that ``A properly-formed Sudoku puzzle must have only one solution.'' in \cite{Sheldon2006Sudoku}. However, in fact, \emph{Thomas Snyder} of the USA, who won the 2007 World Sudoku Championship proved that a puzzle could have two solutions (of course that every solution). Therefore, the point of view is still not universal, thus in this report, the solution of Sudoku puzzle is assumed to be ``Unique'' .

In the book ``The Hidden Logic of Sudoku\cite{Berthier2007Sudoku}'', \emph{Denis Berthier} expounds a set of resolution of solving Sudoku puzzle that were already distributed, and the terms to utilise them. All these rules are used when the situation is matched to the prerequisite of the rule. For each cell, the rules will be run one by one until there is one can be applied on, and this procedure will keep running until a final solution has found. To list all the steps could be in listing way or in a systematic procedure,
the listing of the steps to solve the puzzle (by \emph{J. F. Crook} from \cite{Crook2009Algorithm}, Page 467 ):
\begin{enumerate}
\item find all possible candidates in the cell
\item make up an annotated instance
\item applying rules to fill in an empty cell until
\begin{enumerate}
\item a cell has been filled
\item the puzzle is finished
\end{enumerate}
\item if the puzzle has not finished yet, then go back to step 3.
\end{enumerate}
in the systematic procedure way is (from \cite{Berthier2007Sudoku}, Page 24):
\begin{quote}
\begin{tabbing}
Loop \=Until  a solution is found (or until it is proven there can be no solution)\\
\>Do \=until a rule applies  effectively \\
\>\>Take the first rule not yet tried in the list \\
\>\>Do \=until its conditions pattern effectively maps to the cell \\
\>\>\>Try all possible mappings of the conditions pattern \\
\>\>End do\\
\>End do\\
\>Apply rule on selected matching pattern\\
End loop\\
\end{tabbing}
\end{quote}
No matter in either way, a loop will keep running to find the most suitable rule which could match the mapping of the cells left, and apply the rule to either get one or more candidate asserted or eliminated till the whole Sudoku puzzle has solved or the puzzle has proven that it has no solution.


There are many rules exist to be applied flexibly to solve the Sudoku puzzle among its difficulty. To classify the rules, firstly there are divided in to two, which are ``Direct Elimination Techniques'' and ``Candidates Elimination Techniques''. Direct elimination techniques, is a method that does not have to pencil-mark all the possible candidates in the empty cells no matter the difficulty of the puzzle. All it required is a pen with logical thinking brain. It could solve most problem but might not be the most efficient way to do. In candidates elimination techniques, it provides different techniques that could be utilised in considering one candidate, a pair candidates, triplet candidates, or quad candidates differently. (\emph{Michael Mepham} has published an introductory guide teaching people how to solve a Sudoku puzzle. \cite{Mepham2005Solving})

\emph{Berthier} also issues a concept of providing information in the different representation of Sudoku puzzle. Three other perspectives, row-number, column-number, and block-number are firstly introduced in the book (\cite{Berthier2007Sudoku}, Chapter II). No matter which representation (expect block-number view is slightly different) that the Sudoku puzzle is displayed in, the information given is always same among three factors: row, column, and candidate (number). In fact, three representations which mixed and matched by the factors are compressed from six mix-match as Figure \ref{fig:mixmatch}.

\begin{figure}[htbp]
\setlength{\tabcolsep}{3pt}
\renewcommand{\arraystretch}{2}
\begin{center}
\begin{tabular}{ >{\centering\arraybackslash}m{0.6in}|| >{\centering\arraybackslash}m{1.2in}|| >{\centering\arraybackslash}m{1.2in}|| >{\centering\arraybackslash}m{1.2in}||}
\multicolumn{1}{c}{} & \multicolumn{1}{c}{row} & \multicolumn{1}{c}{column} & \multicolumn{1}{c}{number}\\ \cline{2-4}
row & X & row-column & row-number\\ \cline{2-4}
column & column-row \cellcolor[gray]{.8}& X& column- number\\ \cline{2-4}
number & number-row \cellcolor[gray]{.8}& number-column \cellcolor[gray]{.8}& X\\ \cline{2-4}
\end{tabular}
\end{center}
\caption{Mix-match among row, column, and number}
\label{fig:mixmatch}
\end{figure}


\section{Overview on the dissertation}
\label{sec:overview}


\subsection{What has been achieved}
\label{sec:whatachieved}

XXX



\subsection{Outline}
\label{sec:OUtline}

XXX




\chapter{Basic notions and notations}
\label{cha:basicnotnotat}

Before introducing the techniques for solving Sudoku, there are basic notions that have to be defined.

First we define the concrete ``problems'' we want to tackle, for example Figure \ref{sudokuEx}. Actually, we prefer not to speak of ``problem'', since in computer science this typically means a ``general problem'' (like ``how to solve Sudokus''). We also do not want to talk about ``puzzle'', since this is a rather fuzzy term. But we use the standard terminology of computer science, which speaks about ``general problems'' and their ``instances''.


\section{Sudoku instances and their solutions}
\label{sec:Sudokuinstances}

A \textit{Sudoku instance} is an $N \times N$ matrix (that is, having $N$ rows (horizontal lines) and $N$ columns (vertical lines)). The entries of such a matrix are called \textit{cells} in this context. We use indices as usual for matrices; so for example the cell $(1,4)$ (first row, fourth column) in instance Figure \ref{sudokuEx} carries number $2$.

There are two further conditions on a Sudoku instance:
\begin{enumerate}
\item The cells are either \emph{empty} or one of $1, \dots, 9$.
\item The instance has \emph{exactly one solution} (it is a ``proper'' instance).
\end{enumerate}

It remains to specify what a ``solution'' is. The solution for the instance Figure \ref{sudokuEx} is given in Figure \ref{fig:solutionsudokuEx}. We see that a \textit{solution} is the same type of matrix, with the empty cells being filled in. More precisely:
\begin{enumerate}
\item Exactly the empty cells have been filled in, with numbers in $1,\dots,9$ (nothing else has been changed).
\item In every row and in every column every number from $1,\dots,9$ occurs exactly once (that is, we have a \emph{Latin square}).
\item In every ``block'' (or ``box'') also every number from $1,\dots,9$ occurs exactly once.
\end{enumerate}

The matrix is subdivided into $N$ \textit{blocks}, as indicated in Figures \ref{sudokuEx}, \ref{fig:solutionsudokuEx}. These blocks are numbered from $1$ to $9$, from top to bottom and left to right. Each block is itself a $n \times n$ matrix containing $n \cdot n = N$ cells; see Figure \ref{fig:exampleblock}.

\begin{figure}[htbp]
\begin{sudoku}
 |{\makebox[0pt]{\hspace{-1em}\large r1\raisebox{5ex}[0ex]{\hspace{1.5em}c1}}}|{\makebox[0pt]{\raisebox{5ex}{\hspace{0em}\large c2}}}|{\makebox[0pt]{\raisebox{5ex}{\hspace{0em}\large c3}}}| {\makebox[0pt]{\raisebox{5ex}{\hspace{0em}\large c4}}}|{\makebox[0pt]{\raisebox{5ex}{\hspace{0em}\large c5}}}|{\makebox[0pt]{\raisebox{5ex}{\hspace{0em}\large c6}}}|{\makebox[0pt]{\raisebox{5ex}{\hspace{0em}\large c7}}}|{\makebox[0pt]{\raisebox{5ex}{\hspace{0em}\large c8}}}|{\makebox[0pt]{\raisebox{5ex}{\hspace{0em}\large c9}}}|.
 |{\makebox[0pt]{\hspace{-2.25em}\large r2}}|{\LARGE b1}| | |{\LARGE b2}| | |{\LARGE b3}| |.
 |{\makebox[0pt]{\hspace{-2.25em}\large r3}}| | | | | | | | |.
 |{\makebox[0pt]{\hspace{-2.25em}\large r4}}| | | | | | | | |.
 |{\makebox[0pt]{\hspace{-2.25em}\large r5}}|{\LARGE b4}| | |{\LARGE b5}| | |{\LARGE b6}| |.
 |{\makebox[0pt]{\hspace{-2.25em}\large r6}}| | | | | | | | |.
 |{\makebox[0pt]{\hspace{-2.25em}\large r7}}| | | | | | | | |.
 |{\makebox[0pt]{\hspace{-2.25em}\large r8}}|{\LARGE b7}| | |{\LARGE b8}| | |{\LARGE b9}| |.
 |{\makebox[0pt]{\hspace{-2.25em}\large r9}}| | | | | | | | |.
\end{sudoku}
\caption{The 9 blocks (``boxes'') in a Sudoku instance}
\label{fig:exampleblock}
\end{figure}



\section{Annotated instances}
\label{sec:Annotatedinstances}

We now know what the input of our intended solution procedure is, namely a Sudoku instances, and what the output shall be, namely a solution, which is defined as all empty cells filled in according to the rules. This is based on $N \times N$ matrices, where the entries are either numbers from $1,\dots,9$, or an indicator ``empty'' (which could be handled by carrying the value $0$).

We are considering in this report only techniques for arriving at a solution which do not backtrack, but which proceed in applications of ``rules'', where each such application makes clear progress towards the solution. These rules are based on the common techniques of ``pencil-marking'' the empty cells, writing down in each cell the candidates according to ``current knowledge'': more and more candidates are excluded, based on sound reasoning, until only one candidate is left, and the cell is solved. So we need to move from Sudoku instances (as defined in Section \ref{sec:Sudokuinstances}) to \textbf{annotated Sudoku instances}:
\begin{itemize}
\item Now each cell carries a partition of $\set{1,\dots,n}$ into two subsets, the \emph{possible candidates} and the \emph{excluded candidates}.
\item This is denoted by a pair $(P,E)$ with $P, E \subseteq \set{1,\dots,N}$, where $P \cap E = \emptyset$ and $P \cup E = \set{1,\dots,N}$ (``P'' for ``possible'' and ``E'' for ``excluded'').
\item For a proper annotation the (unique) solution must always be an element of the candidates (i.e., excluding candidates must always be right).
\end{itemize}

In Figure \ref{fig:rcWhole} we have annotated the example from Figure \ref{sudokuEx}. For example, the annotation for cell $(3,4)$ is $(\set{1, 6, 8, 9}, \set{2, 3, 4, 5, 7})$, which means that candidates $2,3,4,5,7$ are already excluded from possible candidates, and only $4$ candidates are left. We see that in the figures we only show the candidates. To have another example, the annotation of cell $(1,6)$ is $(\set{8}, \set{1, 2, 3, 4, 5, 6, 7, 9})$, and thus here only one candidate is left --- which must be the solution for this cell. For the purpose of visualising the solution process, once only one candidate is left, we just write out the solution as one big number for that cell; see Figure \ref{fig:rcWholeupdated} for the updated visualisation (with two cells simplified).

However for the systematic treatment we only use the annotated instances:
\begin{enumerate}
\item An instance is transformed into an annotated instance by
  \begin{itemize}
  \item translating cells with filled-in value $v$ as $(\set{v}, \set{1,\dots,N} \setminus \set{v})$,
  \item and translating empty cells as $(\set{1,\dots,N},\emptyset)$.
  \end{itemize}
\item The solution is reached once every cell is of the form $(\set{v}, \set{1,\dots,N} \setminus \set{v})$, that is, exactly one candidate is left for each cell.
\end{enumerate}

For an annotated instance $A$ we write $A_{i,j}$ for the content of cell $(i,j)$, just using matrix notation; for $A$ as in Figure \ref{fig:rcWhole} for example we have $A_{3,4} = (\set{1, 6, 8, 9}, \set{2, 3, 4, 5, 7})$.


\begin{figure}[htbp]
\begin{sudoku}
|1|{\cell {}{}3{}56{}{}{}}|{\cell {}{}3{}{}678{}}|2|{\cell {}{}{}{}{}6{}89}|{\cell {}{}{}{}{}{}{}8{}}|{\cell {}{}{}{}5{}{}89}|{\cell {}{}3{}{}{}{}89}|4|.
|{\cell {}2{}{}{}{}{}8{}}|9|{\cell {}{}{}{}{}{}{}8{}}|5|{\cell {}{}{}4{}{}{}8{}}|3|6| 7|{\cell 12{}{}{}{}{}8{}}|.
|{\cell {}23{}56{}8{}}|4|{\cell {}{}3{}{}6{}8{}}|{\cell 1{}{}{}{}6{}89}|7|{\cell 1{}{}{}{}{}{}8{}}|{\cell 12{}{}5{}{}89}|{\cell {}{}3{}{}{}{}89}|{\cell 123{}{}{}{}8{}}|.
|{\cell {}{}3{}{}6{}8{}}|1|{\cell {}{}34{}6{}8{}}|{\cell {}{}{}4{}678{}}|{\cell {}234{}6{}8{}}|{\cell {}2{}4{}{}78{}}|{\cell {}{}{}4{}{}78{}}|5|9|.
|{\cell {}{}{}{}56{}8{}}|{\cell {}{}{}{}56{}{}{}}|2|{\cell 1{}{}4{}6789}|{\cell {}{}{}456{}89}|{\cell 1{}{}45{}78{}}|3|{\cell {}{}{}4{}{}{}8{}}|{\cell 1{}{}{}{}678{}}|.
|9|7|{\cell {}{}34{}6{}8{}}|{\cell 1{}{}4{}6{}8{}}|{\cell {}{}3456{}8{}}|{\cell 1{}{}45{}{}8{}}|{\cell 1{}{}4{}{}{}8{}}|2|{\cell 1{}{}{}{}6{}8{}}|.
|{\cell {}23{}{}{}7{}{}}|{\cell {}23{}{}{}{}{}{}}|{\cell {}{}3{}{}{}7{}9}|{\cell {}{}{}4{}{}78{}}|1|{\cell {}2{}45{}78{}}|{\cell {}2{}4{}{}789}|6|{\cell {}23{}{}{}78{}}|.
|{\cell {}2{}{}{}67{}{}}|8|5|3| {\cell {}2{}4{}{}{}{}{}}|9| {\cell {}2{}4{}{}7{}{}}|1| {\cell {}2{}{}{}{}7{}{}}|.
|4| {\cell {}23{}{}{}{}{}{}}|{\cell 1{}3{}{}{}7{}9}|{\cell {}{}{}{}{}{}78{}}|{\cell {}2{}{}{}{}{}8{}}|6|{\cell {}2{}{}{}{}789}|{\cell {}{}3{}{}{}{}89}| 5|.
\end{sudoku}
\caption{Annotated Sudoku instance}
\label{fig:rcWhole}
\end{figure}

\begin{figure}[htbp]
\begin{sudoku}
|1|{\cell {}{}3{}56{}{}{}}|{\cell {}{}3{}{}67{\sout 8}{}}|2|{\cell {}{}{}{}{}6{}{\sout 8}9}|{\cell {}{}{}{}{}{}{}{\underline 8}{}}|{\cell {}{}{}{}5{}{}{\sout 8}9}|{\cell {}{}3{}{}{}{}{\sout 8}9}|4|.
|{\cell {}2{}{}{}{}{}8{}}|9|{\cell {}{}{}{}{}{}{}{\underline 8}{}}|5|{\cell {}{}{}4{}{}{}{\sout 8}{}}|3|6| 7|{\cell 12{}{}{}{}{}8{}}|.
|{\cell {}23{}56{}8{}}|4|{\cell {}{}3{}{}6{}8{}}|{\cell 1{}{}{}{}6{}{\sout 8}9}|7|{\cell 1{}{}{}{}{}{}{\sout 8}{}}|{\cell 12{}{}5{}{}89}|{\cell {}{}3{}{}{}{}89}|{\cell 123{}{}{}{}8{}}|.
|{\cell {}{}3{}{}6{}8{}}|1|{\cell {}{}34{}6{}8{}}|{\cell {}{}{}4{}678{}}|{\cell {}234{}6{}8{}}|{\cell {}2{}4{}{}7{\sout 8}{}}|{\cell {}{}{}4{}{}78{}}|5|9|.
|{\cell {}{}{}{}56{}8{}}|{\cell {}{}{}{}56{}{}{}}|2|{\cell 1{}{}4{}6789}|{\cell {}{}{}456{}89}|{\cell 1{}{}45{}7{\sout 8}{}}|3|{\cell {}{}{}4{}{}{}8{}}|{\cell 1{}{}{}{}678{}}|.
|9|7|{\cell {}{}34{}6{}8{}}|{\cell 1{}{}4{}6{}8{}}|{\cell {}{}3456{}8{}}|{\cell 1{}{}45{}{}{\sout 8}{}}|{\cell 1{}{}4{}{}{}8{}}|2|{\cell 1{}{}{}{}6{}8{}}|.
|{\cell {}23{}{}{}7{}{}}|{\cell {}23{}{}{}{}{}{}}|{\cell {}{}3{}{}{}7{}9}|{\cell {}{}{}4{}{}78{}}|1|{\cell {}2{}45{}7{\sout 8}{}}|{\cell {}2{}4{}{}789}|6|{\cell {}23{}{}{}78{}}|.
|{\cell {}2{}{}{}67{}{}}|8|5|3| {\cell {}2{}4{}{}{}{}{}}|9| {\cell {}2{}4{}{}7{}{}}|1| {\cell {}2{}{}{}{}7{}{}}|.
|4| {\cell {}23{}{}{}{}{}{}}|{\cell 1{}3{}{}{}7{}9}|{\cell {}{}{}{}{}{}78{}}|{\cell {}2{}{}{}{}{}8{}}|6|{\cell {}2{}{}{}{}789}|{\cell {}{}3{}{}{}{}89}| 5|.
\end{sudoku}
\caption{The annotated instance of Figure \ref{fig:rcWhole}, with completed visualisation}
\label{fig:rcWholeupdated}
\end{figure}



\section{On the nature of ``Sudoku rules''}
\label{sec:naturesudokurules}

Having the notion of annotated instances at hand, we can now say what a ``rule'' is:
\begin{itemize}
\item A  is a partial map, mapping certain annotated instances $A$ to annotated instances $A'$ for those $A$ in the \emph{domain} of the rule, such that:
  \begin{itemize}
  \item For some cells certain candidates have been excluded.
  \item No other type of change takes places (that is, we only exclude candidates, while excluded candidates never get included again).
  \item These changes are corrected, that is, the (unique) solution is still possible.
  \item At least one cell has been changed (something happened).
  \end{itemize}
\end{itemize}


\subsection{Updating the annotated instances I}
\label{sec:UpdatingtheinstancesI}

When a proper \textbf{Sudoku rule} (which said in section \ref{sec:naturesudokurules}) has been successful applied, certain candidates must have excluded from some cells. If any of the cells in a mapped annotated instance $A'$ has an annotation like $A'\ (i,j) =\ (\set{v}, \set{1,\dots,N} \setminus \set{v})$, then all other cells shared a row, a column, or a block with the cell $A'\ (i,j)$ will have candidate $v$, excluded and the annotation updated. The annotation for the cells will be updated as followings:
\begin{itemize}
\item all the cells shared a row with $A'\ (i,j)$ have candidate $v$ excluded.
\item all the cells shared a column with $A'\ (i,j)$ have candidate $v$ excluded.
\item all the cells shared a block with $A'\ (i,j)$ have candidate $v$ excluded.
\item highlight the candidate $v$ in cell $A'\ (i,j)$.
\end{itemize} 

From the Figure \ref{fig:rcWholeupdated}, we found the annotation for $A'\ (1,6)$ is $A'\ (1,6) =\ (\set{8}, \set{1, 2, 3, 4, 5, 6, 7, 9} \setminus \set{8})$, which is matched the format of an annotation for a cell has one candidate left $A'\ (i,j) =\ (\set{v}, \set{1,\dots,N} \setminus \set{v})$. Thus the update has to be made for all other cells shared a row, a column, or a block with $A'\ (1,6)$, that is candidate $v$ has to be excluded from the row, column, and block. The examples are given for each situation:
\begin{itemize}
\item $A'\ (1,6) =\ (\set{8}, \set{1, 2, 3, 4, 5, 6, 7, 9} \setminus \set{8}) \Rightarrow v = 8$
\item Shared a ``row'': $A'\ (1,3)$, $A'\ (1,5)$, $A'\ (1,7)$ and $A'\ (1,8)$ all have candidate $v$ and shared a row as $A'\ (1,6)$
\begin{eqnarray*}
A'\ (1,3) =\ (\set{3, 6, 7, 8}, \set{1, 2, 4, 5, 9})\ \Rightarrow A'\ (1,3) =\ (\set{3, 6, 7, 8}\setminus \set{8}, \set{1, 2, 4, 5, 8, 9}).\\
A'\ (1,5) =\ (\set{6, 8, 9}, \set{1, 2, 3, 4, 5, 7})\ \Rightarrow A'\ (1,5) =\ (\set{6, 8, 9}\setminus \set{8}, \set{1, 2, 3, 4, 5, 7, 8}).\\
A'\ (1,7) =\ (\set{5, 8, 9}, \set{1, 2, 3, 4, 6, 7})\ \Rightarrow A'\ (1,7) =\ (\set{5, 8, 9}\setminus \set{8}, \set{1, 2, 3, 4, 6, 7, 8}).\\
A'\ (1,8) =\ (\set{3, 8, 9}, \set{1, 2, 4, 5, 6, 7})\ \Rightarrow A'\ (1,8) =\ (\set{3, 8, 9}\setminus \set{8}, \set{1, 2, 4, 5, 6, 7, 8}).
\end{eqnarray*}
\item Shared a ``column'': $A'\ (3,6)$, $A'\ (4,6), $A'\ (5,6), $A'\ (6,6) and $A'\ (7,6) all have candidate $v$ and shared a row as $A'\ (1,6)$
\begin{eqnarray*}
A'\ (3,6) =\ (\set{1, 8}, \set{2, 3, 4, 5, 6, 7, 9})\ \Rightarrow A'\ (3,6) =\ (\set{1, 8}\setminus \set{8}, \set{2, 3, 4, 5, 6, 7, 8, 9}).\\
A'\ (4,6) =\ (\set{2, 4, 7, 8}, \set{1, 3, 5, 6, 9})\ \Rightarrow A'\ (4,6) =\ (\set{2, 4, 7, 8}\setminus \set{8}, \set{1, 3, 5, 6, 8, 9}).\\
A'\ (5,6) =\ (\set{1, 4, 5, 7, 8}, \set{2, 3, 6, 9})\ \Rightarrow A'\ (5,6) =\ (\set{1, 4, 5, 7, 8}\setminus \set{8}, \set{2, 3, 6, 8, 9}).\\
A'\ (6,6) =\ (\set{1, 4, 5, 8}, \set{2, 3, 6, 7, 9})\ \Rightarrow A'\ (6,6) =\ (\set{1, 4, 5, 8}\setminus \set{8}, \set{2, 3, 6, 7, 8, 9}).\\
A'\ (7,6) =\ (\set{2, 4, 5, 7, 8}, \set{1, 3, 6, 9})\ \Rightarrow A'\ (7,6) =\ (\set{2, 4, 5, 7, 8}\setminus \set{8}, \set{1, 3, 6, 8, 9}).\\
\end{eqnarray*}
\item Shared a ``block'': $A'\ (1,5)$ $A'\ (2,5)$ $A'\ (3,4)$ and $A'\ (3,6)$ all have candidate $v$ and shared a block as $A'\ (1,6)$
\begin{eqnarray*}
A'\ (1,5) =\ (\set{6 ,8, 9}, \set{1, 2, 3, 4, 5, 7})\ \Rightarrow A'\ (1,5) =\ (\set{6 ,8, 9}\setminus \set{8}, \set{1, 2, 3, 4, 5, 7, 8}).\\
A'\ (2,5) =\ (\set{4, 8}, \set{1, 2, 3, 5, 6, 7, 9})\ \Rightarrow A'\ (2,5) =\ (\set{4, 8}\setminus \set{8}, \set{1, 2, 3, 5, 6, 7, 8, 9}).\\
A'\ (3,4) =\ (\set{1, 6, 8, 9}, \set{2, 3, 4, 5, 7})\ \Rightarrow A'\ (3,4) =\ (\set{1, 6, 8, 9}\setminus \set{8}, \set{2, 3, 4, 5, 7, 8}).\\
A'\ (3,6) =\ (\set{1, 8}, \set{2, 3, 4, 5, 6, 7, 9})\ \Rightarrow A'\ (3,6) =\ (\set{1, 8}\setminus \set{8}, \set{2, 3, 4, 5, 6, 7, 8, 9}).
\end{eqnarray*}
\end{itemize}


\chapter{Exploiting symmetries by using different views}
\label{cha:exploitingsymm}


\section{The general idea}
\label{sec:diffviewsidea}

In an annotated instance we can find $4$ ``dimensions'':
\begin{enumerate}
\item rows (``r'')
\item columns (``c'')
\item numbers (the content of the cells; ``n'')
\item blocks (``b'').
\begin{itemize}
\item squares(``s'').
\end{itemize}
\end{enumerate}
We have introduced ``blocks'' in section \ref{sec:Sudokuinstances} which is the matrix subdivided in to $N$ blocks and these blocks are numbered from 1 to 9, from top to bottom and left to right. Here comes a new element ``square'', which is, a block subdivided into $N$ squares, and theses squares are numbered from 1to 9, from top to bottom and left to right (see Figure \ref{fig:squres}). 

\begin{figure}[htbp]
\begin{sudoku}
 |{\makebox[0pt]{\hspace{-1em}\large b1\raisebox{5ex}[0ex]{\hspace{1.5em}n1}}}|{\makebox[0pt]{\raisebox{5ex}{\hspace{0em}\large n2}}}|{\makebox[0pt]{\raisebox{5ex}{\hspace{0em}\large n3}}}| {\makebox[0pt]{\raisebox{5ex}{\hspace{0em}\large n4}}}|{\makebox[0pt]{\raisebox{5ex}{\hspace{0em}\large n5}}}|{\makebox[0pt]{\raisebox{5ex}{\hspace{0em}\large n6}}}|{\makebox[0pt]{\raisebox{5ex}{\hspace{0em}\large n7}}}|{\makebox[0pt]{\raisebox{5ex}{\hspace{0em}\large n8}}}|{\makebox[0pt]{\raisebox{5ex}{\hspace{0em}\large n9}}}|.
 |{\makebox[0pt]{\hspace{-2.25em}\large b2}}| | | | | | | | |.
 |{\makebox[0pt]{\hspace{-2.25em}\large b3}}| | | | | | | | |.
 |{\makebox[0pt]{\hspace{-2.25em}\large b4}}| | |{\LARGE s1}|{\LARGE s2}|{\LARGE s3}| | | |.
 |{\makebox[0pt]{\hspace{-2.25em}\large b5}}| | |{\LARGE s4}|{\LARGE s5}|{\LARGE s6}| | | |.
 |{\makebox[0pt]{\hspace{-2.25em}\large b6}}| | |{\LARGE s7}|{\LARGE s8}|{\LARGE s9}| | | |.
 |{\makebox[0pt]{\hspace{-2.25em}\large b7}}| | | | | | | | |.
 |{\makebox[0pt]{\hspace{-2.25em}\large b8}}| | | | | | | | |.
 |{\makebox[0pt]{\hspace{-2.25em}\large b9}}| | | | | | | | |.
\end{sudoku}
\caption{The 9 squares in a block}
\label{fig:squres}
\end{figure}

We are now introducing four ``views'' on an annotated instance:
\begin{enumerate}
\item ``rc-view''
\item ``rn-view''
\item ``cn-view''
\item ``bn-view''.
\end{enumerate}
The rc-view (``row-column view'') is our standard point of view: the rows and columns of the matrix are the rows and the columns of the (annotated) instance, and the entries refer to the possible (or impossible) numbers for that cell. The other three views allow us to derive new rules from old rules, by just changing the perspective:
\begin{center}
The starting observation is that rows, columns, numbers and blocks are all numbers $1,\dots,N$ --- so perhaps they can change role?!
\end{center}




\section{Different  perspectives}
\label{sec:Differentperspectives}

An annotated Sudoku instance (see \ref{sec:Annotatedinstances}) was defined in a standard row-column view that the rows and columns of the matrix are the rows and the columns of the annotated instance. Therefore, here, a new concept of three other views (which has been introduced in the beginning of this chapter) for an annotated instance will be introduced.

Four different views:
\begin{enumerate}
\item ``rc-view'' (see Figure \ref{fig:rcView}) - the rows and columns of the matrix are the rows and the columns of the annotated instance.
\item ``rn-view'' (see Figure \ref{fig:rnView}) - the rows and numbers of the matrix are the rows and the columns of the annotated instance
\item ``cn-view'' (see Figure \ref{fig:cnView}) - the columns and numbers of the matrix are the rows and the columns of the annotated instance
\item ``bn-view'' (see Figure \ref{fig:bnView}) - the blocks and numbers of the matrix are the rows and the columns of the annotated instance
\end{enumerate}

\begin{figure}[htbp]
\begin{sudoku}
 |{\makebox[0pt]{\hspace{-1em}\large r1\raisebox{5ex}[0ex]{\hspace{2em}\large c1}}1}|{\makebox[0pt]{\raisebox{2ex}[1ex]{\hspace{1em}\large c2}}{\cell {}{}3{}56{}{}{}}}|{\makebox[0pt]{\raisebox{2ex}[1ex]{\hspace{1em}\large c3}}{\cell {}{}3{}{}678{}}}|{\makebox[0pt]{\raisebox{2.5ex}[1ex]{\hspace{0.5em}\large c4}}2}|{\makebox[0pt]{\raisebox{2ex}[1ex]{\hspace{1em}\large c5}}{\cell {}{}{}{}{}6{}89}}|{\makebox[0pt]{\raisebox{2.5ex}[1ex]{\hspace{0.5em}\large c6}}8}|{\makebox[0pt]{\raisebox{2ex}[1ex]{\hspace{1em}\large c7}}{\cell {}{}{}{}5{}{}89}}|{\makebox[0pt]{\raisebox{2ex}[1ex]{\hspace{1em}\large c8}}{\cell {}{}3{}{}{}{}89}}|{\makebox[0pt]{\raisebox{2.5ex}[1ex]{\hspace{0.5em}\large c9}}4}|.
 |{\makebox[0pt]{\hspace{-2.25em}\large r2}{\cell {}2{}{}{}{}{}8{}}}|9|8|5|{\cell {}{}{}4{}{}{}{\sout 8}{}}|3|6| 7|{\cell 12{}{}{}{}{}8{}}|.
 |{\makebox[0pt]{\hspace{-2.25em}\large r3}{\cell {}23{}56{}8{}}}|4|{\cell {}{}3{}{}6{}8{}}|{\cell 1{}{}{}{}6{}{\sout 8}9}|7|{\cell 1{}{}{}{}{}{}{\sout 8}{}}|{\cell 12{}{}5{}{}89}|{\cell {}{}3{}{}{}{}89}|{\cell 123{}{}{}{}8{}}|.
 |{\makebox[0pt]{\hspace{-2.25em}\large r4}{\cell {}{}3{}{}6{}8{}}}|1|{\cell {}{}34{}6{}8{}}|{\cell {}{}{}4{}678{}}|{\cell {}234{}6{}8{}}|{\cell {}2{}4{}{}7{\sout 8}{}}|{\cell {}{}{}4{}{}78{}}|5|9|.
 |{\makebox[0pt]{\hspace{-2.25em}\large r5}{\cell {}{}{}{}56{}8{}}}|{\cell {}{}{}{}56{}{}{}}|2|{\cell 1{}{}4{}6789}|{\cell {}{}{}456{}89}|{\cell 1{}{}45{}7{\sout 8}{}}|3|{\cell {}{}{}4{}{}{}8{}}|{\cell 1{}{}{}{}678{}}|.
 |{\makebox[0pt]{\hspace{-2.5em}\large r6}9}|7|{\cell {}{}34{}6{}8{}}|{\cell 1{}{}4{}6{}8{}}|{\cell {}{}3456{}8{}}|{\cell 1{}{}45{}{}{\sout 8}{}}|{\cell 1{}{}4{}{}{}8{}}|2|{\cell 1{}{}{}{}6{}8{}}|.
 |{\makebox[0pt]{\hspace{-2em}\large r7}{\cell {}23{}{}{}7{}{}}}|{\cell {}23{}{}{}{}{}{}}|{\cell {}{}3{}{}{}7{}9}|{\cell {}{}{}4{}{}78{}}|1|{\cell {}2{}45{}7{\sout 8}{}}|{\cell {}2{}4{}{}789}|6|{\cell {}23{}{}{}78{}}|.
 |{\makebox[0pt]{\hspace{-2em}\large r8}{\cell {}2{}{}{}67{}{}}}|8|5|3| {\cell {}2{}4{}{}{}{}{}}|9| {\cell {}2{}4{}{}7{}{}}|1| {\cell {}2{}{}{}{}7{}{}}|.
 |{\makebox[0pt]{\hspace{-2.5em}\large r9}4}| {\cell {}23{}{}{}{}{}{}}|{\cell 1{}3{}{}{}7{}9}|{\cell {}{}{}{}{}{}78{}}|{\cell {}2{}{}{}{}{}8{}}|6|{\cell {}2{}{}{}{}789}|{\cell {}{}3{}{}{}{}89}| 5|.
\end{sudoku}
\caption{An annotated instance in rc-view.}
\label{fig:rcView}
\end{figure}


\begin{figure}[htbp]
\begin{sudoku}
|{\makebox[0pt]{\hspace{-1em}\large r1\raisebox{5ex}[0ex]{\hspace{2em}\large n1}}1}|{\makebox[0pt]{\raisebox{2.5ex}[1ex]{\hspace{0.5em}\large n2}}4}|{\makebox[0pt]{\raisebox{2ex}[1ex]{\hspace{1em}\large n3}}{\cell {}23{}{}{}{}8{}}}|{\makebox[0pt]{\raisebox{2.5ex}[1ex]{\hspace{0.5em}\large n4}}9}|{\makebox[0pt]{\raisebox{2ex}[1ex]{\hspace{1em}\large n5}}{\cell {}2{}{}{}{}7{}{}}}|{\makebox[0pt]{\raisebox{2ex}[1ex]{\hspace{1em}\large n6}}{\cell {}23{}5{}{}{}{}}}|{\makebox[0pt]{\raisebox{2ex}[1ex]{\hspace{1em}\large n7}}{\cell {}{}3{}{}{}{}{}{}}}|{\makebox[0pt]{\raisebox{2ex}[1ex]{\hspace{1em}\large n8}}{\cell {}{}3{}5678{}}}|{\makebox[0pt]{\raisebox{2ex}[1ex]{\hspace{1em}\large n9}}{\cell {}{}{}{}5{}78{}}}|.
|{\makebox[0pt]{\hspace{-2em}\large r2}{\cell {}{}{}{}{}{}{}{}9}}|{\cell 1{}{}{}{}{}{}{}9}|6|{\cell {}{}{}{}5{}{}{}{}}|4|7| 8|{\cell 1{}3{}5{}{}{}9}|2|.
|{\makebox[0pt]{\hspace{-2em}\large r3}{\cell {}{}{}4{}67{}9}}|{\cell 1{}{}{}{}{}7{}9}|{\cell 1{}3{}{}{}{}89}|2|{\cell 1{}{}{}{}{}7{}{}}|{\cell 1{}34{}{}{}{}{}}|5|{\cell 1{}34{}6789}|{\cell {}{}{}4{}{}78{}}|.
|{\makebox[0pt]{\hspace{-2.5em}\large r4}2}|{\cell {}{}{}{}56{}{}{}}|{\cell 1{}3{}5{}{}{}{}}|{\cell {}{}34567{}{}}|8|{\cell 1{}345{}{}{}{}}|{\cell {}{}{}4{}67{}{}}|{\cell 1{}34567{}{}}|9|.
|{\makebox[0pt]{\hspace{-2em}\large r5}{\cell {}{}{}4{}6{}{}9}}|3|7|{\cell {}{}{}456{}8{}}|{\cell 12{}{}56{}{}{}}|{\cell 12{}45{}{}{}9}|{\cell {}{}{}4{}6{}{}9}|{\cell 1{}{}456{}89}|{\cell {}{}{}45{}{}{}{}}|.
|{\makebox[0pt]{\hspace{-2em}\large r6}{\cell {}{}{}4{}67{}9}}|8|{\cell {}{}3{}5{}{}{}{}}|{\cell {}{}34567{}{}}|{\cell {}{}{}{}56{}{}{}}|{\cell {}{}345{}{}{}9}|2|{\cell {}{}34567{}9}|1|.
|{\makebox[0pt]{\hspace{-2.5em}\large r7}5}|{\cell 12{}{}{}67{}9}|{\cell 123{}{}{}{}{}9}|{\cell {}{}{}4{}67{}{}}|{\cell {}{}{}{}{}6{}{}{}}|8|{\cell 1{}34{}67{}9}|{\cell {}{}{}4{}67{}9}|{\cell {}{}3{}{}{}7{}{}}|.
|{\makebox[0pt]{\hspace{-2.5em}\large r8}8}|{\cell 1{}{}{}5{}7{}9}|4| {\cell {}{}{}{}5{}7{}{}}|3| {\cell 1{}{}{}{}{}{}{}{}}|{\cell 1{}{}{}{}{}7{}9}|2|6|.
|{\makebox[0pt]{\hspace{-2em}\large r9}{\cell {}{}3{}{}{}{}{}{}}}| {\cell {}2{}{}5{}7{}{}}|{\cell {}23{}{}{}{}8{}}|1|9|6|{\cell {}{}34{}{}7{}{}}| {\cell {}{}{}45{}78{}}|{\cell {}{}3{}{}{}78{}}|.
\end{sudoku}
\caption{An annotated instance in rn-view.}
\label{fig:rnView}
\end{figure}



\begin{figure}[htbp]
\begin{sudoku}
|{\makebox[0pt]{\hspace{-1em}\large c1\raisebox{5ex}[0ex]{\hspace{2em}\large n1}}1}|{\makebox[0pt]{\raisebox{2ex}[1ex]{\hspace{1em}\large n2}}{\cell {}23{}{}{}78{}}}|{\makebox[0pt]{\raisebox{2ex}[1ex]{\hspace{1em}\large n3}}{\cell {}{}34{}{}7{}{}}}|{\makebox[0pt]{\raisebox{2.6ex}[1ex]{\hspace{0.5em}\large n4}}9}|{\makebox[0pt]{\raisebox{2ex}[1ex]{\hspace{1em}\large n5}}{\cell {}{}3{}5{}{}{}{}}}|{\makebox[0pt]{\raisebox{2ex}[1ex]{\hspace{1em}\large n6}}{\cell {}{}345{}{}8{}}}|{\makebox[0pt]{\raisebox{2ex}[1ex]{\hspace{1em}\large n7}}{\cell {}{}{}{}{}{}78{}}}|{\makebox[0pt]{\raisebox{2ex}[1ex]{\hspace{1em}\large n8}}{\cell {}{}345{}{}{}{}}}|{\makebox[0pt]{\raisebox{2.5ex}[1ex]{\hspace{0.5em}\large n9}}6}|.
|{\makebox[0pt]{\hspace{-2em}\large c2}4}|{\cell {}{}{}{}{}{}7{}9}|{\cell 1{}{}{}{}{}7{}9}|3|{\cell 1{}{}{}5{}{}{}{}}|{\cell 1{}{}{}5{}{}{}{}}|6|8|2|.
|{\makebox[0pt]{\hspace{-2em}\large c3}{\cell {}{}{}{}{}{}{}{}9}}|5|{\cell 1{}34{}67{}9}|{\cell {}{}{}4{}6{}{}{}}|8|{\cell 1{}34{}6{}{}{}}|{\cell 1{}{}{}{}{}7{}9}|{\cell 1234{}6{}{}{}}|{\cell {}{}{}{}{}{}7{}9}|.
|{\makebox[0pt]{\hspace{-2.5em}\large c4}{\cell {}{}3{}{}6{}{}{}}}|1|8|{\cell {}{}{}4567{}{}}|2|{\cell {}{}3456{}{}{}}|{\cell {}{}{}45{}7{}9}|{\cell {}{}34567{}9}|{\cell {}{}3{}5{}{}{}{}}|.
|{\makebox[0pt]{\hspace{-2em}\large c5}7}|{\cell {}{}{}4{}{}{}89}|{\cell {}{}3{}5{}{}{}{}}|{\cell {}2{}456{}8{}}|{\cell {}{}{}{}56{}{}{}}|{\cell 1{}{}456{}{}{}}|3|{\cell 12{}456{}{}9}|{\cell 1{}{}{}5{}{}{}{}}|.
|{\makebox[0pt]{\hspace{-2em}\large c6}{\cell {}{}3{}56{}{}{}}}|{\cell {}{}{}4{}{}7{}{}}|2|{\cell {}{}{}4567{}{}}|{\cell {}{}{}{}567{}{}}|9|{\cell {}{}{}45{}7{}{}}|{\cell 1{}34567{}{}}|8|.
|{\makebox[0pt]{\hspace{-2.5em}\large c7}{\cell {}{}3{}{}6{}{}{}}}|{\cell {}{}3{}{}{}789}|5|{\cell {}{}{}4{}678{}}|{\cell 1{}3{}{}{}{}{}{}}|2|{\cell {}{}{}4{}{}789}|{\cell 1{}34{}67{}9}|{\cell 1{}3{}{}{}7{}9}|.
|{\makebox[0pt]{\hspace{-2.5em}\large c8}8}|6|{\cell 1{}3{}{}{}{}{}9}|{\cell {}{}{}{}5{}{}{}{}}|4|7|2|{\cell 1{}3{}5{}{}{}9}|{\cell 1{}3{}{}{}{}{}9}|.
|{\makebox[0pt]{\hspace{-2em}\large c9}{\cell {}23{}56{}{}{}}}|{\cell {}23{}{}{}78{}}|{\cell {}{}3{}{}{}7{}{}}|1|9|{\cell {}{}{}{}56{}{}{}}|{\cell {}{}{}{}5{}{}89}|{\cell {}23{}567{}{}}|4|.
\end{sudoku}
\caption{An annotated instance in cn-view.}
\label{fig:cnView}
\end{figure}



\begin{figure}[htbp]
\begin{sudoku}
|{\makebox[0pt]{\hspace{-1em}\large b1\raisebox{5ex}[0ex]{\hspace{2em}\large n1}}1}|{\makebox[0pt]{\raisebox{2ex}[1ex]{\hspace{1em}\large n2}}{\cell {}{}{}4{}{}7{}{}}}|{\makebox[0pt]{\raisebox{2ex}[1ex]{\hspace{1em}\large n3}}{\cell {}23{}{}{}7{}9}}|{\makebox[0pt]{\raisebox{2.6ex}[1ex]{\hspace{0.5em}\large n4}}8}|{\makebox[0pt]{\raisebox{2ex}[1ex]{\hspace{1em}\large n5}}{\cell {}2{}{}{}{}7{}{}}}|{\makebox[0pt]{\raisebox{2ex}[1ex]{\hspace{1em}\large n6}}{\cell {}23{}{}{}7{}9}}|{\makebox[0pt]{\raisebox{2ex}[1ex]{\hspace{1em}\large n7}}{\cell {}{}3{}{}{}{}{}{}}}|{\makebox[0pt]{\raisebox{2ex}[1ex]{\hspace{1em}\large n8}}{\cell {}{}34{}67{}9}}|{\makebox[0pt]{\raisebox{2.5ex}[1ex]{\hspace{0.5em}\large n9}}5}|.
|{\makebox[0pt]{\hspace{-2.1em}\large b2}{\cell {}{}{}{}{}{}7{}9}}|1|6|{\cell {}{}{}{}5{}{}{}{}}|4|{\cell {}2{}{}{}{}7{}{}}|8|{\cell {}23{}5{}{}89}|{\cell {}2{}{}{}{}7{}{}}|.
|{\makebox[0pt]{\hspace{-2em}\large b3}{\cell {}{}{}{}{}67{}9}}|{\cell {}{}{}{}{}67{}9}|{\cell {}2{}{}{}{}{}89}|3|{\cell 1{}{}{}{}{}7{}{}}|4|5|{\cell 12{}{}{}6789}|{\cell 12{}{}{}{}78{}}|.
|{\makebox[0pt]{\hspace{-2.5em}\large b4}2}|6|{\cell 1{}3{}{}{}{}{}9}|{\cell {}{}3{}{}{}{}{}9}|{\cell {}{}{}45{}{}{}{}}|{\cell 1{}345{}{}{}9}|8|{\cell 1{}34{}{}{}{}9}|7|.
|{\makebox[0pt]{\hspace{-2em}\large b5}{\cell {}{}{}4{}67{}9}}|{\cell {}23{}{}{}{}{}{}}|{\cell {}2{}{}{}{}{}8{}}|{\cell 123456789}|{\cell {}{}{}{}56{}89}|{\cell 12{}45{}78{}}|{\cell 1{}34{}6{}{}{}}|{\cell 123456789}|{\cell {}{}{}45{}{}{}{}}|.
|{\makebox[0pt]{\hspace{-2em}\large b6}{\cell {}{}{}{}{}67{}9}}|8|4|{\cell 1{}{}{}5{}7{}{}}|2|{\cell {}{}{}{}{}6{}{}9}|{\cell 1{}{}{}{}6{}{}{}}|{\cell 1{}{}{}567{}9}|3|.
|{\makebox[0pt]{\hspace{-2.5em}\large b7}{\cell {}{}{}{}{}{}{}{}9}}|{\cell 12{}4{}{}{}8{}}|{\cell 123{}{}{}{}89}|7|6|{\cell {}{}{}4{}{}{}{}{}}|{\cell 1{}34{}{}{}{}9}|5|{\cell {}{}3{}{}{}{}{}9}|.
|{\makebox[0pt]{\hspace{-2.5em}\large b8}2}|{\cell {}{}3{}5{}{}8{}}|4|{\cell 1{}3{}5{}{}{}{}}|{\cell {}{}3{}{}{}{}{}{}}|9|{\cell 1{}3{}{}{}7{}{}}|{\cell 1{}3{}{}{}78{}}|6|.
|{\makebox[0pt]{\hspace{-2.5em}\large b9}5}|{\cell 1{}34{}67{}{}}|{\cell {}{}3{}{}{}{}8{}}|{\cell 1{}{}4{}{}{}{}{}}|9|2|{\cell 1{}34{}67{}{}}|{\cell 1{}3{}{}{}78{}}|{\cell 1{}{}{}{}{}78{}}.
\end{sudoku}
\caption{An annotated instance in bn-view.}
\label{fig:bnView}
\end{figure}

Each view gives different information for the view to provide various point of views. The information given in an rc-view (standard) annotated instance is, possible and excluded candidates for each cell (row, column) which is written as:
\begin{displaymath}
A'\ (row,\ column) =\ (\set{{P}ossible\ candidates}, \set{{E}xcluded\ candidates})
\end{displaymath}


In an annotated instance in rn-view (see Figure \ref{fig:rnView}), the information contained in each cell $A'\ (row,\ number)$ is, the cell has a number of one row could possibly appeared in such ``possible columns'' and not in the ``excluded columns''. The equation for the cell (row, number) is written as:
\begin{displaymath}
A'\ (row,\ number) =\ (\set{{P}ossible\ columns}, \set{{E}xcluded\ columns})
\end{displaymath}
Suppose we take a cell $A'\ (7,\ 4)$ in Figure \ref{fig:rnView}, the equation for cell $A'\ (7,\ 4)$ is written as $A'\ (7,\ 4) =\ (\set{4, 6, 7}, \set{1, 2, 3, 5, 8, 9} )$, number $4$ for row $7$ could only occur in possible columns 4, 6, and 7, the rest of columns are excluded from the possibilities.


In an annotated instance in cn-view (see Figure \ref{fig:cnView}), the information contained in each cell $A'\ (column,\ number)$ is, the cell has a number of one column could possibly appeared in such ``possible rows'' and not in the ``excluded rows''. The equation for the cell (column, number) is written as:
\begin{displaymath}
A'\ (column,\ number) =\ (\set{{P}ossible\ rows}, \set{{E}xcluded\ rows})
\end{displaymath}
Suppose we take a cell $A'\ (5,\ 6)$ in Figure \ref{fig:cnView}, the equation for cell $A'\ (5,\ 6)$ is written as $A'\ (5,\ 6) =\ (\set{1, 4, 5, 6}, \set{2, 3, 7, 8, 9} )$, number $6$ for row $5$ could only occur in possible rows 1, 4, 5, and 6, the rest of rows are excluded from the possibilities.

And in an annotated instance in bn-view (see Figure \ref{fig:bnView}), the information contained in each cell $A'\ (block,\ number)$ is, the cell has a number of one block could possibly appeared in such ``possible squares of the block'' in an standard instance in rc-view. The equation for the cell (block, number) is written as:
\begin{displaymath}
A'\ (block,\ number) =\ (\set{{P}ossible\ squares\ of\ a\ block}, \set{{E}xcluded\ squares\ of\ a\ block}
\end{displaymath}
Suppose we take a cell $A'\ (6,\ 6)$ in Figure \ref{fig:bnView}, the equation for cell $A'\ (6,\ 6)$ is written as $A'\ (6,\ 6) =\ (\set{6, 9}, \set{1, 2, 3, 4, 5, 7, 8} )$, number $9$ for block $6$ could only occur in possible squares 6 and 9 of block $6$ in the standard rc-view instance (see Figure \ref{fig:rcView}), the rest of squares are excluded from the possibilities for block $6$.


To have an annotated instance built in other views based on an standard annotated instance in rc-view, here are the examples for the process. Figure \ref{fig:rccolumn} is a row taken from the figure \ref{fig:rcView}. In the rc-view (see Figure \ref{fig:rccolumn}), the number in each cel simply represent the known value $v$ or the candidates for the cell. However, in the rn-view (see Figure \ref{fig:rncolumn}), the numbers in each cell will represent the known column or the candidate columns for the cell.

\begin{figure}[htbp]
\setlength{\tabcolsep}{3pt}
\renewcommand{\arraystretch}{2}
\begin{center}
\begin{tabular}{ >{\centering\arraybackslash}m{0.2in}| >{\centering\arraybackslash}m{0.3in}| >{\centering\arraybackslash}m{0.3in}| >{\centering\arraybackslash}m{0.3in}| >{\centering\arraybackslash}m{0.3in}| >{\centering\arraybackslash}m{0.3in}| >{\centering\arraybackslash}m{0.3in}| >{\centering\arraybackslash}m{0.3in}| >{\centering\arraybackslash}m{0.3in}| >{\centering\arraybackslash}m{0.3in}| }
\multicolumn{1}{c}{} & \multicolumn{1}{c}{$c_{1}$} & \multicolumn{1}{c}{$c_{2}$} & \multicolumn{1}{c}{$c_{3}$} &  \multicolumn{1}{c}{$c_{4}$} & \multicolumn{1}{c}{$c_{5}$} & \multicolumn{1}{c}{$c_{6}$} & \multicolumn{1}{c}{$c_{7}$} &  \multicolumn{1}{c}{$c_{8}$} & \multicolumn{1}{c}{$c_{9}$}\\ \cline{2-10}
$r_{1}$ &\LARGE 1 &{\cell {}{}3{}56{}{}{}} & {\cell {}{}3{}{}678{}} & \LARGE 2 &{\cell {}{}{}{}{}6{}89}& {\cell {}{}{}{}{}{}{}8{}} &{\cell {}{}{}{}5{}{}89} & {\cell {}{}3{}{}{}{}89} & \LARGE 4\\ \cline{2-10}
\end{tabular}
\end{center}
\caption{A row from Figure \ref{fig:rcWhole} in rc-view.}
\label{fig:rccolumn}
\end{figure}

\begin{table}
\begin{center}
  \begin{tabular}{|p{7cm}|p{7cm}|}
    \hline
    \textbf{r1 in the rc-view} & \textbf{r1 in the rn-view} \\ \hline
    filled-in value 1 only occur in $(r1,\ c1)$ & $(r1,\ n1)$ contains only c1\\ \hline
    filled-in value 2 only occur in $(r1,\ c4)$ & $(r1,\ n2)$ contains only c4 \\ \hline
    candidate 3 occur in $(r1,\ c2)$,  $(r1,\ c3)$, and $(r1,\ c8)$ & $(r1,\ n3)$ contains c2, c3, and c8 \\ \hline
    filled-in value 4 only occur in $(r1,\ c9)$ & $(r1,\ n4)$ contains only c9 \\ \hline
    candidate 5 occur in $(r1,\ c2)$ and $(r1,\ c7)$ & $(r1,\ n5)$ contains c2, and c7 \\ \hline
    candidate 6 occur in $(r1,\ c2)$,  $(r1,\ c3)$, and $(r1,\ c5)$ & $(r1,\ n6)$ contains c2, c3, and c5 \\ \hline
    candidate 7 occur in $(r1,\ c3)$ & $(r1,\ n7)$ contains only c3 \\ \hline
    candidate 8 occur in $(r1,\ c3)$,  $(r1,\ c5)$,  $(r1,\ c6)$,  $(r1,\ c7)$, and $(r1,\ c8)$ & $(r1,\ n8)$ contains only c3, c5, c6, c7 and c8 \\ \hline
    candidate 9 occur in $(r1,\ c5)$, $(r1,\ c7)$, and  $(r1,\ c8)$ & $(r1,\ n9)$ contains only c5, c7, and c8 \\ \hline
  \end{tabular}
\end{center}
\caption{The information of r1 in both rc-view and rn-view}
\label{tab:rcandrn}
\end{table}

The Table \ref{tab:rcandrn} is the information analysis from number 1 to 9 of r1 in both rc-view and rn-view. As the analysis from the table, we gain the equations for the cells below to get a row built in rn-view (see Figure \ref{fig:rncolumn}):
\begin{itemize}
\item $(r1,\ n1)\ =\ (\set{1}, \set{2, 3, 4, 5, 6, 7, 8, 9})$
\item $(r1,\ n2)\ =\ (\set{4}, \set{1, 2, 3, 5, 6, 7, 8, 9})$
\item $(r1,\ n3)\ =\ (\set{2, 3, 8}, \set{1, 4, 5, 6, 7, 9})$
\item $(r1,\ n4)\ =\ (\set{9}, \set{1, 2, 3, 4, 5, 6, 7, 8})$
\item $(r1,\ n5)\ =\ (\set{2, 7}, \set{1, 3, 4, 5, 6, 8, 9})$
\item $(r1,\ n6)\ =\ (\set{2, 3, 5}, \set{1, 4, 6, 7, 8, 9})$
\item $(r1,\ n7)\ =\ (\set{3}, \set{1, 2, 4, 5, 6, 7, 8, 9})$
\item $(r1,\ n8)\ =\ (\set{3, 5, 6, 7, 8}, \set{1, 2, 4, 9})$
\item $(r1,\ n9)\ =\ (\set{5, 7, 8}, \set{1, 2, 3, 4, 6, 9})$
\end{itemize}


\begin{figure}[htbp]
\setlength{\tabcolsep}{3pt}
\renewcommand{\arraystretch}{2} 
\begin{center}
\begin{tabular}{ >{\centering\arraybackslash}m{0.2in}| >{\centering\arraybackslash}m{0.3in}| >{\centering\arraybackslash}m{0.3in}| >{\centering\arraybackslash}m{0.3in}| >{\centering\arraybackslash}m{0.3in}| >{\centering\arraybackslash}m{0.3in}| >{\centering\arraybackslash}m{0.3in}| >{\centering\arraybackslash}m{0.3in}| >{\centering\arraybackslash}m{0.3in}| >{\centering\arraybackslash}m{0.3in}| }
\multicolumn{1}{c}{} &  \multicolumn{1}{c}{$n_{1}$}  &  \multicolumn{1}{c}{ $n_{2}$} &  \multicolumn{1}{c}{ $n_{3}$} &  \multicolumn{1}{c}{$n_{4}$}  &  \multicolumn{1}{c}{$n_{5}$}  & \multicolumn{1}{c}{$n_{6}$}  &  \multicolumn{1}{c}{ $n_{7}$} &  \multicolumn{1}{c}{$n_{8}$}  &  \multicolumn{1}{c}{$n_{9}$}  \\ \cline{2-10}
$r_{1}$ &\LARGE 1 &\LARGE 4 & {\cell {}23{}{}{}{}8{}} & \LARGE 9  & {\cell {}2{}{}{}{}7{}{}} & {\cell {}23{}5{}{}{}{}} & {\cell {}{}3{}{}{}{}{}{}} & {\cell {}{}3{}5678{}} & {\cell {}{}{}{}5{}78{}} \\ \cline{2-10}
\end{tabular}
\end{center}
\caption{An row rebuilt in rn-view from Figure \ref{fig:rccolumn}.}
\label{fig:rncolumn}
\end{figure}


To re-build a column of an annotated instance in cn-view from rc-view, we have the analysis in Table \ref{tab:rcandcn}.
\begin{table}
\begin{center}
  \begin{tabular}{|p{7cm}|p{7cm}|}
    \hline
    \textbf{c1 in the rc-view} & \textbf{c1 in the cn-view} \\ \hline
    filled-in value 1 only occur in $(r1,\ c1)$ & $(c1,\ n1)$ contains only r1\\ \hline
    candidate 2 occur in $(r2,\ c1)$, $(r3,\ c1)$, $(r7,\ c1)$, $(r8,\ c1)$ & $(c1,\ n2)$ contains r2, r3, r7, r8 \\ \hline
    candidate 3 occur in $(r3,\ c1)$,  $(r4,\ c1)$, and $(r7,\ c1)$ & $(c1,\ n3)$ contains r3, r4, and r7 \\ \hline
    filled-in value 4 only occur in $(r9,\ c1)$ & $(c1,\ n4)$ contains only r9 \\ \hline
    candidate 5 occur in $(r3,\ c1)$ and $(r5,\ c1)$ & $(c1,\ n5)$ contains r3, and r5 \\ \hline
    candidate 6 occur in $(r3,\ c1)$,  $(r4,\ c1)$, $(r5,\ c1)$ and $(r8,\ c1)$ & $(c1,\ n6)$ contains r3, r4 ,r5 and r8 \\ \hline
    candidate 7 occur in $(r7,\ c1)$ and $(r8,\ c1)$ & $(c1,\ n7)$ contains r7 and r8\\ \hline
    candidate 8 occur in $(r3,\ c1)$,  $(r4,\ c1)$ and $(r5,\ c1)$ & $(c1,\ n8)$ contains only r3, r4 and r5 \\ \hline
    filled-in value 9 occur in $(r6,\ c1)$ & $(c1,\ n9)$ contains only r6 \\ \hline
  \end{tabular}
\caption{The information of r1 in both rc-view and cn-view}
\label{tab:rcandcn}
\end{center}
\end{table}

After the analysis from the Table \ref{tab:rcandcn}, we gain the equations for the cells in the c1 in cn-view (see Figure \ref{fig:cncolumn}):
\begin{itemize}
\item $(c1,\ n1)\ =\ (\set{1}, \set{2, 3, 4, 5, 6, 7, 8, 9})$
\item $(c1,\ n2)\ =\ (\set{2, 3, 7, 8}, \set{1, 4, 5, 6, 9})$
\item $(c1,\ n3)\ =\ (\set{3, 4, 7}, \set{1, 2, 5, 6, 8, 9})$
\item $(c1,\ n4)\ =\ (\set{9}, \set{1, 2, 3, 4, 5, 6, 7, 8})$
\item $(c1,\ n5)\ =\ (\set{3, 5}, \set{1, 2, 4, 6, 7, 8, 9})$
\item $(c1,\ n6)\ =\ (\set{3, 4, 5, 8}, \set{1, 2, 6, 7, 9})$
\item $(c1,\ n7)\ =\ (\set{7, 8}, \set{1, 2, 3, 4, 5, 6, 9})$
\item $(c1,\ n8)\ =\ (\set{3, 4, 5}, \set{1, 2, 6, 7, 8, 9})$
\item $(c1,\ n9)\ =\ (\set{6}, \set{1, 2, 3, 4, 5, 7, 8, 9})$
\end{itemize}

\begin{figure}[htbp]
\setlength{\tabcolsep}{3pt}
\renewcommand{\arraystretch}{2} 
\begin{center}
\begin{tabular}{ >{\centering\arraybackslash}m{0.2in}| >{\centering\arraybackslash}m{0.3in}| >{\centering\arraybackslash}m{0.3in}| >{\centering\arraybackslash}m{0.3in}| >{\centering\arraybackslash}m{0.3in}| >{\centering\arraybackslash}m{0.3in}| >{\centering\arraybackslash}m{0.3in}| >{\centering\arraybackslash}m{0.3in}| >{\centering\arraybackslash}m{0.3in}| >{\centering\arraybackslash}m{0.3in}| }
\multicolumn{1}{c}{} &  \multicolumn{1}{c}{$n_{1}$}  &  \multicolumn{1}{c}{ $n_{2}$} &  \multicolumn{1}{c}{ $n_{3}$} &  \multicolumn{1}{c}{$n_{4}$}  &  \multicolumn{1}{c}{$n_{5}$}  & \multicolumn{1}{c}{$n_{6}$}  &  \multicolumn{1}{c}{ $n_{7}$} &  \multicolumn{1}{c}{$n_{8}$}  &  \multicolumn{1}{c}{$n_{9}$}  \\ \cline{2-10}
$rc_{1}$ &\LARGE 1 & {\cell {}23{}{}{}78{}} & {\cell {}{}34{}{}7{}{}} & \LARGE 9  & {\cell {}{}3{}5{}{}{}{}} & {\cell {}{}345{}{}8{}} & {\cell {}{}{}{}{}{}78{}} & {\cell {}{}345{}{}{}{}} & \LARGE 4 \\ \cline{2-10}
\end{tabular}
\end{center}
\caption{An row rebuilt in cn-view from Figure \ref{fig:rccolumn}.}
\label{fig:cncolumn}
\end{figure}

To re-build a block of an annotated instance in bn-view from the block 1 of Figure \ref{fig:rcView} in rc-view, we have the analysis in Table \ref{tab:rcandbn}.

\begin{table}
\begin{center}
  \begin{tabular}{|p{7cm}|p{7cm}|}
    \hline
    \textbf{b1 in the rc-view} & \textbf{b1 in the bn-view} \\ \hline
    filled-in value 1 only occur in $(r1,\ c1)$ & $(b1,\ n1)$ contains only s1\\ \hline
    candidate 2 occur in $(r2,\ c1)$, $(r3,\ c1)$ & $(b1,\ n2)$ contains s4, and s7 \\ \hline
    candidate 3 occur in $(r1,\ c2)$,  $(r1,\ c3)$, $(r3,\ c1)$ and $(r3,\ c3)$ & $(b1,\ n3)$ contains s2, s3, s7 and s9 \\ \hline
    filled-in value 4 only occur in $(r3,\ c2)$ & $(b1,\ n4)$ contains only s8 \\ \hline
    candidate 5 occur in $(r1,\ c2)$ and $(r3,\ c1)$ & $(b1,\ n5)$ contains s2, and s7 \\ \hline
    candidate 6 occur in $(r1,\ c2)$,  $(r1,\ c3)$, $(r3,\ c1)$ and $(r3,\ c3)$ & $(b1,\ n6)$ contains s2, s3, s7 and s9 \\ \hline
    candidate 7 occur in $(r1,\ c3)$& $(b1,\ n7)$ contains s3\\ \hline
    filled-in value 8 only occur in $(r2,\ c3)$ & $(b1,\ n8)$ contains only s6 \\ \hline
    filled-in value 9 occur in $(r2,\ c2)$ & $(b1,\ n9)$ contains only s5 \\ \hline
  \end{tabular}
\caption{The information of b1 in both rc-view and bn-view}
\label{tab:rcandbn}
\end{center}
\end{table}

The equations below are gained from the analysis in the Table \ref{tab:rcandbn}. 
\begin{itemize}
\item $(b1,\ n1)\ =\ (\set{1}, \set{2, 3, 4, 5, 6, 7, 8, 9})$
\item $(b1,\ n2)\ =\ (\set{4, 7}, \set{1, 2, 3, 5, 6, 8, 9})$
\item $(b1,\ n3)\ =\ (\set{2, 3, 7, 9}, \set{1, 4, 5, 6, 8})$
\item $(b1,\ n4)\ =\ (\set{8}, \set{1, 2, 3, 4, 5, 6, 7, 9})$
\item $(b1,\ n5)\ =\ (\set{2, 7}, \set{1, 3, 4, 5, 6, 8, 9})$
\item $(b1,\ n6)\ =\ (\set{2, 3, 7, 9}, \set{1, 4, 5, 6, 8})$
\item $(b1,\ n7)\ =\ (\set{3}, \set{1, 2, 4, 5, 6, 7, 8, 9})$
\item $(b1,\ n8)\ =\ (\set{6}, \set{1, 2, 3, 4, 5, 7, 8, 9})$
\item $(b1,\ n9)\ =\ (\set{5}, \set{1, 2, 3, 4, 6, 7, 8, 9})$
\end{itemize}
And these equation could now have a row generated like Figure \ref{fig:bncolumn}.

\begin{figure}[htbp]
\setlength{\tabcolsep}{3pt}
\renewcommand{\arraystretch}{2}
\begin{center}
\begin{tabular}{ >{\centering\arraybackslash}m{0.2in}| >{\centering\arraybackslash}m{0.3in}| >{\centering\arraybackslash}m{0.3in}| >{\centering\arraybackslash}m{0.3in}| >{\centering\arraybackslash}m{0.3in}| >{\centering\arraybackslash}m{0.3in}| >{\centering\arraybackslash}m{0.3in}| >{\centering\arraybackslash}m{0.3in}| >{\centering\arraybackslash}m{0.3in}| >{\centering\arraybackslash}m{0.3in}| }
\multicolumn{1}{c}{} & \multicolumn{1}{c}{$n_{1}$} & \multicolumn{1}{c}{ $n_{2}$} & \multicolumn{1}{c}{ $n_{3}$} & \multicolumn{1}{c}{$n_{4}$} & \multicolumn{1}{c}{$n_{5}$} & \multicolumn{1}{c}{$n_{6}$} & \multicolumn{1}{c}{ $n_{7}$} & \multicolumn{1}{c}{$n_{8}$} & \multicolumn{1}{c}{$n_{9}$}\\ \cline{2-10}
$b_{1}$ &\LARGE 1 & {\cell {}{}{}4{}{}7{}{}} & {\cell {}23{}{}{}7{}9} & \LARGE 8 & {\cell {}2{}{}{}{}7{}{}} & {\cell {}23{}{}{}7{}9} &\LARGE 3 &\LARGE 6 & \LARGE 5\\ \cline{2-10}
\end{tabular}
\end{center}
\caption{An row rebuilt in bn-view from block 1 of Figure \ref{fig:rcView}}
\label{fig:bncolumn}
\end{figure}







\section{Updating the annotated instances II}
\label{sec:UpdatingtheinstancesII}

In the subsection \ref{sec:UpdatingtheinstancesI}, we have discussed the update made when a proper Sudoku rule has been applied in an annotated instance. Therefore, that is only about the update in the annotated instance in rc-view. Since we have more different perspectives introduced, we''ll introduce more update effect in the three other views in this section.


\subsection{Rn-View}
\label{sec:basicrulern}

When a Sudoku rule is successful applied in the rn-view, certain columns must have excluded from some cells. If any of the cells in a mapped rn-vew annotated instance $A'$ has an annotation like $A'\ (row, number) =\ (\set{v}, \set{1,\dots,N} \setminus \set{v})$, then all other cells shared a row, a number with the cell $A'\ (row, number)$ will have candidate $v$, excluded and the annotation updated. The annotation for the cells will be updated as followings:
\begin{itemize}
\item all the cells shared a row with $A'\ (row, number)$ have column $v$ excluded.
\item all the cells shared a number with $A'\ (row, number)$ have column $v$ excluded.
\end{itemize} 

From the Figure \ref{fig:rnView}, we found the annotation for $A'\ (1,6)$ is $A'\ (1,6) =\ (\set{8}, \set{1, \dots, 9} \setminus \set{8})$, which is matched the format of an annotation for a cell has one candidate left $A'\ (i,j) =\ (\set{v}, \set{1,\dots,N} \setminus \set{v})$. Thus the update has to be made for all other cells shared a row, or a number with $A'\ (1,7)$, that is candidate $v$ has to be excluded from the row, and the number. The examples are given for each situation, and an updated figure for this example is Figure \ref{fig:rnUpdate}:
\begin{itemize}
\item $A'\ (1,7) =\ (\set{3}, \set{1, \dots, 9} \setminus \set{3}) \Rightarrow v = 3$
\item Shared a ``row'': $A'\ (1,3)$, $A'\ (1,6)$ and $A'\ (1,8)$ all have column $v$ in the cell and shared a row with cell $A'\ (1,7)$
\begin{eqnarray*}
A'\ (1,3) =\ (\set{2, 3, 8}, \set{1, 4, 5, 6, 7, 9})\ \Rightarrow A'\ (1,3) =\ (\set{2, 3, 8}\setminus \set{3}, \set{1, 3, 4, 5, 6, 7, 9}).\\
A'\ (1,6) =\ (\set{2, 3, 5}, \set{1, 4, 6, 7, 8, 9})\ \Rightarrow A'\ (1,3) =\ (\set{2, 3, 5}\setminus \set{3}, \set{1, 3, 4, 6, 7, 8, 9}).\\
A'\ (1,8) =\ (\set{3, 5, 6, 7, 8}, \set{1, 2, 4, 9})\ \Rightarrow A'\ (1,3) =\ (\set{3, 5, 6, 7, 8}\setminus \set{3}, \set{1, 2, 3, 4, 9}).
\end{eqnarray*}
\item Shared a ``number'': $A'\ (7,7)$ and $A'\ (9,7)$ has column $v$ in the cell and shared a number with cell $A'\ (1,7)$
\begin{eqnarray*}
A'\ (7,7) =\ (\set{1, 3, 4, 6, 7, 9}, \set{2, 5, 8})\ \Rightarrow A'\ (7,7) =\ (\set{1, 3, 4, 6, 7, 9}\setminus \set{3}, \set{2, 3, 5, 8}).\\
A'\ (9,7) =\ (\set{3, 4, 7}, \set{1, 2, 5, 6, 8, 9})\ \Rightarrow A'\ (9,7) =\ (\set{3, 4, 7}\setminus \set{3}, \set{1, 2, 3, 5, 6, 8, 9}).
\end{eqnarray*}
\end{itemize}


\begin{figure}[htbp]
\begin{sudoku}
|{\makebox[0pt]{\hspace{-1em}\large r1\raisebox{5ex}[0ex]{\hspace{2em}\large n1}}1}|{\makebox[0pt]{\raisebox{2.5ex}[1ex]{\hspace{0.5em}\large n2}}4}|{\makebox[0pt]{\raisebox{2ex}[1ex]{\hspace{1em}\large n3}}{\cell {}2{\sout 3}{}{}{}{}8{}}}|{\makebox[0pt]{\raisebox{2.5ex}[1ex]{\hspace{0.5em}\large n4}}9}|{\makebox[0pt]{\raisebox{2ex}[1ex]{\hspace{1em}\large n5}}{\cell {}2{}{}{}{}7{}{}}}|{\makebox[0pt]{\raisebox{2ex}[1ex]{\hspace{1em}\large n6}}{\cell {}2{\sout 3}{}5{}{}{}{}}}|{\makebox[0pt]{\raisebox{2ex}[1ex]{\hspace{1em}\large n7}}{\cell {}{}3{}{}{}{}{}{}}}|{\makebox[0pt]{\raisebox{2ex}[1ex]{\hspace{1em}\large n8}}{\cell {}{}{\sout 3}{}5678{}}}|{\makebox[0pt]{\raisebox{2ex}[1ex]{\hspace{1em}\large n9}}{\cell {}{}{}{}5{}78{}}}|.
|{\makebox[0pt]{\hspace{-2em}\large r2}{\cell {}{}{}{}{}{}{}{}9}}|{\cell 1{}{}{}{}{}{}{}9}|6|{\cell {}{}{}{}5{}{}{}{}}|4|7| 8|{\cell 1{}3{}5{}{}{}9}|2|.
|{\makebox[0pt]{\hspace{-2em}\large r3}{\cell {}{}{}4{}67{}9}}|{\cell 1{}{}{}{}{}7{}9}|{\cell 1{}3{}{}{}{}89}|2|{\cell 1{}{}{}{}{}7{}{}}|{\cell 1{}34{}{}{}{}{}}|5|{\cell 1{}34{}6789}|{\cell {}{}{}4{}{}78{}}|.
|{\makebox[0pt]{\hspace{-2.5em}\large r4}2}|{\cell {}{}{}{}56{}{}{}}|{\cell 1{}3{}5{}{}{}{}}|{\cell {}{}34567{}{}}|8|{\cell 1{}345{}{}{}{}}|{\cell {}{}{}4{}67{}{}}|{\cell 1{}34567{}{}}|9|.
|{\makebox[0pt]{\hspace{-2em}\large r5}{\cell {}{}{}4{}6{}{}9}}|3|7|{\cell {}{}{}456{}8{}}|{\cell 12{}{}56{}{}{}}|{\cell 12{}45{}{}{}9}|{\cell {}{}{}4{}6{}{}9}|{\cell 1{}{}456{}89}|{\cell {}{}{}45{}{}{}{}}|.
|{\makebox[0pt]{\hspace{-2em}\large r6}{\cell {}{}{}4{}67{}9}}|8|{\cell {}{}3{}5{}{}{}{}}|{\cell {}{}34567{}{}}|{\cell {}{}{}{}56{}{}{}}|{\cell {}{}345{}{}{}9}|2|{\cell {}{}34567{}9}|1|.
|{\makebox[0pt]{\hspace{-2.5em}\large r7}5}|{\cell 12{}{}{}67{}9}|{\cell 123{}{}{}{}{}9}|{\cell {}{}{}4{}67{}{}}|{\cell {}{}{}{}{}6{}{}{}}|8|{\cell 1{}{\sout 3}4{}67{}9}|{\cell {}{}{}4{}67{}9}|{\cell {}{}3{}{}{}7{}{}}|.
|{\makebox[0pt]{\hspace{-2.5em}\large r8}8}|{\cell 1{}{}{}5{}7{}9}|4| {\cell {}{}{}{}5{}7{}{}}|3| {\cell 1{}{}{}{}{}{}{}{}}|{\cell 1{}{}{}{}{}7{}9}|2|6|.
|{\makebox[0pt]{\hspace{-2em}\large r9}{\cell {}{}3{}{}{}{}{}{}}}| {\cell {}2{}{}5{}7{}{}}|{\cell {}23{}{}{}{}8{}}|1|9|6|{\cell {}{}{\sout 3}4{}{}7{}{}}| {\cell {}{}{}45{}78{}}|{\cell {}{}3{}{}{}78{}}|.
\end{sudoku}
\caption{An updated annotated instance of Figure \ref{fig:rnView} in rn-view.}
\label{fig:rnUpdate}
\end{figure}




\subsection{Cn-View}
\label{sec:basicrulecn}

When a Sudoku rule is successful applied in the cn-view, certain columns must have excluded from some cells. If any of the cells in a mapped cn-vew annotated instance $A'$ has an annotation like $A'\ (column, number) =\ (\set{v}, \set{1,\dots,N} \setminus \set{v})$, then all other cells shared a column, a number with the cell $A'\ (column, number)$ will have candidate $v$, excluded and the annotation updated. The annotation for the cells will be updated as followings:
\begin{itemize}
\item all the cells shared a column with $A'\ (column, number)$ have row $v$ excluded.
\item all the cells shared a number with $A'\ (column, number)$ have row $v$ excluded.
\end{itemize} 

From the Figure \ref{fig:cnView}, we found the annotation for $A'\ (8,4)$ is $A'\ (8, 4) =\ (\set{5}, \set{1, \dots, 9} \setminus \set{5})$, which is matched the format of an annotation for a cell has one candidate left $A'\ (i,j) =\ (\set{v}, \set{1,\dots,N} \setminus \set{v})$. Thus the update has to be made for all other cells shared a column, or a number with $A'\ (8,4)$, that is candidate $v$ has to be excluded from the column, and the number. The examples are given for each situation, and an updated figure for this example is Figure \ref{fig:cnUpdate}:
\begin{itemize}
\item $A'\ (8,4) =\ (\set{5}, \set{1, \dots, 9} \setminus \set{5}) \Rightarrow v = 5$
\item Shared a ``column'': $A'\ (8,8)$ has row $v$ in the cell and shared a column with cell $A'\ (8,4)$
\begin{eqnarray*}
A'\ (8,8) =\ (\set{1, 3, 5, 9}, \set{2, 4, 6, 7, 8, 9})\ \Rightarrow A'\ (8,8) =\ (\set{1, 3, 5, 9}\setminus \set{5}, \set{2, 4, 6, 7, 8, 9}).
\end{eqnarray*}
\item Shared a ``number'': $A'\ (4,4)$, $A'\ (5,4)$ and $A'\ (6,4)$ has row $v$ in the cell and shared a number with cell $A'\ (8,4)$
\begin{eqnarray*}
A'\ (4,4) =\ (\set{4, 5, 6, 7}, \set{1, 2, 3, 8, 9})\ \Rightarrow A'\ (4,4) =\ (\set{4, 5, 6, 7}\setminus \set{5}, \set{1, 2, 3, 5, 8, 9}).\\
A'\ (5,4) =\ (\set{2, 4, 5, 6, 8}, \set{1, 3, 7, 9})\ \Rightarrow A'\ (5,4) =\ (\set{2, 4, 5, 6, 8}\setminus \set{5}, \set{1, 3, 5, 7, 9}).\\
A'\ (4,4) =\ (\set{4, 5, 6, 7}, \set{1, 2, 3, 8, 9})\ \Rightarrow A'\ (4,4) =\ (\set{4, 5, 6, 7}\setminus \set{5}, \set{1, 2, 3, 5, 8, 9}).
\end{eqnarray*}
\end{itemize}


\begin{figure}[htbp]
\begin{sudoku}
|{\makebox[0pt]{\hspace{-1em}\large c1\raisebox{5ex}[0ex]{\hspace{2em}\large n1}}1}|{\makebox[0pt]{\raisebox{2ex}[1ex]{\hspace{1em}\large n2}}{\cell {}23{}{}{}78{}}}|{\makebox[0pt]{\raisebox{2ex}[1ex]{\hspace{1em}\large n3}}{\cell {}{}34{}{}7{}{}}}|{\makebox[0pt]{\raisebox{2.6ex}[1ex]{\hspace{0.5em}\large n4}}9}|{\makebox[0pt]{\raisebox{2ex}[1ex]{\hspace{1em}\large n5}}{\cell {}{}3{}5{}{}{}{}}}|{\makebox[0pt]{\raisebox{2ex}[1ex]{\hspace{1em}\large n6}}{\cell {}{}345{}{}8{}}}|{\makebox[0pt]{\raisebox{2ex}[1ex]{\hspace{1em}\large n7}}{\cell {}{}{}{}{}{}78{}}}|{\makebox[0pt]{\raisebox{2ex}[1ex]{\hspace{1em}\large n8}}{\cell {}{}345{}{}{}{}}}|{\makebox[0pt]{\raisebox{2.5ex}[1ex]{\hspace{0.5em}\large n9}}6}|.
|{\makebox[0pt]{\hspace{-2em}\large c2}4}|{\cell {}{}{}{}{}{}7{}9}|{\cell 1{}{}{}{}{}7{}9}|3|{\cell 1{}{}{}5{}{}{}{}}|{\cell 1{}{}{}5{}{}{}{}}|6|8|2|.
|{\makebox[0pt]{\hspace{-2em}\large c3}{\cell {}{}{}{}{}{}{}{}9}}|5|{\cell 1{}34{}67{}9}|{\cell {}{}{}4{}6{}{}{}}|8|{\cell 1{}34{}6{}{}{}}|{\cell 1{}{}{}{}{}7{}9}|{\cell 1234{}6{}{}{}}|{\cell {}{}{}{}{}{}7{}9}|.
|{\makebox[0pt]{\hspace{-2.5em}\large c4}{\cell {}{}3{}{}6{}{}{}}}|1|8|{\cell {}{}{}4{\sout 5}67{}{}}|2|{\cell {}{}3456{}{}{}}|{\cell {}{}{}45{}7{}9}|{\cell {}{}34567{}9}|{\cell {}{}3{}5{}{}{}{}}|.
|{\makebox[0pt]{\hspace{-2em}\large c5}7}|{\cell {}{}{}4{}{}{}89}|{\cell {}{}3{}5{}{}{}{}}|{\cell {}2{}4{\sout 5}6{}8{}}|{\cell {}{}{}{}56{}{}{}}|{\cell 1{}{}456{}{}{}}|3|{\cell 12{}456{}{}9}|{\cell 1{}{}{}5{}{}{}{}}|.
|{\makebox[0pt]{\hspace{-2em}\large c6}{\cell {}{}3{}56{}{}{}}}|{\cell {}{}{}4{}{}7{}{}}|2|{\cell {}{}{}4{\sout 5}67{}{}}|{\cell {}{}{}{}567{}{}}|9|{\cell {}{}{}45{}7{}{}}|{\cell 1{}34567{}{}}|8|.
|{\makebox[0pt]{\hspace{-2.5em}\large c7}{\cell {}{}3{}{}6{}{}{}}}|{\cell {}{}3{}{}{}789}|5|{\cell {}{}{}4{}678{}}|{\cell 1{}3{}{}{}{}{}{}}|2|{\cell {}{}{}4{}{}789}|{\cell 1{}34{}67{}9}|{\cell 1{}3{}{}{}7{}9}|.
|{\makebox[0pt]{\hspace{-2.5em}\large c8}8}|6|{\cell 1{}3{}{}{}{}{}9}|{\cell {}{}{}{}5{}{}{}{}}|4|7|2|{\cell 1{}3{}{\sout 5}{}{}{}9}|{\cell 1{}3{}{}{}{}{}9}|.
|{\makebox[0pt]{\hspace{-2em}\large c9}{\cell {}23{}56{}{}{}}}|{\cell {}23{}{}{}78{}}|{\cell {}{}3{}{}{}7{}{}}|1|9|{\cell {}{}{}{}56{}{}{}}|{\cell {}{}{}{}5{}{}89}|{\cell {}23{}567{}{}}|4|.
\end{sudoku}
\caption{An updated annotated instance of Figure \ref{fig:cnView} in cn-view.}
\label{fig:cnUpdate}
\end{figure}



\subsection{Bn-View}
\label{sec:basicruleBn}

When a Sudoku rule is successful applied in the bn-view, certain squares must have excluded from some cells. If any of the cells in a mapped bn-vew annotated instance $A'$ has an annotation like $A'\ (block, number) =\ (\set{v}, \set{1,\dots,N} \setminus \set{v})$, then all other cells shared a block with the cell $A'\ (block, number)$ will have square $v$, excluded and the annotation updated. The annotation for the cells will be updated as followings:
\begin{itemize}
\item all the cells shared a block with $A'\ (block, number)$ have square $v$ excluded.
\end{itemize} 

From the Figure \ref{fig:bnView}, we found the annotation for $A'\ (2,4)$ is $A'\ (2, 4) =\ (\set{5}, \set{1, \dots, 9} \setminus \set{5})$, which is matched the format of an annotation for a cell has one candidate left $A'\ (i,j) =\ (\set{v}, \set{1,\dots,N} \setminus \set{v})$. Thus the update has to be made for all other cells shared a column, or a number with $A'\ (2,4)$, that is candidate $v$ has to be excluded from the column, and the number. The examples are given for each situation, and an updated figure for this example is Figure \ref{fig:bnUpdate}:
\begin{itemize}
\item $A'\ (2,4) =\ (\set{5}, \set{1, \dots, 9} \setminus \set{5}) \Rightarrow v = 5$
\item Shared a ``block'': $A'\ (2,8)$ has row $v$ in the cell and shared a block with cell $A'\ (2,4)$
\begin{eqnarray*}
A'\ (2,8) =\ (\set{2, 3, 5, 8, 9}, \set{1, 4, 6, 7})\ \Rightarrow A'\ (1,8) =\ (\set{2, 3, 5, 8, 9}\setminus \set{5}, \set{1, 4, 6, 7}).
\end{eqnarray*}
\end{itemize}


\begin{figure}[htbp]
\begin{sudoku}
|{\makebox[0pt]{\hspace{-1em}\large b1\raisebox{5ex}[0ex]{\hspace{2em}\large n1}}1}|{\makebox[0pt]{\raisebox{2ex}[1ex]{\hspace{1em}\large n2}}{\cell {}{}{}4{}{}7{}{}}}|{\makebox[0pt]{\raisebox{2ex}[1ex]{\hspace{1em}\large n3}}{\cell {}23{}{}{}7{}9}}|{\makebox[0pt]{\raisebox{2.6ex}[1ex]{\hspace{0.5em}\large n4}}8}|{\makebox[0pt]{\raisebox{2ex}[1ex]{\hspace{1em}\large n5}}{\cell {}2{}{}{}{}7{}{}}}|{\makebox[0pt]{\raisebox{2ex}[1ex]{\hspace{1em}\large n6}}{\cell {}23{}{}{}7{}9}}|{\makebox[0pt]{\raisebox{2ex}[1ex]{\hspace{1em}\large n7}}{\cell {}{}3{}{}{}{}{}{}}}|{\makebox[0pt]{\raisebox{2ex}[1ex]{\hspace{1em}\large n8}}{\cell {}{}34{}67{}9}}|{\makebox[0pt]{\raisebox{2.5ex}[1ex]{\hspace{0.5em}\large n9}}5}|.
|{\makebox[0pt]{\hspace{-2.1em}\large b2}{\cell {}{}{}{}{}{}7{}9}}|1|6|{\cell {}{}{}{}5{}{}{}{}}|4|{\cell {}2{}{}{}{}7{}{}}|8|{\cell {}23{}{\sout 5}{}{}89}|{\cell {}2{}{}{}{}7{}{}}|.
|{\makebox[0pt]{\hspace{-2em}\large b3}{\cell {}{}{}{}{}67{}9}}|{\cell {}{}{}{}{}67{}9}|{\cell {}2{}{}{}{}{}89}|3|{\cell 1{}{}{}{}{}7{}{}}|4|5|{\cell 12{}{}{}6789}|{\cell 12{}{}{}{}78{}}|.
|{\makebox[0pt]{\hspace{-2.5em}\large b4}2}|6|{\cell 1{}3{}{}{}{}{}9}|{\cell {}{}3{}{}{}{}{}9}|{\cell {}{}{}45{}{}{}{}}|{\cell 1{}345{}{}{}9}|8|{\cell 1{}34{}{}{}{}9}|7|.
|{\makebox[0pt]{\hspace{-2em}\large b5}{\cell {}{}{}4{}67{}9}}|{\cell {}23{}{}{}{}{}{}}|{\cell {}2{}{}{}{}{}8{}}|{\cell 123456789}|{\cell {}{}{}{}56{}89}|{\cell 12{}45{}78{}}|{\cell 1{}34{}6{}{}{}}|{\cell 123456789}|{\cell {}{}{}45{}{}{}{}}|.
|{\makebox[0pt]{\hspace{-2em}\large b6}{\cell {}{}{}{}{}67{}9}}|8|4|{\cell 1{}{}{}5{}7{}{}}|2|{\cell {}{}{}{}{}6{}{}9}|{\cell 1{}{}{}{}6{}{}{}}|{\cell 1{}{}{}567{}9}|3|.
|{\makebox[0pt]{\hspace{-2.5em}\large b7}{\cell {}{}{}{}{}{}{}{}9}}|{\cell 12{}4{}{}{}8{}}|{\cell 123{}{}{}{}89}|7|6|{\cell {}{}{}4{}{}{}{}{}}|{\cell 1{}34{}{}{}{}9}|5|{\cell {}{}3{}{}{}{}{}9}|.
|{\makebox[0pt]{\hspace{-2.5em}\large b8}2}|{\cell {}{}3{}5{}{}8{}}|4|{\cell 1{}3{}5{}{}{}{}}|{\cell {}{}3{}{}{}{}{}{}}|9|{\cell 1{}3{}{}{}7{}{}}|{\cell 1{}3{}{}{}78{}}|6|.
|{\makebox[0pt]{\hspace{-2.5em}\large b9}5}|{\cell 1{}34{}67{}{}}|{\cell {}{}3{}{}{}{}8{}}|{\cell 1{}{}4{}{}{}{}{}}|9|2|{\cell 1{}34{}67{}{}}|{\cell 1{}3{}{}{}78{}}|{\cell 1{}{}{}{}{}78{}}.
\end{sudoku}
\caption{An updated annotated instance of Figure \ref{fig:bnView} in bn-view.}
\label{fig:bnUpdate}
\end{figure}










\chapter{Techniques for solving Sudoku}
\label{sec:Techniques}

To solve an Sudoku instance, there are loads of rules could be used to support finding a solution for an Sudoku instance. These rules are roughly divided into two different type of techniques, one is called “Direct Elimination Technique”, and another is“Candidates Elimination Techniques".

``Direct Elimination Technique'' itself is a rule which could be easily applied which does not require to used on an annotated Sudoku instance. It eliminates the impossibilities from the cells through analysing the existing numbers given in the Suodku instance in order. From the human point of view, it is rather simple than ``Candidates Elimination Techniques'' to used on solving an Sudoku instance. However, for the computer point of view, it is not efficient and time consuming.

Consequently, here comes another technique ``Candidates Elimination Techniques'' are provided to solve an Sudoku instance. To apply any rules of this technique, an annotated Sudoku instance is necessary to begin(see section \ref{sec:Annotatedinstances} how to generate an annotated instance). 

Generally, most of the Sudoku instance could simply find a solution by apply naked single and hidden single rules repeatedly. However, there are many Sudoku instances have been designed in different difficulty. Thus other advanced rules are also necessary to support on finding a solution for a Sudoku instance.

Naked single and hidden single rules are usually the most basic rules of Candidates Elimination Techniques to begin the game. When the annotated instance is not able to get any more candidates eliminated, more powerful rules come. More powerful rules are: naked pair, hidden pair, naked triplet, hidden triplet, naked quad and hidden quad. Moreover, if more powerful rules still could not satisfy the annotated instance, we have the advanced rules, X-wing, XY-wing, XYZ-wing, WXYZ-wing, and Swordfish to solve those difficult Sudoku instance. Nevertheless, advanced rule are rarely used in most of the time unless high difficulty Sudoku instance has appeared.

\section{Basic rules}
\label{sec:Basicrules}
The basic rules in ``Candidates Elimination Techniques'' are naked single rule and the hidden single rule.

\subsection{Naked Single}
\label{sec:Nakedsingle}

Naked single is in a situation that ``there is a cell has only one candidate possibly to be filled into this cell''. If there is one cell has an equation written like $A'\ (i,j) =\ (\set{v}, \set{1,\dots,N} \setminus \set{v})$, then eliminate candidate $v$ from all other cells in the same row, same column, or same block.


When a proper \textbf{Sudoku rule} (which said in section \ref{sec:naturesudokurules}) has been successful applied, certain candidates must have excluded from some cells. If any of the cells in a mapped annotated instance $A'$ has an annotation like $A'\ (i,j) =\ (\set{v}, \set{1,\dots,N} \setminus \set{v})$, then all other cells shared a row, a column, or a block with the cell $A'\ (i,j)$ will have candidate $v$, excluded and the annotation updated. The annotation for the cells will be updated as followings:

And a modification of the instance after we have found a cell which tally with the prerequisite is eliminating candidate in all other cells which shared a unit with the cell. Figure \ref{fig:rcWhole} is an instance which has two cells are obviously matched to the prerequisite. Candidate 8 in $(r_{1},\ c_{6})$ and $(r_{2},\ c_{3})$ it the only one possibility to fill into both cells, thus, the this instance will be updated to Figure \ref{fig:nakedsingle}. Candidate 8 has been filled into both cells, and this candidate will cross out from other cells which shared a row, column or block with $(r_{1},\ c_{6})$ and $(r_{2},\ c_{3})$.

\begin{figure}[htbp]
\begin{sudoku}
|1|{\cell {}{}3{}56{}{}{}}|{\cell {}{}3{}{}67{\sout 8}{}}|2|{\cell {}{}{}{}{}6{}{\sout 8}9}|{\cell {}{}{}{}{}{}{}8{}}|{\cell {}{}{}{}5{}{}{\sout 8}9}|{\cell {}{}3{}{}{}{}{\sout 8}9}|4|.
|{\cell {}2{}{}{}{}{}{\sout 8}{}}|9|{\cell {}{}{}{}{}{}{}8{}}|5|{\cell {}{}{}4{}{}{}{\sout 8}{}}|3|6| 7|{\cell 12{}{}{}{}{}8{}}|.
|{\cell {}23{}56{}{\sout 8}{}}|4|{\cell {}{}3{}{}6{}{\sout 8}{}}|{\cell 1{}{}{}{}6{}{\sout 8}9}|7|{\cell 1{}{}{}{}{}{}{\sout 8}{}}|{\cell 12{}{}5{}{}89}|{\cell {}{}3{}{}{}{}89}|{\cell 123{}{}{}{}8{}}|.
|{\cell {}{}3{}{}6{}8{}}|1|{\cell {}{}34{}6{}{\sout 8}{}}|{\cell {}{}{}4{}678{}}|{\cell {}234{}6{}8{}}|{\cell {}2{}4{}{}7{\sout 8}{}}|{\cell {}{}{}4{}{}78{}}|5|9|.
|{\cell {}{}{}{}56{}8{}}|{\cell {}{}{}{}56{}{}{}}|2|{\cell 1{}{}4{}6789}|{\cell {}{}{}456{}89}|{\cell 1{}{}45{}7{\sout 8}{}}|3|{\cell {}{}{}4{}{}{}8{}}|{\cell 1{}{}{}{}678{}}|.
|9|7|{\cell {}{}34{}6{}{\sout 8}{}}|{\cell 1{}{}4{}6{}8{}}|{\cell {}{}3456{}8{}}|{\cell 1{}{}45{}{}{\sout 8}{}}|{\cell 1{}{}4{}{}{}8{}}|2|{\cell 1{}{}{}{}6{}8{}}|.
|{\cell {}23{}{}{}7{}{}}|{\cell {}23{}{}{}{}{}{}}|{\cell {}{}3{}{}{}7{}9}|{\cell {}{}{}4{}{}78{}}|1|{\cell {}2{}45{}7{\sout 8}{}}|{\cell {}2{}4{}{}789}|6|{\cell {}23{}{}{}78{}}|.
|{\cell {}2{}{}{}67{}{}}|8|5|3| {\cell {}2{}4{}{}{}{}{}}|9| {\cell {}2{}4{}{}7{}{}}|1| {\cell {}2{}{}{}{}7{}{}}|.
|4| {\cell {}23{}{}{}{}{}{}}|{\cell 1{}3{}{}{}7{}9}|{\cell {}{}{}{}{}{}78{}}|{\cell {}2{}{}{}{}{}8{}}|6|{\cell {}2{}{}{}{}789}|{\cell {}{}3{}{}{}{}89}| 5|.
\end{sudoku}
\caption{Example in naked single}
\label{fig:nakedsingle}
\end{figure}

\subsection{Hidden Single}
\label{sec:Hidden Single}

Hidden single is under a situation which has one candidate could only be used in one of the cells in a row, a column, or a block, but this cell has more than one candidates. However, this technique may sometimes replaced by Cross-Hatching ??? only use defined/explained concepts!!! XXX

 which is a technique that simply find a place for a specific candidate by scanning that number vertically and horizontally without an annotated instance needed. In the Figure \ref{fig:hiddensingle}, they are three cells have matched to this rule. In the block $b_{1}$, candidate 7 could only be used in cell $(r_{1}, c_{3})$ even though that the cell has other candidates in. In the block $b_{7}$, candidate 1 could only be used in cell $(r_{9}, c_{3})$ even though that the cell has others, and so the cell $(r_{7}, c_{6})$. As we said before, this cell could actually be solved by Cross-Hatching. For example, assume candidate number 1 is being scanning to fill in in block $b_{5}$, block $b_{1}$ has candidate 1 in $r_{1}$ so the cell $(r_{7}, c_{1})$ and $(r_{8}, c_{1})$ are excluded, $b_{4}$ has it in $r_{2}$ so the cell $(r_{7}, c_{2})$ and $(r_{9}, c_{2})$ are excluded, $b_{8}$ has it in $c_{5}$ so the cell $(r_{7}, c_{3})$ are excluded. All the cells have been excluded expect $(r_{9}, c_{3})$, thus, for block $b_{5}$, we can know $(r_{9}, c_{3})$ is the one for candidate $n_{1}$ in block $b_{5}$.

\begin{figure}[htbp]
\begin{sudoku}
|1|{\cell {}{}3{}56{}{}{}}|{\cell {}{}3{}{}6{\underline 7}{}{}}|2|{\cell {}{}{}{}{}6{}{}9}|8|{\cell {}{}{}{}5{}{}{}9}|{\cell {}{}3{}{}{}{}{}9}|4|.
|2|9|8|5|4|3|6| 7|1|.
|{\cell {}{}3{}56{}{}{}}|4|{\cell {}{}3{}{}6{}{}{}}|{\cell 1{}{}{}{}6{}{}9}|7|{\cell 1{}{}{}{}{}{}{}{}}|{\cell {}2{}{}5{}{}89}|{\cell {}{}3{}{}{}{}89}|{\cell 123{}{}{}{}8{}}|.
|{\cell {}{}3{}{}6{}8{}}|1|{\cell {}{}34{}6{}{}{}}|{\cell {}{}{}4{}678{}}|{\cell {}234{}6{}8{}}|{\cell {}2{}4{}{}7{}{}}|{\cell {}{}{}4{}{}78{}}|5|9|.
|{\cell {}{}{}{}56{}8{}}|{\cell {}{}{}{}56{}{}{}}|2|{\cell 1{}{}4{}6789}|{\cell {}{}{}456{}89}|{\cell 1{}{}45{}7{}{}}|3|{\cell {}{}{}4{}{}{}8{}}|{\cell {}{}{}{}{}678{}}|.
|9|7|{\cell {}{}34{}6{}{}{}}|{\cell 1{}{}4{}6{}8{}}|{\cell {}{}3456{}8{}}|{\cell 1{}{}45{}{}{}{}}|{\cell 1{}{}4{}{}{}8{}}|2|{\cell {}{}{}{}{}6{}8{}}|.
|{\cell {}{}3{}{}{}7{}{}}|{\cell {}23{}{}{}{}{}{}}|{\cell {}{}3{}{}{}7{}9}|{\cell {}{}{}4{}{}78{}}|1|{\cell {}2{}4{\underline 5}{}7{}{}}|{\cell {}2{}4{}{}789}|6|{\cell {}23{}{}{}78{}}|.
|{\cell {}{}{}{}{}67{}{}}|8|5|3| {\cell {}2{}4{}{}{}{}{}}|9| {\cell {}2{}4{}{}7{}{}}|1| {\cell {}2{}{}{}{}7{}{}}|.
|4| {\cell {}23{}{}{}{}{}{}}|{\cell {\underline 1}{}3{}{}{}7{}9}|{\cell {}{}{}{}{}{}78{}}|{\cell {}2{}{}{}{}{}8{}}|6|{\cell {}2{}{}{}{}789}|{\cell {}{}3{}{}{}{}89}| 5|.
\end{sudoku}
\caption{Example in hidden single}
\label{fig:hiddensingle}
\end{figure}



\subsection{Intersection Removal}
\label{sec:Intersection Removal}
Sometimes, advanced techniques is needed to be used on supporting more difficult puzzle whilst both naked single and hidden single are not able to find out any way out. The intersection removal technique provides an advanced elimination skill to exclude more impossibilities out of the row, or the column which are in one object block. The prerequisites for this techniques to be utilized are, 
\begin{enumerate}
\item All the cells in one block holds a specific candidate just right in the same row or column or
\item All the cells in one row or one column hold a possible specific candidate just right in the same block.
\end{enumerate}
If the first prerequisite is tenable, then the specific candidate could be excluded from all other cells in the same row or column. And if the second prerequisite is tenable, the the specific candidate could be excluded from all other cells in the same block. 

\begin{figure}[htbp]
\begin{sudoku}
|2|{\cell {}{}{}{}56789}|{\cell {}{}{}{}5{}78{}}|{\cell {}{}3456{}{}9}|{\cell {}{}345{}789}|{\cell {}{}{}456789}|1|{\cell {}{}{}4{}67{}{}}|{\cell {}{}{}{}{}67{}9}|.
|1|4|{\cell {}{}{}{}5{}7{}{}}|{\cell {}{}{}{}56{}{}9}|2|{\cell {}{}{}{}567{}9}|{\cell {}{}{}{}{}67{}9}|8|3|.
|{\cell {}{}{}{}{}67{}9}|{\cell {}{}{}{}{}6789}|3|{\cell {}{}{}4{}6{}{}9}|1|{\cell {}{}{}4{}6789}|5|{\cell {}2{}4{}67{}{}}|{\cell {}2{}{}{}67{}9}|.
|{\cell {}{}345{}7{}{}}|{\cell 123{}5{}7{}{}}|{\cell {}2{}45{}7{}{}}|{\cell {}{}3456{}{}9}|{\cell {}{}345{}{}89}|{\cell {}{}{}456{}89}|{\cell {}{}{}4{}6789}|{\cell 12{}4{}67{}{}}|{\cell 12{}{}{}6789}|.
|{\cell {}{}34{}{}{}{}{}}|{\cell 123{}{}{}{}{}{}}|6|7|{\cell {}{}34{}{}{}89}|{\cell {}{}{}4{}{}{}89}|{\cell {}{}{}4{}{}{}89}|5|{\cell 12{}{}{}{}{}89}|.
|8|{\cell {}{}{}{}5{}7{}{}}|9|2|{\cell {}{}{}45{}{}{}{}}|1|3|{\cell {}{}{}4{}67{}{}}|{\cell {}{}{}{}{}67{}{}}|.
|{\cell {}{}{}4567{}9}|{\cell {}{}{}{}56789}|{\cell {}{}{}45{}78{}}|{\cell {}{}{}45{}{}{}9}|{\cell {}{}{}45{}7{}9}|3|2|{\cell 1{}{}{}{}67{}{}}|{\cell 1{}{}{}5678{}}|.
|{\cell {}{}{}4567{}9}|{\cell {}{}{}{}567{}9}|1|8|{\cell {}{}{}45{}7{}9}|2|{\cell {}{}{}{}{}67{}{}}|3|{\cell {}{}{}{}567{}{}}|.
|{\cell {}{}3{}5{}7{}{}}|{\cell {}23{}5{}78{}}|{\cell {}2{}{}5{}78{}}|1|6|{\cell {}{}{}{}5{}7{}{}}|{\cell {}{}{}{}{}{}78{}}|9|4|.
\end{sudoku}
\caption{Example in intersection removal}
\label{fig:intersectionremoval}
\end{figure}

For example, in \ref{fig:intersectionremoval} candidate $n_{4}$ happens to appear in both $(r_{1},\ c_{8})$ and $(r_{3},\ c_{8})$ in block $b_{3}$ which just right in the same column $c_{8}$, hence, this candidate $n_{4}$ then could be excluded from all other cells $(r_{4},\ c_{8})$ and $(r_{6},\ c_{8})$ in the same column $c_{8}$.




\section{Intermediate rules}
\label{sec:Intermediaterules}

\subsection{Naked Pair}
\label{sec:Naked Pair}

Naked Pair could hold when any of these three prerequisites hold, which are, 
\begin{itemize}
\item in a row: two cells happen to shared a row and have only two candidate left in the cells.
\item in a column: two cells happen to shared a column and have only two candidate left in the cells.
\item or in a block: two cells happen to shared a row and have only two candidate left in the cells.
\end{itemize}
The first two prerequisites above could be mixed and matched with the last one, such like two cells have shared a block and also a column or two cells have shared a block and also a row. When one prerequisite has been matched, then all other cells shared the same row, column, or block could have these two candidates excluded. Because the situation told that these two candidates could only be used in these two cells, and no longer able to be used for others. For example, in the instance Figure \ref{fig:nakedpairb4} which is shown, candidates 2 and 3 are the \textbf{only} two candidates left in the cell $(r_{7},\ c_{2})$ and $(r_{9},\ c_{2})$ which happened to be in a column and also a block, it tells that 2 or 3 could only be used in these cells, no longer for others in the same column or same block. Thus, candidates 2 and 3 will be excluded in this column and the block which $(r_{7},\ c_{2})$ and $(r_{9},\ c_{2})$ are both in to a consequence like Figure \ref{fig:nakedpairaf}. 


\begin{figure}[htbp]
\begin{sudoku}
|1|{\cell {}{}3{}56{}{}{}}|{\cell {}{}3{}{}67{}{}}|2|{\cell {}{}{}{}{}6{}{}9}|8|{\cell {}{}{}{}5{}{}{}9}|{\cell {}{}3{}{}{}{}{}9}|4|.
|{\cell {}2{}{}{}{}{}{}{}}|9|8|5|{\cell {}{}{}4{}{}{}{}{}}|3|6| 7|{\cell 12{}{}{}{}{}8{}}|.
|{\cell {}23{}56{}{}{}}|4|{\cell {}{}3{}{}6{}{}{}}|{\cell 1{}{}{}{}6{}{}9}|7|{\cell 1{}{}{}{}{}{}{}{}}|{\cell 12{}{}5{}{}89}|{\cell {}{}3{}{}{}{}89}|{\cell 123{}{}{}{}8{}}|.
|{\cell {}{}3{}{}6{}8{}}|1|{\cell {}{}34{}6{}{}{}}|{\cell {}{}{}4{}678{}}|{\cell {}234{}6{}8{}}|{\cell {}2{}4{}{}7{}{}}|{\cell {}{}{}4{}{}78{}}|5|9|.
|{\cell {}{}{}{}56{}8{}}|{\cell {}{}{}{}56{}{}{}}|2|{\cell 1{}{}4{}6789}|{\cell {}{}{}456{}89}|{\cell 1{}{}45{}7{}{}}|3|{\cell {}{}{}4{}{}{}8{}}|{\cell 1{}{}{}{}678{}}|.
|9|7|{\cell {}{}34{}6{}{}{}}|{\cell 1{}{}4{}6{}8{}}|{\cell {}{}3456{}8{}}|{\cell 1{}{}45{}{}{}{}}|{\cell 1{}{}4{}{}{}8{}}|2|{\cell 1{}{}{}{}6{}8{}}|.
|{\cell {}23{}{}{}7{}{}}|{\cell {}23{}{}{}{}{}{}}|{\cell {}{}3{}{}{}7{}9}|{\cell {}{}{}4{}{}78{}}|1|{\cell {}2{}45{}7{}{}}|{\cell {}2{}4{}{}789}|6|{\cell {}23{}{}{}78{}}|.
|{\cell {}2{}{}{}67{}{}}|8|5|3| {\cell {}2{}4{}{}{}{}{}}|9| {\cell {}2{}4{}{}7{}{}}|1| {\cell {}2{}{}{}{}7{}{}}|.
|4| {\cell {}23{}{}{}{}{}{}}|{\cell 1{}3{}{}{}7{}9}|{\cell {}{}{}{}{}{}78{}}|{\cell {}2{}{}{}{}{}8{}}|6|{\cell {}2{}{}{}{}789}|{\cell {}{}3{}{}{}{}89}| 5|.
\end{sudoku}
\caption{Example in haked pair - before}
\label{fig:nakedpairb4}
\end{figure}

\begin{figure}[htbp]
\begin{sudoku}
|1|{\cell {}{}{\sout 3}{}56{}{}{}}|{\cell {}{}3{}{}67{}{}}|2|{\cell {}{}{}{}{}6{}{}9}|8|{\cell {}{}{}{}5{}{}{}9}|{\cell {}{}3{}{}{}{}{}9}|4|.
|{\cell {}2{}{}{}{}{}{}{}}|9|8|5|{\cell {}{}{}4{}{}{}{}{}}|3|6| 7|{\cell 12{}{}{}{}{}8{}}|.
|{\cell {}23{}56{}{}{}}|4|{\cell {}{}3{}{}6{}{}{}}|{\cell 1{}{}{}{}6{}{}9}|7|{\cell 1{}{}{}{}{}{}{}{}}|{\cell 12{}{}5{}{}89}|{\cell {}{}3{}{}{}{}89}|{\cell 123{}{}{}{}8{}}|.
|{\cell {}{}3{}{}6{}8{}}|1|{\cell {}{}34{}6{}{}{}}|{\cell {}{}{}4{}678{}}|{\cell {}234{}6{}8{}}|{\cell {}2{}4{}{}7{}{}}|{\cell {}{}{}4{}{}78{}}|5|9|.
|{\cell {}{}{}{}56{}8{}}|{\cell {}{}{}{}56{}{}{}}|2|{\cell 1{}{}4{}6789}|{\cell {}{}{}456{}89}|{\cell 1{}{}45{}7{}{}}|3|{\cell {}{}{}4{}{}{}8{}}|{\cell 1{}{}{}{}678{}}|.
|9|7|{\cell {}{}34{}6{}{}{}}|{\cell 1{}{}4{}6{}8{}}|{\cell {}{}3456{}8{}}|{\cell 1{}{}45{}{}{}{}}|{\cell 1{}{}4{}{}{}8{}}|2|{\cell 1{}{}{}{}6{}8{}}|.
|{\cell {}{\sout 2}{\sout 3}{}{}{}7{}{}}|{\cell {}23{}{}{}{}{}{}}|{\cell {}{}{\sout 3}{}{}{}7{}9}|{\cell {}{}{}4{}{}78{}}|1|{\cell {}2{}45{}7{}{}}|{\cell {}2{}4{}{}789}|6|{\cell {}23{}{}{}78{}}|.
|{\cell {}{\sout 2}{}{}{}67{}{}}|8|5|3| {\cell {}2{}4{}{}{}{}{}}|9| {\cell {}2{}4{}{}7{}{}}|1| {\cell {}2{}{}{}{}7{}{}}|.
|4| {\cell {}23{}{}{}{}{}{}}|{\cell 1{}{\sout 3}{}{}{}7{}9}|{\cell {}{}{}{}{}{}78{}}|{\cell {}2{}{}{}{}{}8{}}|6|{\cell {}2{}{}{}{}789}|{\cell {}{}3{}{}{}{}89}| 5|.
\end{sudoku}
\caption{Example in naked pair - after}
\label{fig:nakedpairaf}
\end{figure}


\subsection{Hidden Pair}
\label{sec:Hidden Pair}
The prerequisite for Hidden Pair is, that two of all the possible candidates are only appeared in two cells in a shared unit (a row, a column, or a block). Once the prerequisite is satisfied, all other candidates in these two cells could be excluded surely. 
For instance, in the block $b_{2}$ of Figure \ref{fig:hiddenPair}. $(r_{1},\ c_{5})$ and $(r_{3},\ c_{4})$ do not have only two candidates $n_{6}$ and $n_{9}$ in the cells only, but candidates $n_{6}$ and $n_{9}$ are restricted to these two cells only. Hence, the prerequisite is satisfied, thus all candidates expect $n_{6}$ and $n_{9}$ could all be excluded from these two cells.

\begin{figure}[htbp]
\begin{sudoku}
|1|{\cell {}{}3{}56{}{}{}}|{\cell {}{}3{}{}67{}{}}|2|{\cell {}{}{}{}{}6{}{}9}|8|{\cell {}{}{}{}5{}{}{}9}|{\cell {}{}3{}{}{}{}{}9}|4|.
|2|9|8|5|4|3|6| 7|1|.
|{\cell {}{}3{}56{}{}{}}|4|{\cell {}{}3{}{}6{}{}{}}|{\cell 1{}{}{}{}6{}{}9}|7|{\cell 1{}{}{}{}{}{}{}{}}|{\cell {}2{}{}5{}{}89}|{\cell {}{}3{}{}{}{}89}|{\cell 123{}{}{}{}8{}}|.
|{\cell {}{}3{}{}6{}8{}}|1|{\cell {}{}34{}6{}{}{}}|{\cell {}{}{}4{}678{}}|{\cell {}234{}6{}8{}}|{\cell {}2{}4{}{}7{}{}}|{\cell {}{}{}4{}{}78{}}|5|9|.
|{\cell {}{}{}{}56{}8{}}|{\cell {}{}{}{}56{}{}{}}|2|{\cell 1{}{}4{}6789}|{\cell {}{}{}456{}89}|{\cell 1{}{}45{}7{}{}}|3|{\cell {}{}{}4{}{}{}8{}}|{\cell {}{}{}{}{}678{}}|.
|9|7|{\cell {}{}34{}6{}{}{}}|{\cell 1{}{}4{}6{}8{}}|{\cell {}{}3456{}8{}}|{\cell 1{}{}45{}{}{}{}}|{\cell 1{}{}4{}{}{}8{}}|2|{\cell {}{}{}{}{}6{}8{}}|.
|{\cell {}{}3{}{}{}7{}{}}|{\cell {}23{}{}{}{}{}{}}|{\cell {}{}3{}{}{}7{}9}|{\cell {}{}{}4{}{}78{}}|1|{\cell {}2{}4{\underline 5}{}7{}{}}|{\cell {}2{}4{}{}789}|6|{\cell {}23{}{}{}78{}}|.
|{\cell {}{}{}{}{}67{}{}}|8|5|3| {\cell {}2{}4{}{}{}{}{}}|9| {\cell {}2{}4{}{}7{}{}}|1| {\cell {}2{}{}{}{}7{}{}}|.
|4| {\cell {}23{}{}{}{}{}{}}|{\cell {\underline 1}{}3{}{}{}7{}9}|{\cell {}{}{}{}{}{}78{}}|{\cell {}2{}{}{}{}{}8{}}|6|{\cell {}2{}{}{}{}789}|{\cell {}{}3{}{}{}{}89}| 5|.
\end{sudoku}
\caption{Example in hidden pair}
\label{fig:hiddenPair}
\end{figure}





\subsection{Naked Triplet}
\label{sec:Naked Triplet}

Naked Triplet could hold when any of these three prerequisites hold, which are, 
\begin{itemize}
\item in a row: three cells happen to shared a row and have common three candidate left in the cells.
\item in a column: three cells happen to shared a column and have common three candidate left in the cells.
\item or in a block: three cells happen to shared a row and have common three candidate left in the cells.
\end{itemize}
By this technique, other cells in the same row, column, or the same block could safely have these three candidates excluded expect three cells which have these candidates. 

\begin{figure}[htbp]
\begin{sudoku}
|1|{\cell {}{}{\textbf 3}{}{\textbf 5}{\textbf 6}{}{}{}}|{\cell {}{}3{}{}67{}{}}|2|{\cell {}{}{}{}{}6{}{}9}|8|{\cell {}{}{}{}5{}{}{}9}|{\cell {}{}3{}{}{}{}{}9}|4|.
|2|9|8|5|4|3|6| 7|1|.
|{\cell {}{}{\textbf 3}{}{\textbf{5}}{\textbf 6}{}{}{}}|4|{\cell {}{}{\textbf 3}{}{}{\textbf 6}{}{}{}}|{\cell 1{}{}{}{}6{}{}9}|7|{\cell 1{}{}{}{}{}{}{}{}}|{\cell {}2{}{}5{}{}89}|{\cell {}{}3{}{}{}{}89}|{\cell 123{}{}{}{}8{}}|.
|{\cell {}{}3{}{}6{}8{}}|1|{\cell {}{}34{}6{}{}{}}|{\cell {}{}{}4{}678{}}|{\cell {}234{}6{}8{}}|{\cell {}2{}4{}{}7{}{}}|{\cell {}{}{}4{}{}78{}}|5|9|.
|{\cell {}{}{}{}56{}8{}}|{\cell {}{}{}{}56{}{}{}}|2|{\cell 1{}{}4{}6789}|{\cell {}{}{}456{}89}|{\cell 1{}{}45{}7{}{}}|3|{\cell {}{}{}4{}{}{}8{}}|{\cell {}{}{}{}{}678{}}|.
|9|7|{\cell {}{}34{}6{}{}{}}|{\cell 1{}{}4{}6{}8{}}|{\cell {}{}3456{}8{}}|{\cell 1{}{}45{}{}{}{}}|{\cell 1{}{}4{}{}{}8{}}|2|{\cell {}{}{}{}{}6{}8{}}|.
|{\cell {}{}3{}{}{}7{}{}}|{\cell {}23{}{}{}{}{}{}}|{\cell {}{}3{}{}{}7{}9}|{\cell {}{}{}4{}{}78{}}|1|{\cell {}2{}45{}7{}{}}|{\cell {}2{}4{}{}789}|6|{\cell {}23{}{}{}78{}}|.
|{\cell {}{}{}{}{}67{}{}}|8|5|3| {\cell {}2{}4{}{}{}{}{}}|9| {\cell {}2{}4{}{}7{}{}}|1| {\cell {}2{}{}{}{}7{}{}}|.
|4| {\cell {}23{}{}{}{}{}{}}|{\cell 1{}3{}{}{}7{}9}|{\cell {}{}{}{}{}{}78{}}|{\cell {}2{}{}{}{}{}8{}}|6|{\cell {}2{}{}{}{}789}|{\cell {}{}3{}{}{}{}89}| 5|.
\end{sudoku}
\caption{Example in naked triplet - before}
\label{fig:nakedtripletb4}
\end{figure}
In the Figure \ref{fig:nakedtripletb4}, candidates $n_{3}$, $n_{5}$ and $n_{6}$ are the common three number which appears in the cells $(r_{1},\ c_{2})$, $(r_{3},\ c_{1})$, and $(r_{3},\ c_{3})$ which has shared the a unit (column) here. Hence, all other cells which also have candidate $n_{3}$, $n_{5}$ and $n_{6}$ in a shared unit (block $b_{1}$) could have these three candidates excluded, because these candidates are not longer possible to be the solution for them. The Figure{fig:nakedtripletaf} is the updated instance after the exclusion has been made.

\begin{figure}[htbp]
\begin{sudoku}
|1|{\cell {}{}3{}56{}{}{}}|{\cell {}{}{\sout 3}{}{}{\sout 6}7{}{}}|2|{\cell {}{}{}{}{}6{}{}9}|8|{\cell {}{}{}{}5{}{}{}9}|{\cell {}{}3{}{}{}{}{}9}|4|.
|2|9|8|5|4|3|6| 7|1|.
|{\cell {}{}3{}56{}{}{}}|4|{\cell {}{}3{}{}6{}{}{}}|{\cell 1{}{}{}{}6{}{}9}|7|{\cell 1{}{}{}{}{}{}{}{}}|{\cell {}2{}{}5{}{}89}|{\cell {}{}3{}{}{}{}89}|{\cell 123{}{}{}{}8{}}|.
|{\cell {}{}3{}{}6{}8{}}|1|{\cell {}{}34{}6{}{}{}}|{\cell {}{}{}4{}678{}}|{\cell {}234{}6{}8{}}|{\cell {}2{}4{}{}7{}{}}|{\cell {}{}{}4{}{}78{}}|5|9|.
|{\cell {}{}{}{}56{}8{}}|{\cell {}{}{}{}56{}{}{}}|2|{\cell 1{}{}4{}6789}|{\cell {}{}{}456{}89}|{\cell 1{}{}45{}7{}{}}|3|{\cell {}{}{}4{}{}{}8{}}|{\cell {}{}{}{}{}678{}}|.
|9|7|{\cell {}{}34{}6{}{}{}}|{\cell 1{}{}4{}6{}8{}}|{\cell {}{}3456{}8{}}|{\cell 1{}{}45{}{}{}{}}|{\cell 1{}{}4{}{}{}8{}}|2|{\cell {}{}{}{}{}6{}8{}}|.
|{\cell {}{}3{}{}{}7{}{}}|{\cell {}23{}{}{}{}{}{}}|{\cell {}{}3{}{}{}7{}9}|{\cell {}{}{}4{}{}78{}}|1|{\cell {}2{}45{}7{}{}}|{\cell {}2{}4{}{}789}|6|{\cell {}23{}{}{}78{}}|.
|{\cell {}{}{}{}{}67{}{}}|8|5|3| {\cell {}2{}4{}{}{}{}{}}|9| {\cell {}2{}4{}{}7{}{}}|1| {\cell {}2{}{}{}{}7{}{}}|.
|4| {\cell {}23{}{}{}{}{}{}}|{\cell 1{}3{}{}{}7{}9}|{\cell {}{}{}{}{}{}78{}}|{\cell {}2{}{}{}{}{}8{}}|6|{\cell {}2{}{}{}{}789}|{\cell {}{}3{}{}{}{}89}| 5|.
\end{sudoku}
\caption{Example in naked triplet - after}
\label{fig:nakedtripletaf}
\end{figure}


\subsection{Hidden Triplet}
\label{sec:Hidden Triplet}
Which is a situation evolved from hidden single and hidden pair. The prerequisite of this technique is three of the possible candidates are only happened in only three cells in one unit (either row, column, or block).
In the Figure \ref{fig:hiddentriplet}, candidate $n_{2}$, $n_{5}$, and $n_{7}$ (in boldface font) are only appeared in $c_{4}$, $c_{5}$, and $c_{7}$. Hence, all the candidates apart from $n_{2}$, $n_{5}$, and $n_{7}$ could then be excluded from these three cells.

\begin{figure}[htbp]
\setlength{\tabcolsep}{2.5pt}
\renewcommand{\arraystretch}{1.6}
\hspace{1em}
\begin{center}
\begin{tabular}{{|>{\centering\arraybackslash}m{0.3in}| >{\centering\arraybackslash}m{0.3in}| >{\centering\arraybackslash}m{0.3in}| >{\centering\arraybackslash}m{0.3in}| >{\centering\arraybackslash}m{0.3in}| >{\centering\arraybackslash}m{0.3in}| >{\centering\arraybackslash}m{0.3in}| >{\centering\arraybackslash}m{0.3in}| >{\centering\arraybackslash}m{0.3in}| }}
\hline
{\cell {}{}{}{}{}6{}{}9} &{\cell {}{}{}{}{}6{}8{}} & \LARGE 1 &{\cell {}{\textbf 2}{}{}{\textbf 5}6{}8{}}  & {\cell {}{\textbf 2}{}{}{\textbf 5}{}{\textbf 7}89} & \LARGE 3 & {\cell {}{}{}{}{\textbf 5}{}{\textbf 7}{}9} & \LARGE 4 & {\cell {}{}{}{}{}6{}{}9} \\
\hline
\end{tabular}
\end{center}
\caption{Example in hidden triplet}
\label{fig:hiddentriplet}
\end{figure}



\subsection{Naked Quad}
\label{sec:Naked Quad}
It is in the situation while four cells in a row, a column, or a block are having common four candidate digits left in the cells. If the prerequisite is existed, then it will bring an updated instance which has no these four candidates in other cells which shared a same unit anymore. Figure ~\ref{fig:nakedquad} is an instance which in naked quad situation, in the block $b_{5}$, cell $(r_{5},\ c_{6})$, $(r_{6},\ c_{4})$, $(r_{6},\ c_{5})$, and$(r_{6},\ c_{6})$ have 4 different candidates $n_{2},\ n_{3},\ n_{5},\ and\ n_{9}$ appear commonly in these four cells. Consequently, candidates $n_{2},\ n_{3},\ n_{5},\ and\ n_{9}$ in other cells in the block $b_{5}$ will be excluded (Figure because these four candidates will not longer possible to be used in other cells anymore.

\begin{figure}[htbp]
\begin{sudoku}
|{\cell 1{}{}{}5{}{}{}{}}|9|4|{\cell 1{}{}{}5{}7{}{}}|3|6|2|{\cell 1{}{}{}{}{}78{}}|{\cell 1{}{}{}{}{}{}8{}}|.
|{\cell 12{}{}{}6{}{}{}}|7|{\cell 12{}{}{}6{}8{}}|{\cell 12{}{}{}{}{}{}9}|{\cell 12{}{}{}{}{}{}9}|{\cell {}{}{}{}{}{}{}89}|4|3|5|.
|{\cell 12{}{}5{}{}{}{}}|3|{\cell 12{}{}5{}{}8{}}|4|{\cell 12{}{}{}{}7{}{}}|{\cell {}{}{}{}5{}{}8{}}|{\cell {}{}{}{}67{}{}{}}|{\cell 1{}{}{}{}67{}9}|{\cell 1{}{}{}{}{}{}{}9}|.
|{\cell {}{}34{}6{}{}9}|{\cell {}{}{}4{}6{}{}{}}|{\cell {}{}3{}{}6{}{}9}|8|{\cell {}{}{}{}{}67{}{\sout 9}}|1|5|{\cell {}2{}{}{}{}7{}9}|{\cell {}23{}{}{}{}{}9}|.
|{\cell 1{}3{}56{}{}9}|2|{\cell 1{}3{}56{}{}9}|{\cell {}{}{}{}{\sout 5}67{}{\sout 9}}|4|{\cell {}{}3{}5{}{}{}9}|{\cell {}{}{}{}{}{}78{}}|{\cell {}{}{}{}{}{}789}|{\cell {}{}3{}{}{}{}89}|.
|7|8|{\cell {}{}3{}5{}{}{}9}|{\cell {}2{}{}5{}{}{}9}|{\cell {}2{}{}{}{}{}{}9}|{\cell {}{}3{}5{}{}{}9}|1|4|6|.
|8|{\cell 1{}{}{}{}6{}{}{}}|{\cell {}23{}{}6{}{}{}}|{\cell 1{}3{}{}6{}{}{}}|5|7|9|{\cell 12{}{}{}6{}{}{}}|4|.
|{\cell {}2{}{}{}6{}{}9}|{\cell 1{}{}{}56{}{}{}}|7|{\cell 1{}{}{}{}6{}{}9}|{\cell 1{}{}{}{}6{}89}|4|3|{\cell 12{}{}56{}8{}}|{\cell 12{}{}{}{}{}8{}}|.
|{\cell {}{}34{}6{}{}9}|{\cell 1{}{}456{}{}{}}|{\cell {}{}3{}{}6{}{}9}|{\cell 1{}3{}{}6{}{}9}|{\cell 1{}{}{}{}6{}89}|2|{\cell {}{}{}{}{}6{}8{}}|{\cell 1{}{}{}56{}8{}}|7|.
\end{sudoku}
\caption{Example in naked quad}
\label{fig:nakedquad}
\end{figure}




\subsection{Hidden Quad}
\label{sec:Hidden Quad}
This is a rare circumstance which has four candidates only found in four cells in a unit and each cell has at least two of these candidates appear in the cell. Then all other candidates in these four cells would be excluded afterwards as soon as the situation is found. 
In the block $b_{6}$ of Figure \ref{fig:hiddenquad}, candidates $n_{1}$, $n_{4}$, $n_{6}$ and $n_{7}$ (in boldface font) are the only 4 candidates appear in these four cells $(r_{4}, c_{7})$,  $(r_{4}, c_{9})$,  $(r_{6}, c_{7})$ and $(r_{7}, c_{9})$, thus, all other candidates in these four cells will no longer needed.


\begin{figure}[htbp]
\begin{sudoku}
|6|3|2|1|4|5|9|7|8|.
|8|1|{\cell {}{}{}{}5{}7{}{}}|{\cell {}{}3{}{}6{}{}{}}|9|{\cell {}2{}{}{}{}7{}{}}|{\cell {}23{}56{}{}{}}|{\cell {}23{}5{}{}{}{}}|4|.
|{\cell {}{}{}{}5{}{}{}9}|4|{\cell {}{}{}{}5{}7{}9}|{\cell {}{}3{}{}6{}{}{}}|8|{\cell {}2{}{}{}{}7{}{}}|{\cell {}23{}56{}{}{}}|1|{\cell {}23{}56{}{}{}}|.
|{\cell {}{}3{}{}{}{}{}9}|{\cell {}2{}{}{}{}7{}{}}|{\cell {}{}{}4{}{}{}{}9}|8|5|{\cell 1{}3{}{}{}{}{}{}}|{\cell {\textbf 1}{\sout 2}{}{\textbf 4}{}{\textbf 6}{\textbf 7}{}{}}|{\cell {}2{}{}{}{}{}{}9}|{\cell {\textbf 1}{\sout 2}{}{}{}{\textbf 6}{\textbf 7}{}{}}|.
|1|6|{\cell {}{}{}{}5{}{}89}|2|7|4|{\cell {}{}3{}5{}{}8{}}|{\cell {}{}3{}5{}{}89}|{\cell {}{}3{}5{}{}{}{}}|.
|{\cell {}{}3{}5{}{}{}{}}|{\cell {}2{}{}{}{}7{}{}}|{\cell {}{}{}45{}{}8{}}|9|6|{\cell 1{}3{}{}{}{}{}{}}|{\cell {\textbf 1}{\sout 2}{}{\textbf 4}{\sout 5}{}{\textbf 7}{\sout 8}{}}|{\cell {}2{}{}5{}{}8{}}|{\cell {\textbf 1}{\sout 2}{}{}{\sout 5}{}{\textbf 7}{}{}}|.
|4|8|1|5|2|9|{\cell {}{}3{}{}{}7{}{}}|6|{\cell {}{}3{}{}{}7{}{}}|.
|7|5|3|4|1|6|{\cell {}2{}{}{}{}{}8{}}|{\cell {}2{}{}{}{}{}8{}}|9|.
|2|9|6|7|3|8|{\cell 1{}{}{}5{}{}{}{}}|4|{\cell 1{}{}{}5{}{}{}{}}|.
\end{sudoku}
\caption{Example in hidden quad}
\label{fig:hiddenquad}
\end{figure}


\section{Advanced rules}
\label{sec:Advancedrules}


\subsection{X-wing}
\label{sec:X-wing}
An X-Wing technique is rarely used because it only occurs when two rows or two columns (does not apply on blocks) each contain only two cells that have a matching candidate in between. This candidate must appear in both rows and share the same two columns or vice versa. If the condition holds, then this candidate could then be excluded from other cells in both rows and columns.

\begin{figure}
\begin{sudoku}
| |*| | | | | |*| |.
|*|X|*|*|*|*|*|X|*|.
| |*| | | | | |*| |.
| |*| | | | | |*| |.
| |*| | | | | |*| |.
| |*| | | | | |*| |.
| |*| | | | | |*| |.
|*|X|*|*|*|*|*|X|*|.
| |*| | | | | |*| |.
\end{sudoku}
\caption{Pattern in X-wing}
\label{fig:xwingmodel}
\end{figure}

Figure \ref{fig:xwingmodel} is the pattern look which is satisfied a x-wing situation:
\begin{enumerate}
\item Two cells in both two rows and columns
\item and a matching candidate ``X''  is appeared in all four cells.
\end{enumerate}

In the figure \ref{fig:xwing}, candidate $n_{6}$ is the matching candidate in between $(r_{2}, c_{2})$, $(r_{2}, c_{8})$, $(r_{8}, c_{2})$, and $(r_{8}, c_{8})$ which fits the prerequisite that two consecutive cells in both two rows and columns. The prerequisite is satisfied by these four cells and the candidate $n_{6}$, hence, candidate $n_{6}$ is no longer able to existed in other cells in both rows $r_{2}$ and $r_{2}$ and both columns $c_{2}$ and $c_{8}$. And there are only two solutions to sort these four cells out,
\begin{enumerate}
\item candidate $n_{6}$ in $(r_{2}, c_{2})$ and $(r_{8}, c_{8})$,
\item or candidate $n_{6}$ in $(r_{2}, c_{8})$, and $(r_{8}, c_{2})$.
\end{enumerate}
If the first solution is taken, then cell$(r_{2}, c_{8})$ will have candidate 9 filled in, and cell$(r_{8}, c_{2})$ will have candidate 4 filled in. On the contrary, if the second one is chosen, then the solution for celll$(r_{2}, c_{2})$ will be 9 and for cell$(r_{8}, c_{8})$ will be 4. However, none of them could be decided. the only thing could be done is to eliminate candidate 6 in any other cells which in the same row and column.

\begin{figure}
\begin{sudoku}
|{\cell {}{}{}{}{}{}{}{}{}}|*|{\cell {}{}{}{}{}{}{}{}{}}|{\cell {}{}{}{}{}{}{}{}{}}|{\cell {}{}{}{}{}{}{}{}{}}|{\cell {}{}{}{}{}{}{}{}{}}|{\cell {}{}{}{}{}{}{}{}{}}|*|{\cell {}{}{}{}{}{}{}{}{}}|.
|*|{\cell {}{}{}{}{}{6}{}{}{9}}|*|*|*|*|*|{\cell {}{}{}{}{}{6}{}{}{9}}|*|.
|{\cell {}{}{}{}{}{}{}{}{}}|*|{\cell {}{}{}{}{}{}{}{}{}}|{\cell {}{}{}{}{}{}{}{}{}}|{\cell {}{}{}{}{}{}{}{}{}}|{\cell {}{}{}{}{}{}{}{}{}}|{\cell {}{}{}{}{}{}{}{}{}}|*|{\cell {}{}{}{}{}{}{}{}{}}|.
|{\cell {}{}{}{}{}{}{}{}{}}|*|{\cell {}{}{}{}{}{}{}{}{}}|{\cell {}{}{}{}{}{}{}{}{}}|{\cell {}{}{}{}{}{}{}{}{}}|{\cell {}{}{}{}{}{}{}{}{}}|{\cell {}{}{}{}{}{}{}{}{}}|*|{\cell {}{}{}{}{}{}{}{}{}}|.
|{\cell {}{}{}{}{}{}{}{}{}}|*|{\cell {}{}{}{}{}{}{}{}{}}|{\cell {}{}{}{}{}{}{}{}{}}|{\cell {}{}{}{}{}{}{}{}{}}|{\cell {}{}{}{}{}{}{}{}{}}|{\cell {}{}{}{}{}{}{}{}{}}|*|{\cell {}{}{}{}{}{}{}{}{}}|.
|{\cell {}{}{}{}{}{}{}{}{}}|*|{\cell {}{}{}{}{}{}{}{}{}}|{\cell {}{}{}{}{}{}{}{}{}}|{\cell {}{}{}{}{}{}{}{}{}}|{\cell {}{}{}{}{}{}{}{}{}}|{\cell {}{}{}{}{}{}{}{}{}}|*|{\cell {}{}{}{}{}{}{}{}{}}|.
|{\cell {}{}{}{}{}{}{}{}{}}|*|{\cell {}{}{}{}{}{}{}{}{}}|{\cell {}{}{}{}{}{}{}{}{}}|{\cell {}{}{}{}{}{}{}{}{}}|{\cell {}{}{}{}{}{}{}{}{}}|{\cell {}{}{}{}{}{}{}{}{}}|*|{\cell {}{}{}{}{}{}{}{}{}}|.
|*|{\cell {}{}{}{4}{}{6}{}{}{}}|*|*|*|*|*|{\cell {}{}{}{4}{}{6}{}{}{}}|*|.
|{\cell {}{}{}{}{}{}{}{}{}}|*|{\cell {}{}{}{}{}{}{}{}{}}|{\cell {}{}{}{}{}{}{}{}{}}|{\cell {}{}{}{}{}{}{}{}{}}|{\cell {}{}{}{}{}{}{}{}{}}|{\cell {}{}{}{}{}{}{}{}{}}|*|{\cell {}{}{}{}{}{}{}{}{}}|.
\end{sudoku}
\caption{Example in X-wing}
\label{fig:xwing}
\end{figure}



\subsection{XY-wing}
\label{sec:XY-wing}
That has to be applied when two prerequisites are satisfied:
\begin{enumerate}
\item three different cells not in a same unit has different combination from two of three candidates  X, Y, and Z,
\item and two consecutive cells shared a unit (row, column, or block).
\end{enumerate}
If the condition is satisfied, then the cell at the intersect point between two of these three different cells which could have the candidate that two cells both have eliminated.
In the figure \ref{fig:xywing}, in the cell $(r_{1}, c_{1})$, $(r_{1}, c_{8})$ and $(r_{8}, c_{1})$, they are happened to have three candidates $n_{2}$, $n_{3}$ and $n_{7}$ only and a different combination in each. Thus, assume:
\begin{itemize}
\item $X = n_{2},\ Y = n_{3},\ and\ Z = n_{7}$,
\item $(r_{1}, c_{1})$ has XY (candidate $n_{2}$ and $n_{3}$),
\item $(r_{1}, c_{8})$ has YZ (candidate $n_{3}$ and $n_{7}$),
\item $(r_{8}, c_{1})$ has XZ (candidate $n_{2}$ and $n_{7}$).
\end{itemize}
Cell $(r_{8}, c_{8})$ is at the intersect point between cell $(r_{1}, c_{8})$ and $(r_{8}, c_{1})$ which both have candidate $n_{7}$. Thus, if cell $(r_{8}, c_{8})$ also have the same candidate, it could have this removed then.

\begin{figure}[htbp]
\begin{sudoku}
|{\cell {}{\textbf 2}{\textbf 3}{}{}{}{}{}{}}|{\cell {}{}{}{}{}{}{}{}{}}|{\cell {}{}{}{}{}{}{}{}{}}|{\cell {}{}{}{}{}{}{}{}{}}|{\cell {}{}{}{}{}{}{}{}{}}|{\cell {}{}{}{}{}{}{}{}{}}|{\cell {}{}{}{}{}{}{}{}{}}|{\cell {}{}{\textbf 3}{}{}{}{\textbf 7}{}{}}|{\cell {}{}{}{}{}{}{}{}{}}|.
|{\cell {}{}{}{}{}{}{}{}{}}|{\cell {}{}{}{}{}{}{}{}{}}|{\cell {}{}{}{}{}{}{}{}{}}|{\cell {}{}{}{}{}{}{}{}{}}|{\cell {}{}{}{}{}{}{}{}{}}|{\cell {}{}{}{}{}{}{}{}{}}|{\cell {}{}{}{}{}{}{}{}{}}|{\cell {}{}{}{}{}{}{}{}{}}|.
|{\cell {}{}{}{}{}{}{}{}{}}|{\cell {}{}{}{}{}{}{}{}{}}|{\cell {}{}{}{}{}{}{}{}{}}|{\cell {}{}{}{}{}{}{}{}{}}|{\cell {}{}{}{}{}{}{}{}{}}|{\cell {}{}{}{}{}{}{}{}{}}|{\cell {}{}{}{}{}{}{}{}{}}|{\cell {}{}{}{}{}{}{}{}{}}|{\cell {}{}{}{}{}{}{}{}{}}|.
|{\cell {}{}{}{}{}{}{}{}{}}|{\cell {}{}{}{}{}{}{}{}{}}|{\cell {}{}{}{}{}{}{}{}{}}|{\cell {}{}{}{}{}{}{}{}{}}|{\cell {}{}{}{}{}{}{}{}{}}|{\cell {}{}{}{}{}{}{}{}{}}|{\cell {}{}{}{}{}{}{}{}{}}|{\cell {}{}{}{}{}{}{}{}{}}|{\cell {}{}{}{}{}{}{}{}{}}|.
|{\cell {}{}{}{}{}{}{}{}{}}|{\cell {}{}{}{}{}{}{}{}{}}|{\cell {}{}{}{}{}{}{}{}{}}|{\cell {}{}{}{}{}{}{}{}{}}|{\cell {}{}{}{}{}{}{}{}{}}|{\cell {}{}{}{}{}{}{}{}{}}|{\cell {}{}{}{}{}{}{}{}{}}|{\cell {}{}{}{}{}{}{}{}{}}|{\cell {}{}{}{}{}{}{}{}{}}|.
|{\cell {}{}{}{}{}{}{}{}{}}|{\cell {}{}{}{}{}{}{}{}{}}|{\cell {}{}{}{}{}{}{}{}{}}|{\cell {}{}{}{}{}{}{}{}{}}|{\cell {}{}{}{}{}{}{}{}{}}|{\cell {}{}{}{}{}{}{}{}{}}|{\cell {}{}{}{}{}{}{}{}{}}|{\cell {}{}{}{}{}{}{}{}{}}|{\cell {}{}{}{}{}{}{}{}{}}|.
|{\cell {}{}{}{}{}{}{}{}{}}|{\cell {}{}{}{}{}{}{}{}{}}|{\cell {}{}{}{}{}{}{}{}{}}|{\cell {}{}{}{}{}{}{}{}{}}|{\cell {}{}{}{}{}{}{}{}{}}|{\cell {}{}{}{}{}{}{}{}{}}|{\cell {}{}{}{}{}{}{}{}{}}|{\cell {}{}{}{}{}{}{}{}{}}|{\cell {}{}{}{}{}{}{}{}{}}|.
|{\cell {}{\textbf 2}{}{}{}{}{\textbf 7}{}{}}|{\cell {}{}{}{}{}{}{}{}{}}|{\cell {}{}{}{}{}{}{}{}{}}|{\cell {}{}{}{}{}{}{}{}{}}|{\cell {}{}{}{}{}{}{}{}{}}|{\cell {}{}{}{}{}{}{}{}{}}|{\cell {}{}{}{}{}{}{}{}{}}|{\cell 1{}{}{}{}{}{\sout 7}8{}}|{\cell {}{}{}{}{}{}{}{}{}}|.
|{\cell {}{}{}{}{}{}{}{}{}}|{\cell {}{}{}{}{}{}{}{}{}}|{\cell {}{}{}{}{}{}{}{}{}}|{\cell {}{}{}{}{}{}{}{}{}}|{\cell {}{}{}{}{}{}{}{}{}}|{\cell {}{}{}{}{}{}{}{}{}}|{\cell {}{}{}{}{}{}{}{}{}}|{\cell {}{}{}{}{}{}{}{}{}}|{\cell {}{}{}{}{}{}{}{}{}}|.
\end{sudoku}
\caption{Example in XY-wing}
\label{fig:xywing}
\end{figure}



\subsection{XYZ-wing}
\label{sec:XYZ-wing}

Which is pretty similar to XY-wing, but the only difference is, there is a cell specially has three candidates. So the conditions to be hold are:
\begin{enumerate}
\item three different cells not all in a same unit ,
\item and two consecutive cells shared a unit (row, column, or block),
\item the cell which share two different units with another cell must have all three candidates XYZ in the cell,
\item two cells have different combination XY(for the cell in the same column or row with the cell has XYZ), and YZ(for the cell in the same block with the cell has XYZ) from candidates  X,Y, and Z (Y is the candidate that three cells all have).
\end{enumerate}
If all the conditions above are satisfied, then the cell
\begin{enumerate}
\item which shared a block with two consecutive cells that one has candidate XYZ, and another has candidate YZ,
\item and also in the same column or row with two consecutive cells that one has candidate XYZ, and another has candidate XY
\end{enumerate}
could than have candidate Y removed from the cell.

In the Figure \ref{fig:xyzwing}, assume:
\begin{itemize}
\item $X = n_{9},\ Y = n_{6},\ and\ Z = n_{7}$,
\item $(r_{2}, c_{5})$ has XY (candidate $n_{9}$ and $n_{6}$),
\item $(r_{4}, c_{5})$ has XYZ (candidate $n_{9}$,  $n_{6}$ and $n_{7}$),
\item $(r_{2}, c_{6})$ has YZ (candidate $n_{6}$ and $n_{7}$).
\end{itemize}
There is a cell  $(r_{6}, c_{5})$ shared a block with both $(r_{4}, c_{5})$ and $(r_{2}, c_{6})$, and also shared a column with both $(r_{4}, c_{5})$ and $(r_{2}, c_{5})$ has the same candidate Y = $n_{6}$) with all of these three cells could now have Y removed from it.
\begin{figure}[htbp]
\begin{sudoku}
|{\cell {}{}{}{}{}{}{}{}{}}|{\cell {}{}{}{}{}{}{}{}{}}|{\cell {}{}{}{}{}{}{}{}{}}|{\cell {}{}{}{}{}{}{}{}{}}|{\cell {}{}{}{}{}{}{}{}{}}|{\cell {}{}{}{}{}{}{}{}{}}|{\cell {}{}{}{}{}{}{}{}{}}|{\cell {}{}{}{}{}{}{}{}{}}|{\cell {}{}{}{}{}{}{}{}{}}|.
|{\cell {}{}{}{}{}{}{}{}{}}|{\cell {}{}{}{}{}{}{}{}{}}|{\cell {}{}{}{}{}{}{}{}{}}|{\cell {}{}{}{}{}{}{}{}{}}|{\cell {}{}{}{}{}{\textbf 6}{}{}{\textbf 9}}|{\cell {}{}{}{}{}{}{}{}{}}|{\cell {}{}{}{}{}{}{}{}{}}|{\cell {}{}{}{}{}{}{}{}{}}|.
|{\cell {}{}{}{}{}{}{}{}{}}|{\cell {}{}{}{}{}{}{}{}{}}|{\cell {}{}{}{}{}{}{}{}{}}|{\cell {}{}{}{}{}{}{}{}{}}|{\cell {}{}{}{}{}{}{}{}{}}|{\cell {}{}{}{}{}{}{}{}{}}|{\cell {}{}{}{}{}{}{}{}{}}|{\cell {}{}{}{}{}{}{}{}{}}|{\cell {}{}{}{}{}{}{}{}{}}|.
|{\cell {}{}{}{}{}{}{}{}{}}|{\cell {}{}{}{}{}{}{}{}{}}|{\cell {}{}{}{}{}{}{}{}{}}|1|{\cell {}{}{}{}{}{\textbf 6}{\textbf 7}{}{\textbf 9}}|{\cell {}{}{}{}{}{\textbf 6}{\textbf 7}{}{}}|{\cell {}{}{}{}{}{}{}{}{}}|{\cell {}{}{}{}{}{}{}{}{}}|{\cell {}{}{}{}{}{}{}{}{}}|.
|{\cell {}{}{}{}{}{}{}{}{}}|{\cell {}{}{}{}{}{}{}{}{}}|{\cell {}{}{}{}{}{}{}{}{}}|{\cell {}{}{}{}5{}7{}9}|3|{\cell {}{}{}45{}{}{}{}}|{\cell {}{}{}{}{}{}{}{}{}}|{\cell {}{}{}{}{}{}{}{}{}}|{\cell {}{}{}{}{}{}{}{}{}}|.
|{\cell {}{}{}{}{}{}{}{}{}}|{\cell {}{}{}{}{}{}{}{}{}}|{\cell {}{}{}{}{}{}{}{}{}}|2|{\cell {}{}{}4{}{\sout 6}7{}{}}|8|{\cell {}{}{}{}{}{}{}{}{}}|{\cell {}{}{}{}{}{}{}{}{}}|{\cell {}{}{}{}{}{}{}{}{}}|.
|{\cell {}{}{}{}{}{}{}{}{}}|{\cell {}{}{}{}{}{}{}{}{}}|{\cell {}{}{}{}{}{}{}{}{}}|{\cell {}{}{}{}{}{}{}{}{}}|{\cell {}{}{}{}{}{}{}{}{}}|{\cell {}{}{}{}{}{}{}{}{}}|{\cell {}{}{}{}{}{}{}{}{}}|{\cell {}{}{}{}{}{}{}{}{}}|{\cell {}{}{}{}{}{}{}{}{}}|.
|{\cell {}{}{}{}{}{}{}{}{}}|{\cell {}{}{}{}{}{}{}{}{}}|{\cell {}{}{}{}{}{}{}{}{}}|{\cell {}{}{}{}{}{}{}{}{}}|{\cell {}{}{}{}{}{}{}{}{}}|{\cell {}{}{}{}{}{}{}{}{}}|{\cell {}{}{}{}{}{}{}{}{}}|{\cell {}{}{}{}{}{}{}{}{}}|{\cell {}{}{}{}{}{}{}{}{}}|.
|{\cell {}{}{}{}{}{}{}{}{}}|{\cell {}{}{}{}{}{}{}{}{}}|{\cell {}{}{}{}{}{}{}{}{}}|{\cell {}{}{}{}{}{}{}{}{}}|{\cell {}{}{}{}{}{}{}{}{}}|{\cell {}{}{}{}{}{}{}{}{}}|{\cell {}{}{}{}{}{}{}{}{}}|{\cell {}{}{}{}{}{}{}{}{}}|{\cell {}{}{}{}{}{}{}{}{}}|.
\end{sudoku}
\caption{Example in XYZ-wing}
\label{fig:xyzwing}
\end{figure}


\subsection{WXYZ-wing}
\label{sec:WXYZ-wing}
It is an advanced technique rather than the techniques introduced above. The prerequisites of this techniques are:
\begin{enumerate}
\item four different cells not in a same unit but have a common candidate Z,
\item three cells have candidate XZ, YZ, WZ, and one cell has WXYZ,
\item the cell which has WZ and the one has WXYZ have to be in a block,
\item and the rest of the cells have to be in a same column or row with the one has candidate WXYZ.
\end{enumerate}
As soon as the prerequisites are hold, the cell which has shared a block also a column (or row) with the cell which has candidate WXYZ could have candidate Z excluded.
In the Figure \ref{fig:wxyzwing}, assume that:
\begin{itemize}
\item $W= n_{2}, X = n_{4},\ Y = n_{6},\ and\ Z = n_{5}$,
\item $(r_{1}, c_{8})$ has WXYZ (candidate $n_{2}$, $n_{4}$, $n_{6}$ and $n_{5}$),
\item $(r_{1}, c_{9})$ has WZ (candidate $n_{2}$ and $n_{5}$),
\item $(r_{6}, c_{8})$ has XZ (candidate $n_{4}$ and $n_{5}$),
\item $(r_{7}, c_{8})$ has YZ (candidate $n_{6}$ and $n_{5}$).
\end{itemize}
Four cells all hae a common candidate $n_{5}$, and other conditions are all hold, then we can have this candidate excluded from the cell $(r_{2}, c_{8})$ ($(r_{3}, c_{8})$ does not have candidate $n_{5}$, so no change is needed).

\begin{figure}[htbp]
\begin{sudoku}
|{\cell {}{}{}{}{}{}{}{}{}}|{\cell {}{}{}{}{}{}{}{}{}}|{\cell {}{}{}{}{}{}{}{}{}}|{\cell {}{}{}{}{}{}{}{}{}}|{\cell {}{}{}{}{}{}{}{}{}}|{\cell {}{}{}{}{}{}{}{}{}}|{\cell {}{}{}{}{}{}{}{}{}}|{\cell {}{\textbf 2}{}{\textbf 4}{\textbf 5}{\textbf 6}{}{}{}}|{\cell {}{\textbf 2}{}{}{\textbf 5}{}{}{}{}}|.
|{\cell {}{}{}{}{}{}{}{}{}}|{\cell {}{}{}{}{}{}{}{}{}}|{\cell {}{}{}{}{}{}{}{}{}}|{\cell {}{}{}{}{}{}{}{}{}}|{\cell {}{}{}{}{}{}{}{}{}}|{\cell {}{}{}{}{}{}{}{}{}}|{\cell {}{}{}{}{}{}{}{}{}}|{\cell {}{}3{}{\sout 5}{}7{}{}}|{\cell {}{}{}{}{}{}{}{}{}}|.
|{\cell {}{}{}{}{}{}{}{}{}}|{\cell {}{}{}{}{}{}{}{}{}}|{\cell {}{}{}{}{}{}{}{}{}}|{\cell {}{}{}{}{}{}{}{}{}}|{\cell {}{}{}{}{}{}{}{}{}}|{\cell {}{}{}{}{}{}{}{}{}}|{\cell {}{}{}{}{}{}{}{}{}}|{\cell {}{}34{}{}{}{}{}}|{\cell {}{}{}{}{}{}{}{}{}}|.
|{\cell {}{}{}{}{}{}{}{}{}}|{\cell {}{}{}{}{}{}{}{}{}}|{\cell {}{}{}{}{}{}{}{}{}}|{\cell {}{}{}{}{}{}{}{}{}}|{\cell {}{}{}{}{}{}{}{}{}}|{\cell {}{}{}{}{}{}{}{}{}}|{\cell {}{}{}{}{}{}{}{}{}}|{\cell {}{}{}{}{}{}{}{}{}}|{\cell {}{}{}{}{}{}{}{}{}}|.
|{\cell {}{}{}{}{}{}{}{}{}}|{\cell {}{}{}{}{}{}{}{}{}}|{\cell {}{}{}{}{}{}{}{}{}}|{\cell {}{}{}{}{}{}{}{}{}}|{\cell {}{}{}{}{}{}{}{}{}}|{\cell {}{}{}{}{}{}{}{}{}}|{\cell {}{}{}{}{}{}{}{}{}}|{\cell {}{}{}{}{}{}{}{}{}}|{\cell {}{}{}{}{}{}{}{}{}}|.
|{\cell {}{}{}{}{}{}{}{}{}}|{\cell {}{}{}{}{}{}{}{}{}}|{\cell {}{}{}{}{}{}{}{}{}}|{\cell {}{}{}{}{}{}{}{}{}}|{\cell {}{}{}{}{}{}{}{}{}}|{\cell {}{}{}{}{}{}{}{}{}}|{\cell {}{}{}{}{}{}{}{}{}}|{\cell {}{}{}{\textbf 4}{\textbf 5}{}{}{}{}}|{\cell {}{}{}{}{}{}{}{}{}}|.
|{\cell {}{}{}{}{}{}{}{}{}}|{\cell {}{}{}{}{}{}{}{}{}}|{\cell {}{}{}{}{}{}{}{}{}}|{\cell {}{}{}{}{}{}{}{}{}}|{\cell {}{}{}{}{}{}{}{}{}}|{\cell {}{}{}{}{}{}{}{}{}}|{\cell {}{}{}{}{}{}{}{}{}}|{\cell {}{}{}{}{\textbf 5}{\textbf 6}{}{}{}}|{\cell {}{}{}{}{}{}{}{}{}}|.
|{\cell {}{}{}{}{}{}{}{}{}}|{\cell {}{}{}{}{}{}{}{}{}}|{\cell {}{}{}{}{}{}{}{}{}}|{\cell {}{}{}{}{}{}{}{}{}}|{\cell {}{}{}{}{}{}{}{}{}}|{\cell {}{}{}{}{}{}{}{}{}}|{\cell {}{}{}{}{}{}{}{}{}}|{\cell {}{}{}{}{}{}{}{}{}}|{\cell {}{}{}{}{}{}{}{}{}}|.
|{\cell {}{}{}{}{}{}{}{}{}}|{\cell {}{}{}{}{}{}{}{}{}}|{\cell {}{}{}{}{}{}{}{}{}}|{\cell {}{}{}{}{}{}{}{}{}}|{\cell {}{}{}{}{}{}{}{}{}}|{\cell {}{}{}{}{}{}{}{}{}}|{\cell {}{}{}{}{}{}{}{}{}}|{\cell {}{}{}{}{}{}{}{}{}}|{\cell {}{}{}{}{}{}{}{}{}}|.
\end{sudoku}
\caption{Example in WXYZ-wing}
\label{fig:wxyzwing}
\end{figure}




\subsection{Swordfish}
\label{sec:Swordfish}
Swordfish is an approach extended from X-wing technique, but rarer to be seen. The concept is, a value in one position will force other positions in the same row or same column not to have the same value filled in and vice versa. It is used when there are (a graphic pattern is in Figure \ref{fig:swordfish}):
\begin{enumerate}
\item $X,\ Y,\ Z$ are the row numbers, and $J,\ Q,\ K$ are the column numbers.
\item three different rows, $r_{X}, r_{Y}, r_{Z}$ in three different columns, $c_{J}, c_{Q}, c_{K}$,
\begin{enumerate}
\item $r_{X}$ has $c_{J}$ and $c_{Q}$ among its candidates,
\item $r_{Y}$ has $c_{Q}$ and $c_{K}$ among its candidates,
\item $r_{Z}$has $c_{K}$ and $c_{J}$ among its candidates,
\item and non of $r_{X},\ r_{Y}\, and\ r_{Z}$ has other candidates in $c_{J},\ c_{Q},\ and\ c_{K}.$
\end{enumerate}
\end{enumerate}
If there is one candidate in three rows only appears in particular three columns, then exclude this candidate from other cells in these three columns. In contrast, if one candidate in three columns only appears in particular three rows, then the candidate will be removed from other cells in these three rows.

\begin{figure}[htbp]
\begin{sudoku}
| |{\makebox[0pt]{\raisebox{5ex}{\hspace{0em}\large $c_{J}$}}}| | |{\makebox[0pt]{\raisebox{5ex}{\hspace{0em}\large $c_{Q}$}}}| | |{\makebox[0pt]{\raisebox{5ex}{\hspace{0em}\large $c_{K}$}}}| |.
 |{\makebox[0pt]{\hspace{-2.25em}\large $r_{X}$}}|X| | |X| | |*| |.
| |*| | |*| | |*| |.
| |*| | |*| | |*| |.
|{\makebox[0pt]{\hspace{-2.25em}\large $r_{Y}$}}|*| | |X| | |X| |.
| |*| | |*| | |*| |.
| |*| | |*| | |*| |.
|{\makebox[0pt]{\hspace{-2.25em}\large $r_{Z}$}}|X| | |*| | |X| |.
| |*| | |*| | |*| |.
\end{sudoku}
\caption{Example in Swordfish}
\label{fig:swordfish}
\end{figure}

In the Figure \ref{fig:swordfishreal}, candidate $n_{6}$ appear in $(r_{3}, c_{5})$, $(r_{3}, c_{7})$, $r_{6}, c_{})$, $r_{6}, c_{5})$, $r_{8}, c_{5})$, $r_{8}, c_{7})$ and $r_{8}, c_{8})$, it satisfied the the conditions of Swordfish, and also match the rule ``one candidate in three rows only appears in particular three columns'', thus an exclusion will be made to kick candidate $n_{6}$ out from other cells in the  $r_{3},\ r_{6}\, and\ r_{8}$.

\begin{figure}[htbp]
\begin{sudoku}
 |{\makebox[0pt]{\hspace{-1em}\large r1\raisebox{5ex}[0ex]{\hspace{1.5em}c1}}}|{\makebox[0pt]{\raisebox{5ex}{\hspace{0em}\large c2}}}|{\makebox[0pt]{\raisebox{5ex}{\hspace{0em}\large c3}}}| {\makebox[0pt]{\raisebox{5ex}{\hspace{0em}\large c4}}}|{\makebox[0pt]{\raisebox{5ex}{\hspace{0em}\large c5}}}|{\makebox[0pt]{\raisebox{5ex}{\hspace{0em}\large c6}}}|{\makebox[0pt]{\raisebox{5ex}{\hspace{0em}\large c7}}}|{\makebox[0pt]{\raisebox{5ex}{\hspace{0em}\large c8}}}|{\makebox[0pt]{\raisebox{5ex}{\hspace{0em}\large c9}}}|.
 |{\makebox[0pt]{\hspace{-2.25em}\large r2}}| | | | | | | | |.
 |{\makebox[0pt]{\hspace{-2.25em}\large r3}}4|1|3|7|{\cell {}{}{}{}{}{\textbf 6}{}{}9}|2|{\cell {}{}{}{}{}{\textbf 6}{}{}9}|8|5|.
 |{\makebox[0pt]{\hspace{-2.25em}\large r4}}| | | | | | | | |.
 |{\makebox[0pt]{\hspace{-2.25em}\large r5}}| | | | | | | | |.
 |{\makebox[0pt]{\hspace{-2.25em}\large r6}}3| | |4|2| |{\cell 1{}{}{}5{\textbf 6}{}{}9}|{\cell 1{}{}{}5{\textbf 6}{}{}9}|8|.
 |{\makebox[0pt]{\hspace{-2.25em}\large r7}}| | | | | |{\cell 1{}{}{}{}{\sout 6}{}89}| {\cell 1{}{}{}{}{\sout 6}{}{}9} |.
 |{\makebox[0pt]{\hspace{-2.25em}\large r8}}|3|4| |{\cell {}{}{}{}{}{\textbf 6}{}{}9}|1|{\cell {}2{}{}5{\textbf 6}{}8{}}|{\cell {}{}{}{}5{\textbf 6}{}{}{}}|7|.
 |{\makebox[0pt]{\hspace{-2.25em}\large r9}}| | | | | |{\cell 12{}{}{}{\sout 6}{}8{}}| | |.
\end{sudoku}
\caption{Real example in Swordfish}
\label{fig:swordfishreal}
\end{figure}





\chapter{Environment}
\label{sec:Environment}

\section{Git}
\label{sec:git}


\section{LaTex}
\label{sec:miktex}


\chapter{Conclusion}
\label{sec:Conclusion}
The Sudoku puzzle might be solved by only one technique in some simple case Nevertheless, usually two or more are commonly needed in dealing general problem such like using both naked single and hidden single. The number of techniques used will increase according to the difficulty of the puzzle. The harder problem needs more complex technique to support it.
These special approaches for solving puzzle are simple and in fact popularly used in general problem solving. They are efficient to support player to eliminate the impossibilities in the cells, and leave the rest possibilities to get to the final solution.


\bibliographystyle{plain}

\bibliography{Sudoku}

\end{document}
