% Yu-Ting Lu 24/7/2011 (Swansea)

\documentclass[11pt]{report}

\usepackage{hyperref}
\usepackage{a4}

\usepackage{sudoku}
\usepackage{amsmath}
\usepackage{colortbl}
\usepackage{multirow}

\setlength\sudokusize{8cm}
\newcommand{\cell}[9]{%
  \scriptsize
  \setlength{\tabcolsep}{1pt}
  \renewcommand{\arraystretch}{0.5}
  \hspace{-0.6em}
  \begin{tabular}{ccc}
    #1 & #2 & #3\\
    #4 & #5 & #6\\
    #7 & #8 & #9
  \end{tabular}
}

\newcommand{\set}[1]{\{ #1 \}}


\begin{document}

\title{Sudoku Patterns}
\author{Yu-Ting Lu\\
 Computer Science of Swansea University,\\
 461467@swansea.ac.uk}
\date{July 2011}
\maketitle

\tableofcontents



\chapter{Introduction}
\label{cha:Introduction}


Sudoku is a puzzle which is popularised into the world for ages. And there are amount of people spend times on either hard puzzle generated, or looking for the best techniques to improve the efficiency on solving difficult puzzle.
When the puzzle difficulty becomes harder and harder, then more advanced techniques will be required to be used together or in order to cope the puzzle. Surely, as the difficulties level goes up, the cost of time raise up even thought the suitable technique is given to help. Consequently, in this article, the first chapter is going to know the background of Sudoku and what is Sudoku, and the secondly understand discussed after understanding how these special techniques applied on eliminating impossibilities in puzzle solving. Afterwards, the efficiency of the special technique for solving Sudoku will be the final topic to discuss about in the end.
In this dissertation, the goal to achieve is to experience the time of applying special techniques on solving Sudoku puzzles. In addition, to find out the most efficient way to apply the combination of more than techniques and how many puzzles could it achieve to solve.


\section{What is Sudoku?}
\label{sec:whatissudoku}

Sudoku uses an $N \times N$ matrix, where $N = n \cdot n$, and for standard Sudoku we have $n = 3$, thus $N = 9$. Each entry (``cell'') contains a number in $\{1, \dots, N\}$, or it might be empty, that is, to be filled out. A completed Sudoku puzzle must fulfil the following additional requirements:
\begin{enumerate}
\item In every row and in every column every number occurs only (exactly) once.
\item The same is true for the ``blocks'':
  \begin{enumerate}
  \item The whole matrix is sub-divided into $N$ blocks.
  \item Each block is an $n \times n$ matrix.
  \end{enumerate}
  Now in each block every number has also to occur (exactly) once.
\end{enumerate}
For ``classical Sudoku'' puzzles we have the following additional rules:
\begin{enumerate}
\item A Sudoku puzzle must have at least one solution (no unsolvable puzzles are normally considered).
\item And in fact, a Sudoku puzzle must have a unique solution (no multiple solutions are normally allowed).
\end{enumerate}
An example is given with Figure \ref{sudokuEx}, while for the solution see Figure \ref{fig:solutionsudokuEx}.

\begin{figure}[h]
\begin{sudoku}
 |2|5| | |3| |9| |1|.
 | |1| | | |4| | | |.
 |4| |7| | | |2| |8|.
 | | |5|2| | | | | |.
 | | | | |9|8|1| | |.
 | |4| | | |3| | | |.
 | | | |3|6| | |7|2|.
 | |7| | | | | | |3|.
 |9| |3| | | |6| |4|.
\end{sudoku}
\caption{An example for a Sudoku problem}
\label{sudokuEx}
\end{figure}

\begin{figure}[h]
\begin{sudoku}
  |2|5|8|7|3|6|9|4|1|.
  |6|1|9|8|2|4|3|5|7|.
  |4|3|7|9|1|5|2|6|8|.
  |3|9|5|2|7|1|4|8|6|.
  |7|6|2|4|9|8|1|3|5|.
  |8|4|1|6|5|3|7|2|9|.
  |1|8|4|3|6|9|5|7|2|.
  |5|7|6|1|4|2|8|9|3|.
  |9|2|3|5|8|7|6|1|4|.
\end{sudoku}
\caption{The solution for the Sudoku problem given in Figure \ref{sudokuEx}}
\label{fig:solutionsudokuEx}
\end{figure}





\section{Literature}
\label{sec:introLiterature}

Systematic procedure for solving a puzzle:
\begin{tabbing}
Loop Until \= a solution is found (or until it is proven there can be no solution) \\
\> Do until \= a rule applies  effectively \\
\> \> Take the first rule not yet tried in the list \\
\> \> Do until \= its conditions pattern effectively maps to the cell \\
\> \> \> Try all possible mappings of the conditions pattern \\
\> \> End do\\
\> End do\\
\> Apply rule on selected matching pattern \\
End loop\\
\end{tabbing}

New representations of a puzzle:
\begin{itemize}
\item number n is in rc-cell (r, c),
\item column c is in rn-cell (r, n),
\item row r is in cn-cell (c, n),
\item square s is in bn-cell (b, n) - where (r, c) = [b, s].
\end{itemize}

example: XXX could be re-permuted to get an extended one.

\cite{Berthier2007Sudoku} XXX





\chapter{Basic notions and notations}
\label{cha:basicnotnotat}

Before introducing the techniques for solving Sudoku, there are basic notions that have to be defined.

First we define the concrete ``problems'' we want to tackle, for example Figure \ref{sudokuEx}. Actually, we prefer not to speak of ``problem'', since in computer science this typically means a ``general problem'' (like ``how to solve Sudokus''). We also do not want to talk about ``puzzle'', since this is a rather fuzzy term. But we use the standard terminology of computer science, which speaks about ``general problems'' and their ``instances''.


\section{Sudoku instances and their solutions}
\label{sec:Sudokuinstances}

A \textit{Sudoku instance} is an $N \times N$ matrix (that is, having $N$ rows (horizontal lines) and $N$ columns (vertical lines)). The entries of such a matrix are called \textit{cells} in this context. We use indices as usual for matrices; so for example the cell $(1,7)$ (first row, seventh column) in instance Figure \ref{sudokuEx} carries number $9$.

There are two further conditions on a Sudoku instance:
\begin{enumerate}
\item The cells are either \emph{empty} or one of $1, \dots, 9$.
\item The instance has \emph{exactly one solution} (it is a ``proper'' instance).
\end{enumerate}

It remains to specify what a ``solution'' is. The solution for the instance Figure \ref{sudokuEx} is given in Figure \ref{fig:solutionsudokuEx}. We see that a \textit{solution} is the same type of matrix, with the empty cells being filled in. More precisely:
\begin{enumerate}
\item Exactly the empty cells have been filled in, with numbers in $1,\dots,9$ (nothing else has been changed).
\item In every row and in every column every number from $1,\dots,9$ occurs exactly once (that is, we have a \emph{Latin square}).
\item In every ``block'' (or ``box'') also every number from $1,\dots,9$ occurs exactly once.
\end{enumerate}

The matrix is subdivided into $N$ \textit{blocks}, as indicated in Figures \ref{sudokuEx}, \ref{fig:solutionsudokuEx}. These blocks are numbered from $1$ to $9$, from top to bottom and left to right. Each block is itself a $n \times x$ matrix containing $n \cdot n = N$ cells; see Figure \ref{fig:exampleblock}.

\begin{figure}[h]
\begin{sudoku}
 |{\makebox[0pt]{\hspace{-1em}\large r1\raisebox{5ex}[0ex]{\hspace{1.5em}c1}}}|{\makebox[0pt]{\raisebox{5ex}{\hspace{0em}\large c2}}}|{\makebox[0pt]{\raisebox{5ex}{\hspace{0em}\large c3}}}| {\makebox[0pt]{\raisebox{5ex}{\hspace{0em}\large c4}}}|{\makebox[0pt]{\raisebox{5ex}{\hspace{0em}\large c5}}}|{\makebox[0pt]{\raisebox{5ex}{\hspace{0em}\large c6}}}|{\makebox[0pt]{\raisebox{5ex}{\hspace{0em}\large c7}}}|{\makebox[0pt]{\raisebox{5ex}{\hspace{0em}\large c8}}}|{\makebox[0pt]{\raisebox{5ex}{\hspace{0em}\large c9}}}|.
 |{\makebox[0pt]{\hspace{-2.25em}\large r2}}|{\LARGE b1}| | |{\LARGE b2}| | |{\LARGE b3}| |.
 |{\makebox[0pt]{\hspace{-2.25em}\large r3}}| | | | | | | | |.
 |{\makebox[0pt]{\hspace{-2.25em}\large r4}}| | | | | | | | |.
 |{\makebox[0pt]{\hspace{-2.25em}\large r5}}|{\LARGE b4}| | |{\LARGE b5}| | |{\LARGE b6}| |.
 |{\makebox[0pt]{\hspace{-2.25em}\large r6}}| | | | | | | | |.
 |{\makebox[0pt]{\hspace{-2.25em}\large r7}}| | | | | | | | |.
 |{\makebox[0pt]{\hspace{-2.25em}\large r8}}|{\LARGE b7}| | |{\LARGE b8}| | |{\LARGE b9}| |.
 |{\makebox[0pt]{\hspace{-2.25em}\large r9}}| | | | | | | | |.
\end{sudoku}
\caption{The 9 blocks (``boxes'') in a Sudoku instance}
\label{fig:exampleblock}
\end{figure}



\section{Annotated instances}
\label{sec:Annotatedinstances}

We now know what the input of our intended solution procedure is, namely a Sudoku instances, and what the output shall be, namely a solution, which is defined as all empty cells filled in according to the rules. This is based on $N \times N$ matrices, where the entries are either numbers from $1,\dots,9$, or an indicator ``empty'' (which could be handled by carrying the value $0$).

We are considering in this report only techniques for arriving at a solution which do not backtrack, but which proceed in applications of ``rules'', where each such application makes clear progress towards the solution. These rules are based on the common techniques of ``pencil-marking'' the empty cells, writing down in each cell the candidates according to ``current knowledge'': more and more candidates are excluded, based on sound reasoning, until only one candidate is left, and the cell is solved. So we need to move from Sudoku instances (as defined in Section \ref{sec:Sudokuinstances}) to \textbf{annotated Sudoku instances}:
\begin{itemize}
\item Now each cell carries a partition of $\set{1,\dots,n}$ into two subsets, the \emph{possible candidates} and the \emph{excluded candidates}.
\item This is denoted by a pair $(P,E)$ with $P, E \subseteq \set{1,\dots,N}$, where $P \cap E = \emptyset$ and $P \cup E = \set{1,\dots,N}$ (``P'' for ``possible'' and ``E'' for ``excluded'').
\item For a proper annotation the (unique) solution must always be an element of the candidates (i.e., excluding candidates must always be right).
\end{itemize}

In Figure \ref{fig:rcWhole} we have annotated the example from Figure \ref{sudokuEx}. For example, the annotation for cell $(3,4)$ is $(\set{1, 6, 8, 9}, \set{2, 3, 4, 5, 7})$, which means that candidates $2,3,4,5,7$ are already excluded from possible candidates, and only $4$ candidates are left. We see that in the figures we only show the candidates. To have another example, the annotation of cell $(1,6)$ is $(\set{8}, \set{1, 2, 3, 4, 5, 6, 7, 9})$, and thus here only one candidate is left --- which must be the solution for this cell. For the purpose of visualising the solution process, once only one candidate is left, we just write out the solution as one big number for that cell; see Figure \ref{fig:rcWholeupdated} for the updated visualisation (with two cells simplified).

However for the systematic treatment we only use the annotated instances:
\begin{enumerate}
\item An instance is transformed into an annotated instance by
  \begin{itemize}
  \item translating cells with filled-in value $v$ as $(\set{v}, \set{1,\dots,N} \setminus \set{v})$,
  \item and translating empty cells as $(\set{1,\dots,N},\emptyset)$.
  \end{itemize}
\item The solution is reached once every cell is of the form $(\set{v}, \set{1,\dots,N} \setminus \set{v})$, that is, exactly one candidate is left for each cell.
\end{enumerate}

For an annotated instance $A$ we write $A_{i,j}$ for the content of cell $(i,j)$, just using matrix notation; for $A$ as in Figure \ref{fig:rcWhole} for example we have $A_{3,4} = (\set{1, 6, 8, 9}, \set{2, 3, 4, 5, 7})$.


\begin{figure}[h]
\begin{sudoku}
|1|{\cell {}{}3{}56{}{}{}}|{\cell {}{}3{}{}678{}}|2|{\cell {}{}{}{}{}6{}89}|{\cell {}{}{}{}{}{}{}8{}}|{\cell {}{}{}{}5{}{}89}|{\cell {}{}3{}{}{}{}89}|4|.
|{\cell {}2{}{}{}{}{}8{}}|9|{\cell {}{}{}{}{}{}{}8{}}|5|{\cell {}{}{}4{}{}{}8{}}|3|6| 7|{\cell 12{}{}{}{}{}8{}}|.
|{\cell {}23{}56{}8{}}|4|{\cell {}{}3{}{}6{}8{}}|{\cell 1{}{}{}{}6{}89}|7|{\cell 1{}{}{}{}{}{}8{}}|{\cell 12{}{}5{}{}89}|{\cell {}{}3{}{}{}{}89}|{\cell 123{}{}{}{}8{}}|.
|{\cell {}{}3{}{}6{}8{}}|1|{\cell {}{}34{}6{}8{}}|{\cell {}{}{}4{}678{}}|{\cell {}234{}6{}8{}}|{\cell {}2{}4{}{}78{}}|{\cell {}{}{}4{}{}78{}}|5|9|.
|{\cell {}{}{}{}56{}8{}}|{\cell {}{}{}{}56{}{}{}}|2|{\cell 1{}{}4{}6789}|{\cell {}{}{}456{}89}|{\cell 1{}{}45{}78{}}|3|{\cell {}{}{}4{}{}{}8{}}|{\cell 1{}{}{}{}678{}}|.
|9|7|{\cell {}{}34{}6{}8{}}|{\cell 1{}{}4{}6{}8{}}|{\cell {}{}3456{}8{}}|{\cell 1{}{}45{}{}8{}}|{\cell 1{}{}4{}{}{}8{}}|2|{\cell 1{}{}{}{}6{}8{}}|.
|{\cell {}23{}{}{}7{}{}}|{\cell {}23{}{}{}{}{}{}}|{\cell {}{}3{}{}{}7{}9}|{\cell {}{}{}4{}{}78{}}|1|{\cell {}2{}45{}78{}}|{\cell {}2{}4{}{}789}|6|{\cell {}23{}{}{}78{}}|.
|{\cell {}2{}{}{}67{}{}}|8|5|3| {\cell {}2{}4{}{}{}{}{}}|9| {\cell {}2{}4{}{}7{}{}}|1| {\cell {}2{}{}{}{}7{}{}}|.
|4| {\cell {}23{}{}{}{}{}{}}|{\cell 1{}3{}{}{}7{}9}|{\cell {}{}{}{}{}{}78{}}|{\cell {}2{}{}{}{}{}8{}}|6|{\cell {}2{}{}{}{}789}|{\cell {}{}3{}{}{}{}89}| 5|.
\end{sudoku}
\caption{Annotated Sudoku instance (??? this must be the same instance as before!!!)}
\label{fig:rcWhole}
\end{figure}

\begin{figure}[h]
\begin{sudoku}
|1|{\cell {}{}3{}56{}{}{}}|{\cell {}{}3{}{}678{}}|2|{\cell {}{}{}{}{}6{}89}|8|{\cell {}{}{}{}5{}{}89}|{\cell {}{}3{}{}{}{}89}|4|.
|{\cell {}2{}{}{}{}{}8{}}|9|8|5|{\cell {}{}{}4{}{}{}8{}}|3|6| 7|{\cell 12{}{}{}{}{}8{}}|.
|{\cell {}23{}56{}8{}}|4|{\cell {}{}3{}{}6{}8{}}|{\cell 1{}{}{}{}6{}89}|7|{\cell 1{}{}{}{}{}{}8{}}|{\cell 12{}{}5{}{}89}|{\cell {}{}3{}{}{}{}89}|{\cell 123{}{}{}{}8{}}|.
|{\cell {}{}3{}{}6{}8{}}|1|{\cell {}{}34{}6{}8{}}|{\cell {}{}{}4{}678{}}|{\cell {}234{}6{}8{}}|{\cell {}2{}4{}{}78{}}|{\cell {}{}{}4{}{}78{}}|5|9|.
|{\cell {}{}{}{}56{}8{}}|{\cell {}{}{}{}56{}{}{}}|2|{\cell 1{}{}4{}6789}|{\cell {}{}{}456{}89}|{\cell 1{}{}45{}78{}}|3|{\cell {}{}{}4{}{}{}8{}}|{\cell 1{}{}{}{}678{}}|.
|9|7|{\cell {}{}34{}6{}8{}}|{\cell 1{}{}4{}6{}8{}}|{\cell {}{}3456{}8{}}|{\cell 1{}{}45{}{}8{}}|{\cell 1{}{}4{}{}{}8{}}|2|{\cell 1{}{}{}{}6{}8{}}|.
|{\cell {}23{}{}{}7{}{}}|{\cell {}23{}{}{}{}{}{}}|{\cell {}{}3{}{}{}7{}9}|{\cell {}{}{}4{}{}78{}}|1|{\cell {}2{}45{}78{}}|{\cell {}2{}4{}{}789}|6|{\cell {}23{}{}{}78{}}|.
|{\cell {}2{}{}{}67{}{}}|8|5|3| {\cell {}2{}4{}{}{}{}{}}|9| {\cell {}2{}4{}{}7{}{}}|1| {\cell {}2{}{}{}{}7{}{}}|.
|4| {\cell {}23{}{}{}{}{}{}}|{\cell 1{}3{}{}{}7{}9}|{\cell {}{}{}{}{}{}78{}}|{\cell {}2{}{}{}{}{}8{}}|6|{\cell {}2{}{}{}{}789}|{\cell {}{}3{}{}{}{}89}| 5|.
\end{sudoku}
\caption{The annotated instance of Figure \ref{fig:rcWhole}, with completed visualisation (!!! must be updated after changing to the unique example !!!)}
\label{fig:rcWholeupdated}
\end{figure}



\section{On the nature of ``Sudoku rules''}
\label{sec:naturesudokurules}

Having the notion of annotated instances at hand, we can now say what a ``rule'' is:
\begin{itemize}
\item A \textbf{Sudoku rule} is a partial map, mapping certain annotated instances $A$ to annotated instances $A'$ for those $A$ in the \emph{domain} of the rule, such that:
  \begin{itemize}
  \item For some cells certain candidates have been excluded.
  \item No other type of change takes places (that is, we only exclude candidates, while excluded candidates never get included again).
  \item These changes are corrected, that is, the (unique) solution is still possible.
  \item At least one cell has been changed (something happened).
  \end{itemize}
\end{itemize}



\chapter{Exploiting symmetries by using different views}
\label{cha:exploitingsymm}


\section{The general idea}
\label{sec:diffviewsidea}

In an annotated instance we can find $4$ ``dimensions'':
\begin{enumerate}
\item rows (``r'')
\item columns (``c'')
\item numbers (the content of the cells; ``n'')
\item blocks (``b'').
\end{enumerate}
We are now introducing four ``views'' on an annotated instance:
\begin{enumerate}
\item ``rc-view''
\item ``rn-view''
\item ``cn-view''
\item ``bn-view''.
\end{enumerate}
The rc-view (``row-column view'') is our standard point of view: the rows and columns of the matrix are the rows and the columns of the (annotated) instance, and the entries refer to the possible (or impossible) numbers for that cell. The other three views allow us to derive new rules from old rules, by just changing the perspective:
\begin{center}
  The starting observation is that rows, columns, numbers and blocks are all numbers $1,\dots,N$ --- so perhaps they can change role?!
\end{center}




\section{Different  perspectives}
\label{sec:Differentperspectives}

XXX needs to be rewritten XXX

XXX Also here the one running example from the previous chapter should be used XXX

XXX the equations below are just not meaningful, and shall be written for example as ``$(r_i,n_j) \mapsto (C,E)$'', that is, we use \emph{mappings} XXX

XXX ``$n$'' is not to be used as index, since it stands for ``number'' XXX

To re-permute the rc-view instance into rn-view, element $c$ is exchanged with element $n$, thus, the equation will be rewritten as following $(r,\ c) = n \Rightarrow (r,\ n) = c$. Likewise, the equation of cn-view will be $(c,\ n) = r$. In the rn-view, the value in the cell is to display all possibilities of the specific number could be appeared in those columns. Table \ref{tab:rccolumn} is the example for a row in rc-view, and the Table \ref{tab:rncolumn} is the row after it has been re-permuted. In the Table \ref{tab:rccolumn}, $(r_{n}, c_{n})$ expresses that one of one or more possible candidates could be the value for this cell. In the Table \ref{tab:rncolumn}, $(r_{n}, n_{n})$ show one candidate could have it appeared in one or more column $c_{n}$. For instance, $(r_{n}, n_{1})$ shows that $n_{1}$ could only be in $c_{3}$. In addition, $(r_{n}, n_{2})$ shows that $n_{2}$ are the candidate which is possible be in $c_{4}$ or $c_{5}$. 

\begin{table}
\setlength{\tabcolsep}{3pt}
\renewcommand{\arraystretch}{2}
\begin{center}
\begin{tabular}{ c| c| c| c| c| c| c| c| c| c| }
\multicolumn{1}{c}{} &  \multicolumn{1}{c}{$n_{1}$}  &  \multicolumn{1}{c}{ $n_{2}$} &  \multicolumn{1}{c}{ $n_{3}$} &  \multicolumn{1}{c}{$n_{4}$}  &  \multicolumn{1}{c}{$n_{5}$}  & \multicolumn{1}{c}{$n_{6}$}  &  \multicolumn{1}{c}{ $n_{7}$} &  \multicolumn{1}{c}{$n_{8}$}  &  \multicolumn{1}{c}{$n_{9}$}  \\ \cline{2-10}
$r_{n}$ & 3 & {\cell {}{}{}45{}{}{}{}} & 6 & 8  & {\cell {}{}{}45{}7{}{}} & {\cell 12{}4{}{}{}{}9} & {\cell {}{}{}{}5{}7{}{}} & {\cell {}2{}45{}{}{}{}} & {\cell 1{}{}{}5{}7{}9} \\ \cline{2-10}
\end{tabular}
\end{center}
\caption{One row example after re-permutation from rc-view to rn-view}
\label{tab:rncolumn}
\end{table}

Figure \ref{fig:rnWhole} is the complete rn-view example after re-permutation from rc-view. Figure \ref{fig:cnview} is an extended view made of column and number (called cn-view).
\begin{figure}
\begin{sudoku}
|1|4|{\cell {}23{}{}{}{}8{}}|9|{\cell {}2{}{}{}{}7{}{}}|{\cell {}23{}5{}{}{}{}}|{\cell {}{}3{}{}{}{}{}{}}|{\cell {}{}3{}5678{}}|{\cell {}{}{}{}5{}78{}}|.
|{\cell {}{}{}{}{}{}{}{}9}|{\cell 1{}{}{}{}{}{}{}9}|6|{\cell {}{}{}{}5{}{}{}{}}|4|7| 8|{\cell 1{}3{}5{}{}{}9}|2|.
|{\cell {}{}{}4{}67{}9}|{\cell 1{}{}{}{}{}7{}9}|{\cell 1{}3{}{}{}{}89}|2|{\cell 1{}{}{}{}{}7{}{}}|{\cell 1{}34{}{}{}{}{}}|5|{\cell 1{}34{}6789}|{\cell {}{}{}4{}{}78{}}|.
|2|{\cell {}{}{}{}56{}{}{}}|{\cell 1{}3{}5{}{}{}{}}|{\cell {}{}34567{}{}}|8|{\cell 1{}345{}{}{}{}}|{\cell {}{}{}4{}67{}{}}|{\cell 1{}34567{}{}}|9|.
|{\cell {}{}{}4{}6{}{}9}|3|7|{\cell {}{}{}456{}8{}}|{\cell 12{}{}56{}{}{}}|{\cell 12{}45{}{}{}9}|{\cell {}{}{}4{}6{}{}9}|{\cell 1{}{}456{}89}|{\cell {}{}{}45{}{}{}{}}|.
|{\cell {}{}{}4{}67{}9}|8|{\cell {}{}3{}5{}{}{}{}}|{\cell {}{}34567{}{}}|{\cell {}{}{}{}56{}{}{}}|{\cell {}{}345{}{}{}9}|2|{\cell {}{}34567{}9}|1|.
|5|{\cell 12{}{}{}67{}9}|{\cell 123{}{}{}{}{}9}|{\cell {}{}{}4{}67{}{}}|{\cell {}{}{}{}{}6{}{}{}}|8|{\cell 1{}34{}67{}9}|{\cell {}{}{}4{}67{}9}|{\cell {}{}3{}{}{}7{}{}}|.
|8|{\cell 1{}{}{}5{}7{}9}|4| {\cell {}{}{}{}5{}7{}{}}|3| {\cell 1{}{}{}{}{}{}{}{}}|{\cell 1{}{}{}{}{}7{}9}|2|6|.
|{\cell {}{}3{}{}{}{}{}{}}| {\cell {}2{}{}5{}7{}{}}|{\cell {}23{}{}{}{}8{}}|1|9|6|{\cell {}{}34{}{}7{}{}}| {\cell {}{}{}45{}78{}}|{\cell {}{}3{}{}{}78{}}|.
\end{sudoku}
\caption{Example puzzle in rn-view.}
\label{fig:rnWhole}
\end{figure}


\begin{figure}
\begin{sudoku}
|1|{\cell 12{}{}{}{}78{}}|{\cell {}{}34{}{}7{}{}}|9|{\cell {}{}3{}5{}{}{}{}}|{\cell {}{}345{}{}8{}}|{\cell {}{}{}{}{}{}78{}}|{\cell {}{}345{}{}{}{}}|6|.
|4|{\cell {}{}{}{}{}{}7{}9}|{\cell 1{}{}{}{}{}7{}9}|3|{\cell 1{}{}{}5{}{}{}{}}|{\cell 1{}{}{}5{}{}{}{}}|6|8|2|.
|{\cell {}{}{}{}{}{}{}{}9}|5|{\cell 1{}34{}67{}9}|{\cell {}{}{}4{}6{}{}{}}|8|{\cell 1{}34{}6{}{}{}}|{\cell 1{}{}{}{}{}7{}9}|{\cell 1234{}6{}{}{}}|{\cell {}{}{}{}{}{}7{}9}|.
|{\cell {}{}3{}{}6{}{}{}}|1|8|{\cell {}{}{}4567{}{}}|2|{\cell {}{}3456{}{}{}}|{\cell {}{}{}45{}7{}9}|{\cell {}{}34567{}9}|{\cell {}{}3{}5{}{}{}{}}|.
|7|{\cell {}{}{}4{}{}{}89}|{\cell {}{}3{}5{}{}{}{}}|{\cell {}2{}456{}8{}}|{\cell {}{}{}{}56{}{}{}}|{\cell 1{}{}456{}{}{}}|3|{\cell 12{}456{}{}9}|{\cell 1{}{}{}5{}{}{}{}}|.
|{\cell {}{}3{}56{}{}{}}|{\cell {}{}{}4{}{}7{}{}}|2|{\cell {}{}{}4567{}{}}|{\cell {}{}{}{}567{}{}}|9|{\cell {}{}{}45{}7{}{}}|{\cell 1{}34567{}{}}|8|.
|{\cell {}{}3{}{}6{}{}{}}|{\cell {}{}3{}{}{}789}|5|{\cell {}{}{}4{}678{}}|{\cell 1{}3{}{}{}{}{}{}}|2|{\cell {}{}{}4{}{}789}|{\cell 1{}34{}67{}9}|{\cell 1{}3{}{}{}7{}9}|.
|8|6|{\cell 1{}3{}{}{}{}{}9}|{\cell {}{}{}{}5{}{}{}{}}|4|7|2|{\cell 1{}3{}5{}{}{}9}|{\cell 1{}3{}{}{}{}{}9}|.
|{\cell {}23{}56{}{}{}}|{\cell {}23{}{}{}78{}}|{\cell {}{}3{}{}{}7{}{}}|1|9|{\cell {}{}{}{}56{}{}{}}|{\cell {}{}{}{}5{}{}89}|{\cell {}23{}567{}{}}|4|.
\end{sudoku}
\caption{Example puzzle in cn-view.}
\label{fig:cnview}
\end{figure}

In the block-number view, the coordinate for all the cells are marked in the different way. All the $N \times N$ cells are separated into $N$ blocks ($b_{1} \dots b_{n}$) which arranged in order from the block at top-left to the one at bottom-right ($r_{n}$ ,$c_{n}$) as what the table \ref{tab:blockOrder} shown. The cells in the blocks will be called as ``squares'' to be distinguished from general cells. Cells are the definition for general, the squares are the definition which specially indicated to the specific cell of a block. Every block in the puzzle consists $N = n \times n$ squares. For instance, assume $N$ is $9$ and $n$ is $3$ , the block $b_{1}$ is composed of ($r_{1}$ ,$c_{1}$), ($r_{1}$ ,$c_{2}$), ($r_{1}$ ,$c_{3}$), ($r_{2}$ ,$c_{1}$), ($r_{2}$ ,$c_{2}$), ($r_{2}$ ,$c_{3}$), ($r_{3}$ ,$c_{1}$), ($r_{3}$ ,$c_{2}$), and ($r_{3}$ ,$c_{3}$), for block $b_{9}$ at the bottom-right which is composed of ($r_{n-2}$ ,$c_{n-2}$), ($r_{n-2}$ ,$c_{n-1}$), ($r_{n-2}$ ,$c_{n}$), ($r_{n-1}$ ,$c_{n-2}$), ($r_{n-1}$ ,$c_{n-1}$), ($r_{n-1}$ ,$c_{n}$), ($r_{n}$ ,$c_{n-2}$), ($r_{n}$ ,$c_{n-1}$), and ($r_{n}$ ,$c_{n}$). In the Table \ref{tab:Square}, each block is consisted of 9 squares which represented in order from left to right, from top to bottom, from $s_{1}\dots s_{9}$, use block and square as coordinates, we can have another representation: [b, s]. And every [b, s] coordinate is indicated to one (r, c). For example we have a $(r_{5},\ c_{9})$ which is a (r, c) coordinate here, to find out the corresponding [b, s] coordinate, firstly look at r-coordinate, $r_{5}$ is at the middle part in vertical, and the c-coordinate $c_{9}$ is at the right part in horizontal. Hence, $(r_{5},\ c_{9})$ is in block $b_{6}$. In block $b_{6}$,  $r_{5}$ corresponds to square $s_{4},\  s_{5},\  s_{6}$ and  $c_{9}$ corresponds to square  $s_{3},\  s_{6},\  s_{9}$. Thus, $s_{6}$ is the one both $r_{5}$ and $c_{9}$ corresponds to. Hence, [$b_{6},\  s_{6}$ = $(r_{5},\ c_{9})$.

\begin{table}
\begin{center}
\begin{tabular}{ c| c c c| c c c| c c c|}
\multicolumn{1}{c}{} & $c_{1}$ &  $\dots$ & \multicolumn{1}{c}{$\dots$} & $\dots$ & $\dots$ & \multicolumn{1}{c}{$\dots$} & $\dots$ & $\dots$ & \multicolumn{1}{c}{$b_{n}$}\\ \cline{2-10}
$r_{1}$ & $s_{1}$ &  $s_{2}$ & $s_{3}$ & $s_{1}$ & $s_{2}$ & $s_{3}$ & $s_{1}$ & $s_{2}$ & $s_{3}$\\
$\vdots$ & $s_{4}$ & $s_{5}$ & $s_{6}$ & $s_{4}$ & $s_{5}$ & $s_{6}$ &  $s_{4}$ & $s_{5}$ & $s_{6}$\\
$\vdots$ & $s_{7}$ & $s_{8}$ & $s_{9}$ & $s_{7}$ & $s_{8}$ & $s_{9}$ & $s_{7}$ & $s_{8}$ & $s_{9}$\\ \cline{2-10}
$\vdots$ & $s_{1}$ &  $s_{2}$ & $s_{3}$ & $s_{1}$ & $s_{2}$ & $s_{3}$ & $s_{1}$ & $s_{2}$ & $s_{3}$\\
$\vdots$ & $s_{4}$ & $s_{5}$ & $s_{6}$ & $s_{4}$ & $s_{5}$ & $s_{6}$ &  $s_{4}$ & $s_{5}$ & $s_{6}$\\
$\vdots$ & $s_{7}$ & $s_{8}$ & $s_{9}$ & $s_{7}$ & $s_{8}$ & $s_{9}$ & $s_{7}$ & $s_{8}$ & $s_{9}$\\ \cline{2-10}
$\vdots$ & $s_{1}$ &  $s_{2}$ & $s_{3}$ & $s_{1}$ & $s_{2}$ & $s_{3}$ & $s_{1}$ & $s_{2}$ & $s_{3}$\\
$\vdots$ & $s_{4}$ & $s_{5}$ & $s_{6}$ & $s_{4}$ & $s_{5}$ & $s_{6}$ &  $s_{4}$ & $s_{5}$ & $s_{6}$\\
$r_{n}$ & $s_{7}$ & $s_{8}$ & $s_{9}$ & $s_{7}$ & $s_{8}$ & $s_{9}$ & $s_{7}$ & $s_{8}$ & $s_{9}$\\ \cline{2-10}
\end{tabular}
\end{center}
\caption{Squares in the blocks.}
\label{tab:Square}
\end{table}

\begin{figure}
\begin{sudoku}
|1|{\cell {}{}{}4{}{}7{}{}}|{\cell {}23{}{}{}7{}9}|8|{\cell {}2{}{}{}{}7{}{}}|{\cell 12{}{}{}{}7{}9}|{\cell {}{}3{}{}{}{}{}{}}|{\cell {}{}34{}6{}89}|5|.
|{\cell {}{}{}{}{}{}7{}9}|1|6|{\cell {}{}{}{}5{}{}{}{}}|4|{\cell 1{}{}{}{}{}7{}{}}|8|{\cell {}23{}5{}{}89}|{\cell {}2{}{}{}{}7{}{}}|.
|{\cell {}{}{}{}{}67{}9}|{\cell {}{}{}{}{}67{}9}|{\cell {}2{}{}{}{}{}89}|3|{\cell 1{}{}{}{}{}7{}{}}|4|5|{\cell 12{}{}{}6789}|{\cell 12{}{}{}{}78{}}|.
|2|6|{\cell 1{}3{}{}{}{}{}9}|{\cell {}{}3{}{}{}{}{}9}|{\cell {}{}{}45{}{}{}{}}|{\cell 1{}345{}{}{}9}|8|{\cell 1{}34{}{}{}{}9}|7|.
|{\cell {}{}{}4{}67{}9}|{\cell {}23{}{}{}{}{}{}}|{\cell {}2{}{}{}{}{}8{}}|{\cell 123456789}|{\cell {}{}{}{}56{}89}|{\cell 12{}45{}78{}}|{\cell 1{}34{}6{}{}{}}|{\cell 123456789}|{\cell {}{}{}45{}{}{}{}}|.
|{\cell {}{}{}{}{}67{}9}|8|4|{\cell 1{}{}{}5{}7{}{}}|2|{\cell {}{}{}{}{}6{}{}9}|{\cell 1{}{}{}{}6{}{}{}}|{\cell 1{}{}{}567{}9}|3|.
|{\cell {}{}{}{}{}{}{}{}9}|{\cell 12{}4{}{}{}8{}}|{\cell 123{}{}{}{}89}|7|6|{\cell {}{}{}4{}{}{}{}{}}|{\cell 1{}34{}{}{}{}9}|5|{\cell {}{}3{}{}{}{}{}9}|.
|2|{\cell {}{}3{}5{}{}8{}}|4|{\cell 1{}3{}5{}{}{}{}}|{\cell {}{}3{}{}{}{}{}{}}|9|{\cell 1{}3{}{}{}7{}{}}|{\cell 1{}3{}{}{}78{}}|6|.
|5|{\cell 1{}34{}67{}{}}|{\cell {}{}3{}{}{}{}8{}}|{\cell 1{}{}4{}{}{}{}{}}|9|2|{\cell 1{}34{}67{}{}}|{\cell 1{}3{}{}{}78{}}|{\cell 1{}{}{}{}{}78{}}.
\end{sudoku}
\caption{Example puzzle in bn-view.}
\label{fig:bnview}
\end{figure}

To be prepared for starting Sudoku techniques understanding, there are few things that needed to be known.
\begin{enumerate}
\item r or row will be the abbreviation for row,
\item c or col will be the abbreviation for column,
\item b or blk will be the abbreviation for block.
\end{enumerate}
when the candidate must be eliminated after rules applied, then this specific candidate must remove simultaneously from three representations.
\begin{enumerate}
\item eliminate n from rc-cell (r, c) in the standard rc-representation,
\item eliminate c from rn-cell (r, n) in the rn-representation,
\item eliminate r from cn-cell (c, n) in the cn-representation;
\end{enumerate}
As same, when a certain value has to insert to cell, update has to be done simultaneously in all three representations.
\begin{enumerate}
\item insert n as the value of rc-cell (r, c) in the standard rc-representation,
\item insert c as the value of rn-cell (r, n) in the rn-representation,
\item insert r as the value of cn-cell (c, n) in the cn-representation.
\end{enumerate}
\begin{enumerate}
\item $\forall r\ means\ "for\ all\ row\ r",$
\item $\forall c\ means\ "for\ all\ column\ c",$
\item $\exists n\ always\ means\ "there\ exists\ a\ number\ n".$
\end{enumerate}

XXX how to speak about a ``Sudoku problem''; as in the book, with additional explanations and examples XXX
Notations relating to a specific instance: XXX explaining sub-matrices XXX








\chapter{Techniques for solving Sudoku}
\label{sec:Techniques}

Sudoku problems have classified into different level according to its difficulties. To solve the puzzles, there are loads of techniques could be used to support user to solve Sudoku puzzles in different level of difficulties.
These techniques are usually divided into two main parts, one is called “Direct Elimination Technique”, and the other is named “Candidates Elimination Techniques".
Direct Elimination Technique is an approach could be easily applied without any pencilmarks written on. It eliminates the impossibilities through analysing the existing numbers given in the question step by step. It is the most common and easy way to cope the puzzles, but it will work insufficiently if the harder question has been encountered. 
Consequently, Candidates Elimination Techniques are provided to solve harder question. To apply this skill to deal Sudoku question, pencilmarks are mostly needed to be written in all the cells at the first stage. It is a step to write all the possibilities in all the awaiting cells. Pencilmarks will help with skills applied to eliminate all the impossibilities till the last unique digit is sure for players to fill the right digit in.
Generally, most of the Sudoku problem which is classified to easy level could be solved by applying naked single and hidden single skill if the Direct Elimination Technique is inapplicable. According to the Degree of difficulty, more and more Candidates Elimination Techniques will need to be applied jointly to figure out the solution. 
Naked single and hidden single are the basic methods of Candidates Elimination Techniques mostly used to start the game. When the harder puzzle comes, the techniques: naked pair, hidden pair, naked triplet, hidden triplet, naked quad or even hidden quad techniques will involves in solving the puzzle. Usually the general puzzles could be deal by the techniques introduced lately. X-wing, XY-wing, XYZ-wing, WXYZ-wing, and Swordfish are rare used only if the infrequent problem of high level difficulty appears. 


\section{Naked Single}
\label{sec:Naked Single}

Naked single is the easiest method that could be used to play sudoku puzzle. The situation to apply naked single is, firstly, find out all the possible candidates and pencilmark the possible candidates in all the possible cells. After the work is done for all the cells. Look into the puzzle to find if there is any cell has one possibility only. If so, then, yes it is the situation what we called ``Naked Single'', and candidate found is the only solution to fill into the specific cell of the puzzle.

The formula of Naked Single is \[ \forall r  \forall c\  \{ \exists !n \  candidate(n, r, c) => value(n, r, c)\}\]

\begin{figure}
\begin{sudoku}
|1|{\cell {}{}3{}56{}{}{}}|{\cell {}{}3{}{}678{}}|2|{\cell {}{}{}{}{}6{}89}|{\cell {}{}{}{}{}{}{}8{}}|{\cell {}{}{}{}5{}{}89}|{\cell {}{}3{}{}{}{}89}|4|.
|{\cell {}2{}{}{}{}{}8{}}|9|{\cell {}{}{}{}{}{}{}8{}}|5|{\cell {}{}{}4{}{}{}8{}}|3|6| 7|{\cell 12{}{}{}{}{}8{}}|.
|{\cell {}23{}56{}8{}}|4|{\cell {}{}3{}{}6{}8{}}|{\cell 1{}{}{}{}6{}89}|7|{\cell 1{}{}{}{}{}{}8{}}|{\cell 12{}{}5{}{}89}|{\cell {}{}3{}{}{}{}89}|{\cell 123{}{}{}{}8{}}|.
|{\cell {}{}3{}{}6{}8{}}|1|{\cell {}{}34{}6{}8{}}|{\cell {}{}{}4{}678{}}|{\cell {}234{}6{}8{}}|{\cell {}2{}4{}{}78{}}|{\cell {}{}{}4{}{}78{}}|5|9|.
|{\cell {}{}{}{}56{}8{}}|{\cell {}{}{}{}56{}{}{}}|2|{\cell 1{}{}4{}6789}|{\cell {}{}{}456{}89}|{\cell 1{}{}45{}78{}}|3|{\cell {}{}{}4{}{}{}8{}}|{\cell 1{}{}{}{}678{}}|.
|9|7|{\cell {}{}34{}6{}8{}}|{\cell 1{}{}4{}6{}8{}}|{\cell {}{}3456{}8{}}|{\cell 1{}{}45{}{}8{}}|{\cell 1{}{}4{}{}{}8{}}|2|{\cell 1{}{}{}{}6{}8{}}|.
|{\cell {}23{}{}{}7{}{}}|{\cell {}23{}{}{}{}{}{}}|{\cell {}{}3{}{}{}7{}9}|{\cell {}{}{}4{}{}78{}}|1|{\cell {}2{}45{}78{}}|{\cell {}2{}4{}{}789}|6|{\cell {}23{}{}{}78{}}|.
|{\cell {}2{}{}{}67{}{}}|8|5|3| {\cell {}2{}4{}{}{}{}{}}|9| {\cell {}2{}4{}{}7{}{}}|1| {\cell {}2{}{}{}{}7{}{}}|.
|4| {\cell {}23{}{}{}{}{}{}}|{\cell 1{}3{}{}{}7{}9}|{\cell {}{}{}{}{}{}78{}}|{\cell {}2{}{}{}{}{}8{}}|6|{\cell {}2{}{}{}{}789}|{\cell {}{}3{}{}{}{}89}| 5|.
\end{sudoku}
\caption{Example in Naked Single}
\label{fig:nakedsingle}
\end{figure}

In the Figure \ref{fig:nakedsingle}, candidate 8 in $(r_{1},\ c_{6})$ and $(r_{2},\ c_{3})$ is obviously the only one possibility to fill into that two cells, thus this puzzle could be simply started completing these two cells.

XXX if a cell-matrix contains only one candidate, then the formely empty cell can be filled with this candidate XXX

??? what to do with row/column/block constraints? are the applied automatically ???


\section{Naked Pair}
\label{sec:Naked Pair}

Naked Pair is defined in the situation that two cells have two same candidate digits and no others in both two specific cells which exactly in the same unit (same row, same column, or same block). By applying this technique, elimination might be possible to be made if any cell shared the same unit as that two specific cells and to have the same candidates excluded in the cell.

XXX we are in a ``situation'' where two cells in a row resp.\ column resp.\ block have (exactly) two candidates left, and these two candidates coincide, then these two candidates are not candidates for the other cells in the row resp.\ column. XXX

\begin{figure}
\begin{sudoku}
|{\cell {}2{}45{}{}{}9}|{\cell {}2{}{}5{}789}|{\cell {}2{}45{}{}89}|{\cell {}234{}{}7{}{}}|{\cell {}234{}67{}{}}|{\cell {}{}34{}{}7{}{}}|{\cell {}{}{}{}56{}{}{}}|1|{\cell {}{}3{}45{}9}|.
|1|3|6|9|5|8|{\cell {}2{}{}{}{}7{}{}}|{\cell {}2{}{}{}{}7{}{}}|4|.
|{\cell {}2{}45{}{}{}9}|{\cell {}2{}{}5{}7{}9}|{\cell {}2{}45{}{}{}9}|{\cell 1234{}{}7{}{}}|{\cell {}234{}67{}{}}|{\cell 1{}34{}{}7{}{}}|{\cell {}{}{}{}56{}{}{}}|{\cell {}{}3{}{}6{}{}9}|8|.
|{\cell {}2{}45{}{}{}{}}|{\cell {}2{}{}5{}{}{}{}}|{\cell {}2345{}{}{}{}}|{\cell 1{}345{}78{}}|{\cell {}{}34{}78{}}|6|9|{\cell {}234{}{}7{}{}}|{\cell 1{}3{}5{}{}{}{}}|.
|7|1|{\cell {}{}345{}{}{}9}|{\cell {}{}345{}{}{}{}}|{\cell {}{}34{}{}{}{}9}|2|8|{\cell {}{}34{}6{}{}{}}|{\cell {}{}3{}56{}{}{}}|.
|8|6|{\cell {}2345{}{}{}9}|{\cell 1{}345{}7{}{}}|{\cell {}{}34{}{}7{}9}|{\cell 1{}345{}7{}9}|{\cell 12{}45{}7{}{}}|{\cell {}234{}{}7{}{}}|{\cell 1{}3{}5{}{}{}{}}|.
|3|{\cell {}{}{}{}5{}{}89}|{\cell {}{}{}{}5{}{}89}|{\cell {}{}{}45{}78{}}|1|{\cell {}{}{}45{}7{}9}|{\cell {}{}{}4{}6{}{}{}}|{\cell {}{}{}4{}6{}89}|2|.
|6|4|1|{\cell {}2{}{}5{}{}8{}}|{\cell {}2{}{}{}{}{}89}|{\cell {}{}{}{}5{}{}{}9}|3|{\cell {}{}{}{}{}{}{}89}|7|.
|{\cell {}2{}{}{}{}{}{}9}|{\cell {}2{}{}{}{}{}89}|7|6|{\cell {}234{}{}{}89}|{\cell {}{}34{}{}{}{}9}|{\cell 1{}{}4{}{}{}{}{}}|5|{\cell 1{}{}{}{}{}{}{}9}|.
\end{sudoku}
\caption{Example in Naked Pair - Before}
\label{fig:nakedpairb4}
\end{figure}

Naked pair is in the situation which two cells which shared a unit (such like a row, a column, or a block) both have \textbf{only} two same possibilities to be filled in as an answer. In the Figure \ref{fig:nakedpairb4} which is shown, candidates 5 and 6 are the \textbf{only} two possible candidates in the cell $(r_{1},\ c_{7})$ and $(r_{3},\ c_{7})$. This move is to help eliminating the impossibilities candidates 5 and 6 in other cells which shared a unit (a row, a column, or a block) as that two cells which each has \textbf{only} two same candidates in each. After the move, those cells which in the same unit will have candidates 5 and 6 excluded which has a result as Figure \ref{fig:nakedpairaf}. 

\begin{figure}
\begin{sudoku}
|{\cell {}2{}4{}{}{}{}9}|{\cell {}2{}{}{}{}789}|{\cell {}2{}4{}{}{}89}|{\cell {}234{}{}7{}{}}|{\cell {}234{}67{}{}}|{\cell {}{}34{}{}7{}{}}|{\cell {}{}{}{}56{}{}{}}|1|{\cell {}{}3{}4{}{}9}|.
|1|3|6|9|5|8|{\cell {}2{}{}{}{}7{}{}}|{\cell {}2{}{}{}{}7{}{}}|4|.
|{\cell {}2{}45{}{}{}9}|{\cell {}2{}{}5{}7{}9}|{\cell {}2{}45{}{}{}9}|{\cell 1234{}{}7{}{}}|{\cell {}234{}67{}{}}|{\cell 1{}34{}{}7{}{}}|{\cell {}{}{}{}56{}{}{}}|{\cell {}{}3{}{}6{}{}9}|8|.
|{\cell {}2{}45{}{}{}{}}|{\cell {}2{}{}5{}{}{}{}}|{\cell {}2345{}{}{}{}}|{\cell 1{}345{}78{}}|{\cell {}{}34{}78{}}|6|9|{\cell {}234{}{}7{}{}}|{\cell 1{}3{}5{}{}{}{}}|.
|7|1|{\cell {}{}345{}{}{}9}|{\cell {}{}345{}{}{}{}}|{\cell {}{}34{}{}{}{}9}|2|8|{\cell {}{}34{}6{}{}{}}|{\cell {}{}3{}56{}{}{}}|.
|8|6|{\cell {}2345{}{}{}9}|{\cell 1{}345{}7{}{}}|{\cell {}{}34{}{}7{}9}|{\cell 1{}345{}7{}9}|{\cell 12{}4{}{}7{}{}}|{\cell {}234{}{}7{}{}}|{\cell 1{}3{}5{}{}{}{}}|.
|3|{\cell {}{}{}{}5{}{}89}|{\cell {}{}{}{}5{}{}89}|{\cell {}{}{}45{}78{}}|1|{\cell {}{}{}45{}7{}9}|{\cell {}{}{}4{}{}{}{}{}}|{\cell {}{}{}4{}6{}89}|2|.
|6|4|1|{\cell {}2{}{}5{}{}8{}}|{\cell {}2{}{}{}{}{}89}|{\cell {}{}{}{}5{}{}{}9}|3|{\cell {}{}{}{}{}{}{}89}|7|.
|{\cell {}2{}{}{}{}{}{}9}|{\cell {}2{}{}{}{}{}89}|7|6|{\cell {}234{}{}{}89}|{\cell {}{}34{}{}{}{}9}|{\cell 1{}{}4{}{}{}{}{}}|5|{\cell 1{}{}{}{}{}{}{}9}|.
\end{sudoku}
\caption{Example in Naked Pair - After}
\label{fig:nakedpairaf}
\end{figure}

\section{Naked Triplet}
\label{sec:Naked Triplet}

It is in the situation whilst three specific cells which are all in the same unit (a row, a column, or a block) and must have three common candidate possibilities and no others in these three cells. And an elimination for excluding these three candidates from other cells shared same unit (a row, a column, or a block) as these three cells could be made.

\begin{figure}
\begin{sudoku}
|2|4|{\cell {}{}{}{}{}678{}}|{\cell {}{}{}{}{}6789}|3|{\cell {}{}{}{}5{}78{}}|{\cell {}{}{}{}{}{}7{}9}|{\cell {}{}{}{}56{}8{}}|1|.
|5|9|{\cell {}{}{}{}{}678{}}|{\cell {}{}{}4{}678{}}|1|{\cell {}{}{}4{}{}78{}}|3|2|{\cell {}{}{}{}{}6{}8{}}|.
|{\cell {}{}{}{}{}{}78{}}|{\cell 1{}3{}{}67{}{}}|{\cell 1{}3{}{}678{}}|{\cell {}{}{}{}{}6789}|2|{\cell {}{}{}{}5{}78{}}|{\cell {}{}{}{}{}{}7{}9}|{\cell {}{}{}{}56{}8{}}|4|.
|3|5|2|1|4|6|8|9|7|.
|4|{\cell {}{}{}{}{}67{}{}}|{\cell {}{}{}{}{}67{}{}}|3|8|9|5|1|2|.
|1|8|9|5|7|2|6|4|3|.
|{\cell {}{}{}{}{}{}78{}}|2|{\cell {}{}{}45{}78{}}|{\cell {}{}{}4{}{}78{}}|9|3|1|{\cell {}{}{}{}{}378{}}|{\cell {}{}{}{}56{}8{}}|.
|6|{\cell {}{}3{}{}{}7{}{}}|{\cell {}{}34{}{}78{}}|{\cell {}2{}4{}{}78{}}|5|1|{\cell {}2{}4{}{}{}{}{}}|{\cell {}{}{}{}{}{}78{}}|9|.
|9|{\cell 1{}{}{}{}{}7{}{}}|{\cell 1{}{}45{}78{}}|{\cell {}2{}4{}{}78{}}|6|{\cell {}{}{}4{}{}78{}}|{\cell {}2{}4{}{}{}{}{}}|3|{\cell {}{}{}{}5{}{}8{}}|.
\end{sudoku}
\caption{Example in Naked Triplet - Before}
\label{fig:nakedtripletb4}
\end{figure}
In the Figure{fig:nakedtripletb4}, candidates 6, 7 and 8 are the only common number which appears in the cells $(r_{1},\ c_{3})$, $(r_{2},\ c_{3})$, and $(r_{5},\ c_{3})$ which has shared the same unit (column) here. Hence, the inference to make is those cells who also have candidates 6, 7 and 8 in the same unit (column $c_{3}$) could have these three candidates erased, because they are not longer possible to be the answer to be filled in. The consequence after elimination is the Figure{fig:nakedtripletaf}.

\begin{figure}
\begin{sudoku}
|2|4|{\cell {}{}{}{}{}678{}}|{\cell {}{}{}{}{}6789}|3|{\cell {}{}{}{}5{}78{}}|{\cell {}{}{}{}{}{}7{}9}|{\cell {}{}{}{}56{}8{}}|1|.
|5|9|{\cell {}{}{}{}{}678{}}|{\cell {}{}{}4{}678{}}|1|{\cell {}{}{}4{}{}78{}}|3|2|{\cell {}{}{}{}{}6{}8{}}|.
|{\cell {}{}{}{}{}{}78{}}|{\cell 1{}3{}{}67{}{}}|{\cell 1{}3{}{}678{}}|{\cell {}{}{}{}{}6789}|2|{\cell {}{}{}{}5{}78{}}|{\cell {}{}{}{}{}{}7{}9}|{\cell {}{}{}{}56{}8{}}|4|.
|3|5|2|1|4|6|8|9|7|.
|4|{\cell {}{}{}{}{}67{}{}}|{\cell {}{}{}{}{}67{}{}}|3|8|9|5|1|2|.
|1|8|9|5|7|2|6|4|3|.
|{\cell {}{}{}{}{}{}78{}}|2|{\cell {}{}{}45{}78{}}|{\cell {}{}{}4{}{}78{}}|9|3|1|{\cell {}{}{}{}{}378{}}|{\cell {}{}{}{}56{}8{}}|.
|6|{\cell {}{}3{}{}{}7{}{}}|{\cell {}{}34{}{}78{}}|{\cell {}2{}4{}{}78{}}|5|1|{\cell {}2{}4{}{}{}{}{}}|{\cell {}{}{}{}{}{}78{}}|9|.
|9|{\cell 1{}{}{}{}{}7{}{}}|{\cell 1{}{}45{}78{}}|{\cell {}2{}4{}{}78{}}|6|{\cell {}{}{}4{}{}78{}}|{\cell {}2{}4{}{}{}{}{}}|3|{\cell {}{}{}{}5{}{}8{}}|.
\end{sudoku}
\caption{Example in Naked Triplet - After}
\label{fig:nakedtripletaf}
\end{figure}

\section{Naked Quad}
\label{sec:Naked Quad}
It is in the situation while four cells in the same row, same column, or same block are having common four possibilities digits in. Thus, the result which could be inferred is those related rows, columns, and the block could remove the numbers which as same candidates in as that four cells.

\begin{figure}
\begin{sudoku}
|{\cell 1{}{}{}5{}{}{}{}}|9|4|{\cell 1{}{}{}5{}7{}{}}|3|6|2|{\cell 1{}{}{}{}{}78{}}|{\cell 1{}{}{}{}{}{}8{}}|.
|{\cell 12{}{}{}6{}{}{}}|7|{\cell 12{}{}{}6{}8{}}|{\cell 12{}{}{}{}{}{}9}|{\cell 12{}{}{}{}{}{}9}|{\cell {}{}{}{}{}{}{}89}|4|3|5|.
|{\cell 12{}{}5{}{}{}{}}|3|{\cell 12{}{}5{}{}8{}}|4|{\cell 12{}{}{}{}7{}{}}|{\cell {}{}{}{}5{}{}8{}}|{\cell {}{}{}{}67{}{}{}}|{\cell 1{}{}{}{}67{}9}|{\cell 1{}{}{}{}{}{}{}9}|.
|{\cell {}{}34{}6{}{}9}|{\cell {}{}{}4{}6{}{}{}}|{\cell {}{}3{}{}6{}{}9}|8|{\cell {}{}{}{}{}67{}9}|1|5|{\cell {}2{}{}{}{}7{}9}|{\cell {}23{}{}{}{}{}9}|.
|{\cell 1{}3{}56{}{}9}|2|{\cell 1{}3{}56{}{}9}|{\cell {}{}{}{}567{}9}|4|{\cell {}{}3{}5{}{}{}9}|{\cell {}{}{}{}{}{}78{}}|{\cell {}{}{}{}{}{}789}|{\cell {}{}3{}{}{}{}89}|.
|7|8|{\cell {}{}3{}5{}{}{}9}|{\cell {}2{}{}5{}{}{}9}|{\cell {}2{}{}{}{}{}{}9}|{\cell {}{}3{}5{}{}{}9}|1|4|6|.
|8|{\cell 1{}{}{}{}6{}{}{}}|{\cell {}23{}{}6{}{}{}}|{\cell 1{}3{}{}6{}{}{}}|5|7|9|{\cell 12{}{}{}6{}{}{}}|4|.
|{\cell {}2{}{}{}6{}{}9}|{\cell 1{}{}{}56{}{}{}}|7|{\cell 1{}{}{}{}6{}{}9}|{\cell 1{}{}{}{}6{}89}|4|3|{\cell 12{}{}56{}8{}}|{\cell 12{}{}{}{}{}8{}}|.
|{\cell {}{}34{}6{}{}9}|{\cell 1{}{}456{}{}{}}|{\cell {}{}3{}{}6{}{}9}|{\cell 1{}3{}{}6{}{}9}|{\cell 1{}{}{}{}6{}89}|2|{\cell {}{}{}{}{}6{}8{}}|{\cell 1{}{}{}56{}8{}}|7|.
\end{sudoku}
\caption{Example in Naked Quad}
\label{fig:nakedquad}
\end{figure}

In the Figure ~\ref{fig:nakedquad}, in block $b_{5}$, cell $(r_{5},\ c_{5})$, $(r_{6},\ c_{4})$, $(r_{6},\ c_{5})$, and$(r_{6},\ c_{6})$ have candidates $n_{2},\ n_{3},\ n_{5},\ and\ n_{9}$ appear commonly in these four cells in one block ($b_{5}$). Consequently, candidates $n_{5}$ and $n_{9}$ in other cells in the block $b_{5}$ will be excluded because these two candidates will not longer possible to be used in the cells anymore according to the inference has been made before. 

\section{Hidden Single}
\label{sec:Hidden Single}
Hidden single does not show the solution obviously like what naked single does (showing one candidate only in the cell). It is a skill to find a unique specific digit to write in only if the player looks closely. However, it sometimes replaced by Cross-Hatching which could be even more easily applied to find the answer without the pencilmarks step. 

There are 3 formulas in Hidden Single, which are in row, in column,and in block, respectively.
\begin{itemize}
\item $ \forall r  \forall n\  \{ \exists !c \  candidate(n, r, c) => value(n, r, c)\}$
\item $ \forall c  \forall n\  \{ \exists !r \  candidate(n, r, c) => value(n, r, c)\}$
\item $ \forall b  \forall n\  \{ \exists !s \  candidate[n, b,s] => value[n, b, s]\}$
\end{itemize}

\begin{figure}
\begin{sudoku}
|{\cell {}23{}{}{}{}8{}}|{\cell {}234{}{}78{}}|{\cell {}{}34{}{}78{}}|{\cell {}{}{}{}{}{}78{}}|{\cell 1{}3{}{}{}7{}{}}|{\cell 123{}{}{}78{}}|5|{\cell 12{}4{}6789}|{\cell 123{}{}678{}}|.
|1|6|{\cell {}{}345{}78{}}|9|{\cell {}{}3{}{}{}7{}{}}|{\cell {}23{}5{}78{}}|{\cell {}234{}{}78{}}|{\cell {}2{}4{}{}78{}}|{\cell {}23{}{}{}78{}}|.
|{\cell {}23{}5{}{}8{}}|{\cell {}23{}5{}78{}}|9|{\cell {}{}{}{}5{}78{}}|6|4|{\cell {}23{}{}{}78{}}|{\cell 12{}{}{}{}78{}}|{\cell 123{}{}{}78{}}|.
|{\cell {}23{}56{}89}|{\cell {}23{}5{}789}|{\cell 1{}3{}5678{}}|{\cell {}{}{}{}5678{}}|{\cell 1{}{}{}{}{}7{}9}|{\cell 1{}{}{}5678{}}|{\cell {}23{}{}6789}|{\cell {}2{}{}{}6789}|4|.
|4|{\cell {}{}3{}5{}789}|{\cell {}{}3{}5678{}}|{\cell {}{}{}{}5678{}}|2|{\cell {}{}{}{}5678{}}|1|{\cell {}{}{}{}{}6789}|{\cell {}{}3{}{}678{}}|.
|{\cell {}2{}{}{}6{}89}|{\cell {}2{}{}{}{}789}|{\cell 1{}{}{}{}678{}}|3|{\cell 1{}{}4{}{}7{}9}|{\cell 1{}{}{}{}678{}}|{\cell {}2{}{}{}678{}}|5|{\cell {}2{}{}{}678{}}|.
|{\cell {}{}3{}56{}{}{}}|{\cell {}{}345{}{}{}{}}|2|{\cell {}{}{}4{}67{}{}}|8|9|{\cell {}{}{}4{}67{}{}}|{\cell 1{}{}4{}67{}{}}|{\cell 1{}{}{}567{}{}}|.
|{\cell {}{}{}{}{}6{}89}|1|{\cell {}{}{}4{}6{}8{}}|2|5|{\cell {}{}{}{}{}67{}{}}|{\cell {}{}{}4{}678{}}|3|{\cell {}{}{}{}{}678{}}|.
|7|{\cell {}{}345{}{}8{}}|{\cell {}{}3456{}8{}}|1|{\cell {}{}34{}{}{}{}{}}|{\cell {}{}3{}{}6{}{}{}}|{\cell {}2{}4{}6{}8{}}|{\cell {}2{}4{}6{}8{}}|9|.
\end{sudoku}
\label{fig:hiddensingle}
\caption{Example in Hidden Single}
\end{figure}

In this case, they are four possibilities could be filled into this highlighted cell. Nevertheless, if look carefully, candidate 4 is the only number that appears once only in this block. Hence, candidate 4 is the answer for this cell undoubtedly.
Actually, this cell could be solved by using either hidden single or Cross-Hatching. Looking at candidate 4 by using Cross-Hatching, all the possibilities of candidate 4 in the middle block here has only one cell left after crossing out the impossibility according to the rule that the candidate could only be used once in each row, column and the block.


\section{Hidden Pair}
\label{sec:Hidden Pair}
The prerequisite for Hidden Pair is, that two of all the possible candidates are only appeared in two cells in one unit (either row, column, or block). If the condition is hold, then all other candidates in these two cells could be excluded surely. 

\begin{figure}
\begin{sudoku}
|{\cell {}234{}{}{}89}|{\cell {}{}345{}{}89}|7|{\cell {}23{}5{}{}{}9}|6|1|{\cell {}{}{}45{}{}{}9}|{\cell {}2{}{}{}{}{}89}|{\cell {}2{}4{}{}{}89}|.
|{\cell {}234{}{}{}{}9}|{\cell {}{}345{}{}{}9}|{\cell {}{}345{}{}{}9}|{\cell {}23{}5{}{}{}9}|{\cell {}2{}{}5{}{}{}9}|8|{\cell 1{}{}45{}7{}9}|6|{\cell 12{}4{}{}7{}9}|.
|1|{\cell {}{}{}{}5{}{}89}|6|{\cell {}2{}{}5{}{}{}9}|7|4|{\cell {}{}{}{}5{}{}{}9}|{\cell {}2{}{}{}{}{}89}|3|.
|{\cell {}{}{}4{}{}7{}9}|2|{\cell 1{}{}45{}{}{}9}|{\cell {}{}{}45{}{}{}9}|{\cell {}{}{}{}5{}{}{}9}|3|8|{\cell 1{}{}{}{}{}7{}9}|6|.
|{\cell {}{}34{}{}789}|{\cell 1{}3456{}89}|{\cell 1{}345{}{}{}9}|{\cell {}2{}45{}{}89}|{\cell {}2{}{}5{}{}{}9}|{\cell {}{}{}{}{}67{}{}}|{\cell 1{}34{}{}{}{}9}|{\cell 123{}{}{}7{}9}|{\cell 12{}4{}{}{}{}9}|.
|{\cell {}{}34{}{}789}|{\cell {}{}34{}6{}89}|{\cell {}{}34{}{}{}{}9}|{\cell {}2{}4{}{}{}89}|1|{\cell {}{}{}{}{}67{}{}}|{\cell {}{}34{}{}{}{}9}|5|{\cell {}2{}4{}{}{}{}9}|.
|{\cell {}{}{}4{}{}{}{}9}|{\cell 1{}{}4{}{}{}{}9}|8|7|3|2|6|{\cell 1{}{}{}{}{}{}{}9}|5|.
|6|{\cell {}{}3{}{}{}{}{}9}|2|1|8|5|{\cell {}{}3{}{}{}7{}9}|4|{\cell {}{}{}{}{}{}7{}9}|.
|5|7|{\cell 1{}3{}{}{}{}{}{}}|6|4|9|2|{\cell 1{}3{}{}{}{}8{}}|{\cell 1{}{}{}{}{}{}8{}}|.
\end{sudoku}
\caption{Example in Hidden Pair}
\label{fig:hiddenPair}
\end{figure}

For instance, in the Figure {fig:hiddenPair}. Candidates $n_{1}$ and $n_{7}$ in $(r_{2},\ c_{7})$ are restricted to these two cells only. Once the prerequisite is satisfied, then the all other cells who has either candidates $n_{1}$ and $n_{7}$ in the cell could have these candidates cross out from these two cells.


\section{Hidden Triplet}
\label{sec:Hidden Triplet}
Basically, it is evolved from hidden single and hidden pair. The prerequisite is three of all the possible candidates in the cell are only happened in only three cells in one unit (either row, column, or block).

\begin{table}
\setlength{\tabcolsep}{2.5pt}
\renewcommand{\arraystretch}{1.6}
\hspace{1em}
\begin{center}
\begin{tabular}{ | r| r| r| r| r| r| r| r| r | }
\hline
{\cell {}{}{}{}{}6{}{}9} &{\cell {}{}{}{}{}6{}8{}} & 1 &{\cell {}2{}{}56{}8{}}  & {\cell {}2{}{}5{}789} & 3 & {\cell {}{}{}{}5{}7{}9} & 4 & {\cell {}{}{}{}{}6{}{}9} \\
\hline
\end{tabular}
\end{center}
\caption{Example in Hidden Triplet}
\label{tab:hiddentriplet}
\end{table}

In the Table \ref{tab:hiddentriplet}, candidate $n_{2}$, $n_{5}$, and $n_{7}$, are only appeared in $c_{4}$, $c_{5}$, and $c_{7}$. Hence, all the candidates apart from $n_{2}$, $n_{5}$, and $n_{7}$ could the be eliminated from these three cells.

\section{Hidden Quad}
\label{sec:Hidden Quad}
This is the circumstance which has four digits candidate found to be restricted to specific 4 cells which has equal symmetry. Then other possibilities digits in these four cells should be removed afterwards. It is aims to help player to eliminate impossibilities in the cells by hidden quad skill. 

\begin{figure}
\begin{sudoku}
|6|3|2|1|4|5|9|7|8|.
|8|1|{\cell {}{}{}{}5{}7{}{}}|{\cell {}{}3{}{}6{}{}{}}|9|{\cell {}2{}{}{}{}7{}{}}|{\cell {}23{}56{}{}{}}|{\cell {}23{}5{}{}{}{}}|4|.
|{\cell {}{}{}{}5{}{}{}9}|4|{\cell {}{}{}{}5{}7{}9}|{\cell {}{}3{}{}6{}{}{}}|8|{\cell {}2{}{}{}{}7{}{}}|{\cell {}23{}56{}{}{}}|1|{\cell {}23{}56{}{}{}}|.
|{\cell {}{}3{}{}{}{}{}9}|{\cell {}2{}{}{}{}7{}{}}|{\cell {}{}{}4{}{}{}{}9}|8|5|{\cell 1{}3{}{}{}{}{}{}}|{\cell 12{}4{}67{}{}}|{\cell {}2{}{}{}{}{}{}9}|{\cell 12{}{}{}67{}{}}|.
|1|6|{\cell {}{}{}{}5{}{}89}|2|7|4|{\cell {}{}3{}5{}{}8{}}|{\cell {}{}3{}5{}{}89}|{\cell {}{}3{}5{}{}{}{}}|.
|{\cell {}{}3{}5{}{}{}{}}|{\cell {}2{}{}{}{}7{}{}}|{\cell {}{}{}45{}{}8{}}|9|6|{\cell 1{}3{}{}{}{}{}{}}|{\cell 12{}45{}78{}}|{\cell {}2{}{}5{}{}8{}}|{\cell 12{}{}5{}7{}{}}|.
|4|8|1|5|2|9|{\cell {}{}3{}{}{}7{}{}}|6|{\cell {}{}3{}{}{}7{}{}}|.
|7|5|3|4|1|6|{\cell {}2{}{}{}{}{}8{}}|{\cell {}2{}{}{}{}{}8{}}|9|.
|2|9|6|7|3|8|{\cell 1{}{}{}5{}{}{}{}}|4|{\cell 1{}{}{}5{}{}{}{}}|.
\end{sudoku}
\caption{Example in Hidden Quad}
\label{fig:hiddenquad}
\end{figure}

In the example here, candidates 1, 4, 6 and 7 are the only 4 possibilities could still stay in these four cells, all the others will no longer existed after this technique has applied.

\section{X-wing}
\label{sec:X-wing}
It is an advanced technique to apply on when 4 cells are fully symmetric and have 2 candidates for a given digit in the symmetric cells of parallel rows. All other candidates for that digit from these 2 columns could be eliminated. On the other hand, all other candidates for that digit from these 2 rows could be removed if the situation is reversed.

\begin{figure}
\begin{sudoku}
| |*| | | | | |*| |.
|*|X|*|*|*|*|*|X|*|.
| |*| | | | | |*| |.
| |*| | | | | |*| |.
| |*| | | | | |*| |.
| |*| | | | | |*| |.
| |*| | | | | |*| |.
|*|X|*|*|*|*|*|X|*|.
| |*| | | | | |*| |.
\end{sudoku}
\label{fig:xwingmodel}
\end{figure}

The figure above is the condition look which x-wing techniques consider could be applied for helping eliminating impossible candidates from cells. Those 4 cells which has symbol X in represents the cells to be considered, and symbol * represents the cell is in the same column, or same row. 

In the figure below, candidate 6 is the common number in $(r_{2}, c_{2})$, $(r_{2}, c_{8})$, $(r_{8}, c_{2})$, and $(r_{8}, c_{8})$ whom are consecutive in opposite angle. To apply X-Wing on, there are only one solution will be the correct choice between two mutually exclusive possibilities for candidate 6 to fit in, either has candidate 6 to be filled in 
\begin{enumerate}
\item $(r_{2}, c_{2})$ and $(r_{8}, c_{8})$,
\item or in $(r_{2}, c_{8})$, and $(r_{8}, c_{2})$.
\end{enumerate}
If the first solution is taken, then cell$(r_{2}, c_{8})$ will have candidate 9 filled in, and cell$(r_{8}, c_{2})$ will have candidate 4 filled in. On the contrary, if the second one is chosen, then the solution for celll$(r_{2}, c_{2})$ will be 9 and for cell$(r_{8}, c_{8})$ will be 4. However, none of them could be decided. the only thing could be done is to eliminate candidate number 4, 6, and 9 in any cells which in the same row or same column as $(r_{2}, c_{2})$, $(r_{2}, c_{8})$, $(r_{8}, c_{2})$, and $(r_{8}, c_{8})$.

\begin{figure}
\begin{sudoku}
|{\cell {}{}{}{}{}{}{}{}{}}|*|{\cell {}{}{}{}{}{}{}{}{}}|{\cell {}{}{}{}{}{}{}{}{}}|{\cell {}{}{}{}{}{}{}{}{}}|{\cell {}{}{}{}{}{}{}{}{}}|{\cell {}{}{}{}{}{}{}{}{}}|*|{\cell {}{}{}{}{}{}{}{}{}}|.
|*|{\cell {}{}{}{}{}{6}{}{}{9}}|*|*|*|*|*|{\cell {}{}{}{}{}{6}{}{}{9}}|*|.
|{\cell {}{}{}{}{}{}{}{}{}}|*|{\cell {}{}{}{}{}{}{}{}{}}|{\cell {}{}{}{}{}{}{}{}{}}|{\cell {}{}{}{}{}{}{}{}{}}|{\cell {}{}{}{}{}{}{}{}{}}|{\cell {}{}{}{}{}{}{}{}{}}|*|{\cell {}{}{}{}{}{}{}{}{}}|.
|{\cell {}{}{}{}{}{}{}{}{}}|*|{\cell {}{}{}{}{}{}{}{}{}}|{\cell {}{}{}{}{}{}{}{}{}}|{\cell {}{}{}{}{}{}{}{}{}}|{\cell {}{}{}{}{}{}{}{}{}}|{\cell {}{}{}{}{}{}{}{}{}}|*|{\cell {}{}{}{}{}{}{}{}{}}|.
|{\cell {}{}{}{}{}{}{}{}{}}|*|{\cell {}{}{}{}{}{}{}{}{}}|{\cell {}{}{}{}{}{}{}{}{}}|{\cell {}{}{}{}{}{}{}{}{}}|{\cell {}{}{}{}{}{}{}{}{}}|{\cell {}{}{}{}{}{}{}{}{}}|*|{\cell {}{}{}{}{}{}{}{}{}}|.
|{\cell {}{}{}{}{}{}{}{}{}}|*|{\cell {}{}{}{}{}{}{}{}{}}|{\cell {}{}{}{}{}{}{}{}{}}|{\cell {}{}{}{}{}{}{}{}{}}|{\cell {}{}{}{}{}{}{}{}{}}|{\cell {}{}{}{}{}{}{}{}{}}|*|{\cell {}{}{}{}{}{}{}{}{}}|.
|{\cell {}{}{}{}{}{}{}{}{}}|*|{\cell {}{}{}{}{}{}{}{}{}}|{\cell {}{}{}{}{}{}{}{}{}}|{\cell {}{}{}{}{}{}{}{}{}}|{\cell {}{}{}{}{}{}{}{}{}}|{\cell {}{}{}{}{}{}{}{}{}}|*|{\cell {}{}{}{}{}{}{}{}{}}|.
|*|{\cell {}{}{}{4}{}{6}{}{}{}}|*|*|*|*|*|{\cell {}{}{}{4}{}{6}{}{}{}}|*|.
|{\cell {}{}{}{}{}{}{}{}{}}|*|{\cell {}{}{}{}{}{}{}{}{}}|{\cell {}{}{}{}{}{}{}{}{}}|{\cell {}{}{}{}{}{}{}{}{}}|{\cell {}{}{}{}{}{}{}{}{}}|{\cell {}{}{}{}{}{}{}{}{}}|*|{\cell {}{}{}{}{}{}{}{}{}}|.
\end{sudoku}
\caption{Example in X-wing}
\label{fig:xwing}
\end{figure}



\section{XY-wing}
\label{sec:XY-wing}


\section{XYZ-wing}
\label{sec:XYZ-wing}


\section{WXYZ-wing}
\label{sec:WXYZ-wing}


\section{Swordfish}
\label{sec:Swordfish}
Swordfish is similar approach which extended from X-wing technique, but has really rare opportunity to be applied on. The concept is like the X-Wing, a value in one position will forced the other in the same row or same column not to have the same value filled in. Will have two possibilities could be fulfilled eventually. Nevertheless, at the moment, could only help on making decision to eliminate candidates in other cells in the same columns and rows. It is used, when there are three different columns, $c_{1}, c_{2}, c_{3}$, three different rows, $r_{1}, r_{2}, r_{3}$ and a number n in a situation like :
\begin{enumerate}
\item $r_{1}\ has\ c_{1}\ and\ c_{2}\ among\ its\ candidates,$
\item $r_{2}\ has\ c_{2}\ and\ c_{3}\ among\ its\ candidates,$
\item $r_{3}\ has\ c_{3}\ and\ c_{1}\ among\ its\ candidates,$
\item $and\ non\ of\ r_{1},\ r_{2}\, and\ r_{3}\ has\ other\ candidates\ in\ c_{1},\ c_{2},\ and\ c_{3}.$
\end{enumerate}
All the candidates for specific number n are occurred in only three specific columns, thus, all candidates for number n which not in these three columns could be eliminated immediately. Each defined rows could have two or three candidates for specific number n.\\
The formula of swordfish is
\[ \forall n \forall 3 \neq r_{1} r_{2} r_{3}\ \forall 3 \neq c_{1} c_{2} c_{3}\]
\begin{figure}
\begin{sudoku}
| |*| | |*| | |*| |.
| |X| | |X| | |X| |.
| |*| | |*| | |*| |.
| |*| | |*| | |*| |.
| |X| | |X| | |X| |.
| |*| | |*| | |*| |.
| |*| | |*| | |*| |.
| |X| | |X| | |X| |.
| |*| | |*| | |*| |.
\end{sudoku}
\caption{Swordfish: 3 rows in 3 columns}
\label{fig:swordfish33}
\end{figure}

\begin{figure}
\begin{sudoku}
| |*| | |*| | |*| |.
| | | | |X| | |X| |.
| |*| | |*| | |*| |.
| |*| | |*| | |*| |.
| |X| | |X| | | | |.
| |*| | |*| | |*| |.
| |*| | |*| | |*| |.
| |X| | | | | |X| |.
| |*| | |*| | |*| |.
\end{sudoku}
\caption{Swordfish: 2 rows in 3 columns}
\label{fig:swordfish23}
\end{figure}


\section{Intersection Removal}
\label{sec:Intersection Removal}
Sometimes, advanced techniques is needed to be used on supporting more difficult puzzle whilst both naked single and hidden single are not able to find out any way out. The intersection removal technique provides an advanced elimination skill to exclude more impossibilities out of the row, or the column which are in one object block. The prerequisites for this techniques to be utilized are, 
\begin{enumerate}
\item All the cells in one block holds a specific candidate just right in the same row or column or
\item All the cells in one row or one column hold a possible specific candidate just right in the same block.
\end{enumerate}
If the first prerequisite is tenable, then the specific candidate could be excluded from all other cells in the same row or column. And if the second prerequisite is tenable, the the specific candidate could be excluded from all other cells in the same block. 

\begin{figure}
\begin{sudoku}
|2|{\cell {}{}{}{}56789}|{\cell {}{}{}{}5{}78{}}|{\cell {}{}3456{}{}9}|{\cell {}{}345{}789}|{\cell {}{}{}456789}|1|{\cell {}{}{}4{}67{}{}}|{\cell {}{}{}{}{}67{}9}|.
|1|4|{\cell {}{}{}{}5{}7{}{}}|{\cell {}{}{}{}56{}{}9}|2|{\cell {}{}{}{}567{}9}|{\cell {}{}{}{}{}67{}9}|8|3|.
|{\cell {}{}{}{}{}67{}9}|{\cell {}{}{}{}{}6789}|3|{\cell {}{}{}4{}6{}{}9}|1|{\cell {}{}{}4{}6789}|5|{\cell {}2{}4{}67{}{}}|{\cell {}2{}{}{}67{}9}|.
|{\cell {}{}345{}7{}{}}|{\cell 123{}5{}7{}{}}|{\cell {}2{}45{}7{}{}}|{\cell {}{}3456{}{}9}|{\cell {}{}345{}{}89}|{\cell {}{}{}456{}89}|{\cell {}{}{}4{}6789}|{\cell 12{}4{}67{}{}}|{\cell 12{}{}{}6789}|.
|{\cell {}{}34{}{}{}{}{}}|{\cell 123{}{}{}{}{}{}}|6|7|{\cell {}{}34{}{}{}89}|{\cell {}{}{}4{}{}{}89}|{\cell {}{}{}4{}{}{}89}|5|{\cell 12{}{}{}{}{}89}|.
|8|{\cell {}{}{}{}5{}7{}{}}|9|2|{\cell {}{}{}45{}{}{}{}}|1|3|{\cell {}{}{}4{}67{}{}}|{\cell {}{}{}{}{}67{}{}}|.
|{\cell {}{}{}4567{}9}|{\cell {}{}{}{}56789}|{\cell {}{}{}45{}78{}}|{\cell {}{}{}45{}{}{}9}|{\cell {}{}{}45{}7{}9}|3|2|{\cell 1{}{}{}{}67{}{}}|{\cell 1{}{}{}5678{}}|.
|{\cell {}{}{}4567{}9}|{\cell {}{}{}{}567{}9}|1|8|{\cell {}{}{}45{}7{}9}|2|{\cell {}{}{}{}{}67{}{}}|3|{\cell {}{}{}{}567{}{}}|.
|{\cell {}{}3{}5{}7{}{}}|{\cell {}23{}5{}78{}}|{\cell {}2{}{}5{}78{}}|1|6|{\cell {}{}{}{}5{}7{}{}}|{\cell {}{}{}{}{}{}78{}}|9|4|.
\end{sudoku}
\caption{Example in Intersection Removal}
\label{fig:intersectionremoval}
\end{figure}

For example, in \ref{fig:intersectionremoval} candidate $n_{4}$ happens to appear in both $(r_{1},\ c_{8})$ and $(r_{3},\ c_{8})$ in block $b_{3}$ which just right in the same column $c_{8}$, hence, this candidate $n_{4}$ then could be excluded from all other cells $(r_{4},\ c_{8})$ and $(r_{6},\ c_{8})$ in the same column $c_{8}$.


\chapter{Conclusion}
\label{sec:Conclusion}
The Sudoku puzzle might be solved by only one technique in some simple case Nevertheless, usually two or more are commonly needed in dealing general problem such like using both naked single and hidden single. The number of techniques used will increase according to the difficulty of the puzzle. The harder problem needs more complex technique to support it.
These special approaches for solving puzzle are simple and in fact popularly used in general problem solving. They are efficient to support player to eliminate the impossibilities in the cells, and leave the rest possibilities to get to the final solution.


\bibliographystyle{plain}

\bibliography{Sudoku}

\end{document}
